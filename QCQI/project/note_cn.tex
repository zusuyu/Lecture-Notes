%\documentclass[UTF-8]{ctexart}
\documentclass[8pt]{article}
\usepackage[UTF8]{ctex}
\usepackage[a4paper]{geometry}

\usepackage{graphicx}
\usepackage{subfigure}
\usepackage{amsmath}
\usepackage{tabularx}
\usepackage{color}
\usepackage{hyperref}
\usepackage{ulem}
\usepackage{multirow}
\usepackage[cache=false]{minted}
\hypersetup{
	colorlinks=true,
	linkcolor=blue
}
\documentclass[8pt]{article}
\usepackage[UTF8]{ctex}
\usepackage[a4paper]{geometry}
\usepackage{geometry}
\usepackage{appendix}
\geometry{a4paper,centering,scale=0.8}
\geometry{left=1.5cm, right=1.5cm, top=2.0cm, bottom=2.0cm}
\usepackage[format=hang,font=small,textfont=it]{caption}
\usepackage[nottoc]{tocbibind}

\usepackage{algorithm}
\usepackage{algorithmicx}
\usepackage{algpseudocode}
\usepackage{amssymb}
\usepackage{qcircuit}
\usepackage{fancyhdr}
\usepackage{cleveref}

\def \ket#1{|#1 \rangle}
\def \bra#1{\langle #1|}
\def \braket#1#2{\langle #1|#2\rangle}



\usepackage{tikz}  
\usetikzlibrary{arrows.meta}%画箭头用的包

\makeatletter
\def\@maketitle{%
	\newpage
	\begin{center}%
		\let \footnote \thanks
		{\LARGE \@title \par}%
		\vskip 1.5em%
		{\large
			\lineskip .5em%
			\begin{tabular}[t]{c}%
				\@author
			\end{tabular}\par}%
		\vskip 1em%
		{\large \@date}%
	\end{center}%
	\par
	\vskip 1.5em}
\makeatother

\newtheorem{theorem}{定理}
\newtheorem{lemma}{引理}
\newtheorem{definition}{定义}

\newcommand*{\circled}[1]{\lower.7ex\hbox{\tikz\draw (0pt, 0pt)%
		circle (.5em) node {\makebox[1em][c]{\small #1}};}}

\title{\heiti\zihao{1} Efficient generation of Clifford circuits}
\author{\kaishu\zihao{-3} 周书予\\2000013060@stu.pku.edu.cn}

%\date{\today}

\begin{document}\small
	
 \pagestyle{fancy}
	\lhead{Quantum Computing and Quantum Information, 2021 Fall}
	\chead{}
	\rhead{Efficient generation of Clifford circuits}
	
	
\crefname{theorem}{定理}{定理}
\crefname{lemma}{引理}{引理}
\crefname{figure}{图}{图}
\crefname{table}{表}{表}	
\maketitle

\section{引言}
Clifford 线路是由 Hadamard 门 $H$、相位门 $S$ 以及 controlled-NOT 门 $\text{CNOT}$ 组成的量子线路,这三个门被称为 Clifford 门,而由 Clifford 线路实现的算符被称作 Clifford 算符\footnote{还存在一种别的对于Clifford算符的定义方式,将在稍后提及。}。可以验证所有 Clifford 算符构成一个乘法群,我们把这个群称作 Clifford group,记作$\mathcal C_n$,下标$n$表示作用在$n$个量子比特上。

这次作业的主要工作是研究 Clifford group 的内部结构。\cref{partA} 将会证明任意作用在$n$个量子比特上的 Clifford 算符都可以被$O(n^2)$个 Clifford 门组合实现,这意味着任意 Clifford 算符都具有实现高效性;在\cref{partB} 中我们引入了Canonical form这一记号,并证明它与 Clifford 算符之间存在一一对应;这是卓有意义的,因为双射的存在将允许我们对 Clifford 线路进行随机采样——我们只需要随机采样Canonical form就可以了!这部分工作会在\cref{partC} 中被介绍。

\section{第一部分}\label{partA}

在展开证明前我们先引入一些记号。记$\mathcal P_n$表示$n$个量子比特上的 Pauli group,也就是$\{I, X, Y, Z\}^{\otimes n}$,乘上$\pm 1, \pm i$作为global phase。乘上 global phase 的目的是让群对乘法封闭。可以验证Pauli group $\mathcal P_n$是$\mathcal C_n$的一个子群。

定义$\mathcal P_n$的正规化子(normalizer) $\mathcal N(\mathcal P_n)$为所有满足$\forall g \in \mathcal P, UgU^{\dagger} \in \mathcal{P}_n$的幺正算符$U$构成的集合。通过验证$\mathcal N(\mathcal P_n)$对乘法以及求逆封闭,我们可以知道$\mathcal N(\mathcal P_n)$构成$n$阶幺正群的一个子群。

为了完成证明,我们需要证明如下两个引理:

\begin{lemma}
	任意 Clifford 算符都是 $\mathcal N(\mathcal{P}_n)$中的元素。
	\label{lemma1-1}
\end{lemma}
\begin{lemma}
	任意$\mathcal N(\mathcal{P}_n)$中的算符$U$都可以被$O(n^2)$个 Clifford 门实现,至多差一个global phase。
	\label{lemma1-2}
\end{lemma}

根据以上两个引理可以立即得到,任意 Clifford 算符都可以被$O(n^2)$个 Clifford 门实现,至多差一个global phase。实际上“至多差一个global phase”的修辞是可以去掉的,因为可以验证 Clifford group里中包含的global phase有且仅有$\{\text e^{\pi i k/ 4}: k \in [0, 7]\}$这$8$种(其中$(SH)^3 = \text e^{\pi i / 4}I$可以作为生成元),因此Clifford group中任意global phase均可以通过$O(1)$个Clifford门实现。

值得一提的是,部分文章\cite{paper2}对Clifford算符的定义方式就是$\mathcal N(\mathcal P_n) / U(1)$,即$\mathcal P_n$的正规化子忽略掉global phase。两种定义在global phase的处理上是有所不同的,这会引起一些描述上的偏差,比如说下文中将在忽略global phase的基础上叙述$|\mathcal C_n| = 2^{n^2 + 2n}\prod_{i=1}^{n}(4^i-1)$,而如果考虑global phase的话,其结果还需要乘$8$。

\subsection{\cref{lemma1-1} 的证明}

由于任何Clifford算符都可以写成Hadamard门、相位门以及CNOT门乘积的形式,我们只需要证明这三个Clifford门属于$\mathcal N(\mathcal P_n)$就行了,具体地我们要对于$U \in \{H, S, \text{CNOT}\}$,证明$\forall g \in \mathcal P_n, UgU^{\dagger} \in \mathcal P_n$。

这项工作其实非常简单,只需要列举所有情况 (见\cref{tab1}) 验证就可以了。表格里只展示了$UXU^{\dagger}$和$UZU^{\dagger}$的结果,因为$UYU^{\dagger} = i(UXU^{\dagger})(UZU^{\dagger})$可以被前两者唯一确定。


\subsection{\cref{lemma1-2} 的证明}

这部分的证明就比较麻烦了。总体的思路是对量子比特数$n$作归纳(这是一种极其不直观的方法),分为三步:

\begin{itemize}
	\item 第一步:证明任意$U \in \mathcal N(\mathcal P_1)$都可以由$O(1)$个Hadamard门与相位门实现;
	\item 第二步:假设任意$U' \in \mathcal N(\mathcal P_n)$都可以由$O(n^2)$个Clifford门实现,对于任何满足$UZ_1U^{\dagger} = X_1 \otimes g$以及$UX_1U^{\dagger} = Z_1 \otimes g'$的$U \in \mathcal N(\mathcal P_{n+1})$,证明它都可以由$O(n^2)$个Clifford门实现;
	\item 第三步:把第二步的结果推广,证明不做任何限制的$U \in \mathcal N(\mathcal P_{n+1})$都可以由$O(n^2)$个Clifford门实现。
\end{itemize}

上面的三条叙述都还需要加上“至多差一个global phase”。

由于这部分的证明有点冗长,并且与接下来将要讨论的内容关系不大,因此就全部搬到附录里了,详见\cref{proof1-2}。

\section{第二部分}\label{partB}

以下的内容都是这篇文章\cite{paper1}中介绍的。在引入所谓Canonical form之前,我们先研究一下Clifford group $\mathcal C_n$的子群结构。为了行文方便,在接下来的所有叙述中,我们均忽略global phase\footnote{严谨地讲,是始终只考虑$\mathcal C_n$及其子群在商掉$U(1)$后得到的商群。}。下表中列出了$\mathcal C_n$的一些子群以及对应的生成元集合:

\begin{center}
	\begin{tabular}{c|c|c}
		记号 & 名称 & 生成元集合\\
		\hline
		$\mathcal C_n$ & Clifford group & $H, S, \text{CNOT}$\\
		\hline
		$\mathcal F_n$ & Hadamard-free group & $X, S, \text{CNOT}, \text{CZ}$\\
		\hline
		$\mathcal B_n$ & Borel group & $X, S, \text{CNOT}^{\downarrow}, \text{CZ}$\\
		\hline
		$\mathcal S_n$ & Symmetric group & $\text{SWAP}$\\
		\hline
		$\mathcal P_n$ & Pauli group & $X, Z$\\
	\end{tabular}
\end{center}

其中$\text{CNOT}$和$\text{CZ}$分别表示 Controlled-$X$ 门与 Controlled-$Z$ 门。

注意到$\mathcal C_n$的生成元集合其实也可以写成$\{X, Z, H, S, \text{CNOT}, \text{CZ}\}$,这是因为$Z = S^2, X = HZH, \text{CZ} = H_2\cdot\text{CNOT}\cdot H_2$。这样写就能明显地看出表中的所有群都是$\mathcal C_n$的子群。 

考虑这样一件事情:Hadamard门是(在computational basis下)唯一会产生叠加态的Clifford门,这说明Hadamard-free group $\mathcal F_n$中的任何算符都不能产生任何叠加,即只能把基矢映成基矢,因而可以被表示成如下的形式:
\begin{equation}
F\ket{x} = i^{x^{\textbf T}\Gamma x}O\ket{\Delta x}
\label{equ1}
\end{equation}
其中$x \in \mathbb F_2^{n}, \Gamma, \Delta \in \mathbb F_2^{n \times n}, O \in \mathcal P_n$,所有矩阵乘法均在$\mathbb F_2$下进行。$\Delta$必须是可逆的,它表示了$n$个量子比特之间可能存在的纠缠,而$\Gamma$的唯一作用则是确定相位\footnote{这个相位是关于$x$的函数,因此并不是global phase。},它必须是对称的,这是因为$i^{x^{\textbf T}\Gamma x} = \prod\limits_{i=1}^{n}i^{x_i\Gamma_{ii}}\prod\limits_{1 \le i < j \le n}i^{x_ix_j(\Gamma_{ij} + \Gamma_{ji})}$,一旦$\Gamma_{ij} \neq \Gamma_{ji}$就将导致controlled-$S$门在$\mathcal F_n$中的出现,产生矛盾。

$\mathcal B_n$生成元集合中的$\text{CNOT}^{\downarrow}$记号表示控制比特比目标比特标号小的$\text{CNOT}$门,这使得\cref{equ1}式中的$\Delta$变成了下三角,且对角元全是$1$。这样的简化也使得我们可以直接地写出算符$F$的量子线路表示\footnote{需要注意的是这里的$\Gamma$并不严格对应\cref{equ1}式中的$\Gamma$,因为\cref{equ1}中相位被表示成了关于$x$的双线性函数,而在这里,对应给$\Gamma$的输入等效地变成了$\Delta x$。当然我们可以换一个记号(令$\Gamma' = \Delta \Gamma \Delta^{\text T}$仍是对称的)来实现这样的转化,原论文并没有显式地指出这一点。}:
\begin{equation}
F = O\prod_{i=1}^{n}S_i^{\Gamma_{i, i}}\prod_{1 \le i < j \le n}\text{CZ}_{i, j}^{\Gamma_{i, j}} \prod_{1 \le i < j \le n}\text{CNOT}_{i, j}^{\Delta_{j, i}}
\label{equ2}
\end{equation}

接下来,我们会用$F(O, \Gamma, \Delta)$来表示\cref{equ2}中的$F$。
\begin{definition}
	我们称算符 $F(O, \Gamma, \Delta)$的Pauli部分是平凡的,如果$O = I$。
\end{definition}

顺带一提的是$|\mathcal B_n| = 2^{n^2 + 2n}$,这是因为$O, \Gamma, \Delta$分别有$4^n, 2^{\frac{n^2+n}{2}}, 2^{\frac{n^2-n}{2}}$种取法,注意忽略了global phase。

我们引入本文最为核心的一个定理:
\begin{theorem}[Canonical Form]
	任意$U \in \mathcal C_n$都可以被\textbf{唯一地}写成
	\begin{equation}
		U = F(I, \Gamma, \Delta) \cdot \left( \prod_{i=1}^nH_i^{h_i} \right)\sigma \cdot F(O', \Gamma', \Delta')
		\label{canonical}
	\end{equation}
	其中$h \in \{0, 1\}^n, \sigma \in \mathcal S_n$是$n$阶排列,$F(I, \Gamma, \Delta), F(O', \Gamma', \Delta') \in \mathcal B_n$,其中$\Gamma, \Delta$需要满足条件:对于$\forall 1 \le i, j \le n$:
	\begin{enumerate}
		\renewcommand{\labelenumi}{\textbf{C\theenumi} }
		\item 若$h_i = 0, h_j = 0$,则$\Gamma_{i, j} = 0$;
		\item 若$h_i = 1, h_j = 0, \sigma(i) > \sigma(j)$,则$\Gamma_{i, j} = 0$;
		\item 若$h_i = 0, h_j = 0, \sigma(i) > \sigma(j)$,则$\Delta_{i, j} = 0$;
		\item 若$h_i = 1, h_j = 1, \sigma(i) < \sigma(j)$,则$\Delta_{i, j} = 0$;
		\item 若$h_i = 1, h_j = 0$,则$\Delta_{i, j} = 0$;
	\end{enumerate}
	\label{main}
\end{theorem}
\newpage
这个定理初看是十分迷惑的,因为这些不知所云的古怪限制通过我们难以理解的方式,确立了Canonical form的表示唯一性。通过Bruhat decomposition theorem \cite{bruhat} 以及下面两条引理,我们将得到对于Canonical form更深入的理解。

\begin{theorem}[Bruhat decomposition]
	Clifford group $\mathcal C_n$ 可以写成如下\textbf{不交}并的形式
	\begin{equation}
		\mathcal C_n = \bigsqcup_{h \in \{0, 1\}^n}\bigsqcup_{\sigma \in \mathcal S_n} \mathcal B_n\left(\prod_{i=1}^{n}H_i^{h_i}\right)\sigma\mathcal B_n
	\end{equation}
	\label{brh}
\end{theorem}
\begin{definition}
	对于$h \in \{0, 1\}^n, \sigma \in \mathcal S_n$,定义$\mathcal B_n$的子群
	\begin{equation}
	\mathcal B_n(h, \sigma) = \{F \in \mathcal B_n: W^{-1}FW \in \mathcal B_n\}
	\end{equation}
	(可以很容易地验证这是一个子群)其中$W = \left(\prod\limits_{i=1}^{n}H_i^{h_i}\right)\sigma$,以下会始终沿用这个记号。
\end{definition}
\begin{lemma}
	记$\overline{h} = h \oplus 1^n$表示$h$的按位取反,那么任意算符$F(O, \Gamma, \Delta) \in \mathcal B_n$是$\mathcal B_n(\overline{h}, \sigma)$中的元素,当且仅当$\Gamma, \Delta$对于$h, \sigma$满足\textbf{\textit{C1-C5}},此外还存在关系\begin{equation}
	\mathcal B_n(\overline{h}, \sigma) \cap \mathcal B_n(h, \sigma) = \mathcal P_n
	\label{cap}
	\end{equation}
	\label{lemma2-1}
\end{lemma}
\begin{lemma}
	任意算符$F \in \mathcal B_n$都可以被唯一写成$F = F_LF_R$,其中$F_R \in \mathcal B_n(h, \sigma)$,$F_L \in \mathcal B_n(\overline{h}, \sigma)$且Pauli部分是平凡的。
	\label{lemma2-2}
\end{lemma}

可以意识流地构想一下上述结论是如何导出Canonical form并证明唯一性的:首先,\cref{brh} 指出可以对$\mathcal C_n$中任意元素作形如$\mathcal B_n W \mathcal B_n$的分解,且$W$是唯一确定的;\cref{lemma2-1} 指出了一项重要结论——\textbf{\textit{C1-C5}}把Canonical form中的前一个$F$限制在了$\mathcal B_n$的一个子群$\mathcal B_n(\overline h, \sigma)$中,\cref{lemma2-2} 则说明了Canonical form的存在性,而唯一性证明则是由$F_L$的平凡Pauli部分以及\cref{cap}引出的。

接下来我们通过上述结论形式化地证明一下\cref{main}。由\cref{brh} 知任意$U \in \mathcal C_n$可以被写成$U = LWR$,其中$L, R \in \mathcal B_n$,且$W$是唯一确定的。根据\cref{lemma2-2},$L$可以被进一步分解为$L = BC$,其中$C \in \mathcal B_{n}(h, \sigma), B \in \mathcal B_n(\overline h, \sigma)$且Pauli部分平凡,因此有
\begin{equation}
U = LWR = BCWR = BWW^{-1}CWR = BWC'R
\end{equation}
根据$\mathcal B_n(h, \sigma)$的定义,有$C' \triangleq W^{-1}CW \in \mathcal B_n$,因而$C'R \in \mathcal B_n$,这证明了Canonical form的存在性。至于唯一性,设
\begin{equation}
F_1WF_1' = F_2WF_2'
\label{unique}
\end{equation}
满足$F_1', F_2' \in \mathcal B_n, F_1, F_2 \in \mathcal B_n(\overline h, \sigma)$且Pauli部分平凡,通过对\cref{unique}的一些变形
\begin{equation}
W^{-1}(F_2^{-1}F_1)W = F_2'(F_1')^{-1} \in \mathcal B_n
\end{equation}
根据$\mathcal B_n(h, \sigma)$的定义我们可以得到$F_2^{-1}F_1 \in B_n(h, \sigma)$,故$F_2^{-1}F_1 \in \mathcal B_n(\overline{h}, \sigma) \cap \mathcal B_n(h, \sigma) = \mathcal P_n$,但两者均只有平凡的Pauli部分,这说明了$F_2^{-1}F_1 = I$,从而$F_1 = F_2, F_1' = F_2'$,唯一性得证。\hfill$\blacksquare$

接下来我们将重点讨论两个重要引理的证明。当然和之前一样,只展示核心思路,具体细节会被放在附录中。

\subsection{\cref{lemma2-1} 的证明}
证明“当且仅当”的工作就是证明两个方向的蕴含关系。先考虑“$\Leftarrow$”即充分性,也即证明只要$\Gamma, \Delta$对于$h, \sigma$满足\textbf{\textit{C1-C5}},就可以使$F(O, \Gamma, \Delta) \in \mathcal B(\overline h, \sigma)$。由于$F$总可以被拆解成若干单Clifford门的乘积的形式,因此我们只需要验证$\mathcal B_n$生成元集合中的每一个元素——$X, S, \text{CZ}, \text{CNOT}^{\downarrow}$——满足条件即可。列举出所有情况即可。

然后考虑“$\Rightarrow$”即必要性。记$\mathcal B_n'(h, \sigma)$表示对于$\overline h, \sigma$满足\textbf{\textit{C1-C5}}的算符集合\footnote{这里还没有指出是$\mathcal B_n'(h, \sigma)$一个群。},充分性的证明工作说明了$\mathcal B_n'(h, \sigma) \subseteq \mathcal B_n(h, \sigma)$,我们现在希望能得到$\mathcal B_n'(h, \sigma) = \mathcal B_n(h, \sigma)$。采取的方法是对集合元素计数,我们有如下的集合大小关系
\begin{equation}
|\mathcal C_n| \le \sum_{h \in \{0, 1\}^n} \sum_{\sigma \in \mathcal S_n} \frac{|\mathcal B_n|^2}{|\mathcal B_n(h, \sigma)|} \le \sum_{h \in \{0, 1\}^n} \sum_{\sigma \in \mathcal S_n} \frac{|\mathcal B_n|^2}{|\mathcal B'_n(h, \sigma)|}
\end{equation}

想要证明$|\mathcal B'_n(h, \sigma)| = |\mathcal B_n(h, \sigma)|$,可以考虑直接证明右侧和式的值就等于$|\mathcal C_n|$。根据$\mathcal B_n'(h, \sigma)$的定义,我们知道\begin{equation}
|\mathcal B_n'(h, \sigma)| = |\mathcal B_n|2^{-I_n(h, \sigma)}
\label{B_n}
\end{equation}
其中$I_n(h, \sigma)$表示$\overline h, \sigma$通过\textbf{\textit{C1-C5}}给到$\Gamma, \Delta$的限制数目。可以验证\begin{equation}
I_n(h, \sigma) = \frac{n(n-1)}{2} + |h| + \sum_{1 \le i < j \le n: \sigma(i) < \sigma(j)}(-1)^{h_i + 1}
\label{I_n}
\end{equation}

通过\cref{I_n}我们将会证明$|\mathcal C_n| = \sum_{h \in \{0, 1\}^n} \sum_{\sigma \in \mathcal S_n} \frac{|\mathcal B_n|^2}{|\mathcal B'_n(h, \sigma)|}$,从而得到\cref{lemma2-1} 的必要性作为结论。此外还需要证明\cref{cap}:可以验证$\mathcal B_n'(h, \sigma) \cap \mathcal B_n'(\overline h, \sigma) = \mathcal P_n$——因为同时满足对$(h, \sigma)$与$(\overline h, \sigma)$满足\textbf{\textit{C1-C5}}会直接导致$\Gamma = \Delta = 0^{n\times n}$——那么结合前面的论述就可以得到\cref{cap}了。本部分的具体论证细节将在\cref{proof2-1} 中给出。

\subsection{\cref{lemma2-2} 的证明}

记$F_1, F_2, \cdots, F_m$为$\mathcal B_n(\overline h, \sigma)$中所有Pauli部分平凡的元素排成一列,显然有$m = |\mathcal B_n(\overline h, \sigma) / \mathcal P_n| = 4^{-n}|\mathcal B_n(\overline h, \sigma)|$。我们声称所有左陪集$F_j\mathcal B_n(h, \sigma)$是两两不交的,这指出了分解的唯一性。而要证明存在性,可以考虑对集合元素计数,具体地,需要验证\begin{equation}
|\mathcal B_n(h, \sigma)| \cdot |\mathcal B_n(\overline h, \sigma)| = |\mathcal P_n| \cdot |\mathcal B_n| = 4^n |\mathcal B_n|
\label{size}
\end{equation}
注意到$|\mathcal B_n| = 2^{n^2 + 2n}$以及$|\mathcal B_n(h, \sigma)| = |\mathcal B_n|2^{-I_n(h, \sigma)}$,而又可以验证$I_n(h, \sigma) + I_n(\overline h, \sigma) = n^2$,代入后能直接发现\cref{size}成立,从而完成证明。细节可见\cref{proof2-2}。

\section{第三部分}\label{partC}

现在我们可以通过对Canonical form作随机采样来实现对Clifford group的随机采样了。采样的过程分为两部分,注意到$\mathcal C_n$是$\mathcal B_nW\mathcal B_n$的不交并,我们可以先按照一定的概率分布采样$W$,也即采样$h \in \{0, 1\}^n, \sigma \in \mathcal S_n$。具体地,这个概率分布为\begin{equation}
P_n(h, \sigma) = \frac{|\mathcal B_nW\mathcal B_n|}{|\mathcal C_n|} = \frac{|\mathcal B_n|^2}{|\mathcal B_n(h, \sigma)|\cdot |\mathcal C_n|} = \frac{2^{I_n(h, \sigma)}}{\sum_{h, \sigma}2^{I_n(h, \sigma)}}
\label{distri}
\end{equation}
\cref{distri}的形式与一种被称为Mallows distribution \cite{mallows}的概率分布类似,这里\cite{paper1}的作者将其称为quantum Mallows distribution。可以证明这个采样过程可以被如下的算法实现

\begin{algorithm}\small
	\caption{\small 根据quantum Mallows distribution $P_n(h, \sigma)$来生成$h, \sigma$}
	\begin{algorithmic}[1]
		\State $A \gets [1...n]$
		\For{$i = 1 \ \text{to}\  n$}
			\State $m \gets |A|$
			\State Sample $h_i \in \{0, 1\}$ and $k \in [1...m]$ from the probability distribution
			$$p(h_i, k) = \frac{2^{m-1+h_i+(m-k)(-1)^{1 + h_i}}}{4^m-1}$$
			\State Let $j$ be the $k$-th largest element of $A$
			\State $\sigma(i) \gets j$
			\State $A \gets A \setminus \{j\}$
		\EndFor
		\State \Return $(h, \sigma)$
	\end{algorithmic}
\end{algorithm}

\begin{lemma}
	上述算法可以正确地按照$P_n(h, \sigma)$的分布来采样$h, \sigma$,同时运行时间为$\tilde{O}(n)$\footnote{渐进复杂度记号$\tilde O$(tilde big-O)表示忽略对数因子,即$\tilde O(n) = O(n \log^k n)$。}。
	\label{lemma3}
\end{lemma}

证明用的是归纳,与\cref{lemma2-1} 中证明$|\mathcal C_n| = \sum_{h, \sigma} \frac{|\mathcal B_n|^2}{|\mathcal B'_n(h, \sigma)|}$的方法类似,详细可参考\cref{proof3}。

在得到$W$之后,下一步是对左右两侧的$\mathcal B_n$进行采样。这个过程只需要按照Canonical form \cref{canonical}对$\Gamma, \Delta, O', \Gamma', \Delta'$进行采样即可,因此是简单的。代码展示在\cref{code3} 中。

值得注意的是,即使去掉Canonical form \cref{canonical}中对$\Gamma, \Delta$的限制(\textbf{\textit{C1-C5}}),即把算符$F$的限制从$\mathcal B_n / \mathcal B_n(h, \sigma) \cong \mathcal B_n(\overline h, \sigma) / \mathcal P_n$扩展到$\mathcal B_n / \mathcal P_n$,得到的采样结果也仍是均匀分布的。这有赖于群论中的Lagrange定理\cite{lagrange},它指出了子群的每一个陪集的大小都是相同的。\href[]{https://qiskit.org/documentation/stubs/qiskit.quantum_info.random_clifford.html?highlight=random_clifford#qiskit.quantum_info.random_clifford}{Qiskit 中所实现的 random\_clifford 方法}就采用了这种策略,这使得程序变得更加简短,实现效率上也有所提升,代价则是消耗了更多的随机比特。

\newpage
\bibliographystyle{unsrt}
\bibliography{ref}

\section*{附录}

\begin{appendices}

\section{\cref{lemma1-1} 的详细证明}

\begin{table}[h]
	\centering
	\begin{tabular}{c|c|c}
		$U$ & $g$ & $UgU^{\dagger}$\\
		\hline 
		\multirow{4}{*}{controlled-NOT} & $X_1$ & $X_1X_2$\\
		& $X_2$ & $X_2$\\
		& $Z_1$ & $Z_1$\\
		& $Z_2$ & $Z_1Z_2$\\
		\hline
		\multirow{2}{*}{Hadamard $H$} & $X$ & $Z$\\
		& $Z$ & $X$\\
		\hline
		\multirow{2}{*}{phase $S$} & $X$ & $Y$\\
		& $Z$ & $Z$\\
	\end{tabular}
	\caption{验证Clifford门都是$\mathcal P_n$的正规化子}
	\label{tab1}
\end{table}
\section{\cref{lemma1-2} 的详细证明}\label{proof1-2}
\subsection{第一步}

先假设正规化子$U$满足$UZU^{\dagger} = Z$,那么$UZ = ZU$由此说明$U, Z$对易,从而$U$只能有非零对角元$$U = e^{i\theta}\begin{pmatrix}
1 & 0\\
0 & e^{i\phi}
\end{pmatrix}$$

因为$U$是正规化子,我们也有$$UXU^{\dagger} = \begin{pmatrix}
0 & e^{i\phi}\\
e^{-i\phi} & 0
\end{pmatrix} \in \{\pm X, \pm iX, \pm Y, \pm iY\}$$

那么$e^{i\phi} \in \{\pm 1, \pm i\}$,这说明$U$可以由相位门$S$实现,至多差一个global phase。

现在考虑更一般的情况。对于任意的$U \in \mathcal N(\mathcal P_1)$,必然存在一个$g \in \mathcal P_1 \setminus \{\pm I, \pm iI\}$满足$UgU^{\dagger} = Z$。注意到$HXH^{\dagger} = Z, (HS)Y(HS)^{\dagger} = -Z$,因此总可以找到一个由Hadamard门与相位门组成的$V$,满足$VgV^{\dagger} \in \{\pm Z, \pm iZ\}$。把$U$写为$U = U_1V$,那么$Z = UgU^{\dagger} = U_1VgV^{\dagger}U_1^{\dagger} \in \{\pm U_1ZU_1^{\dagger}, \pm iU_1ZU_1^{\dagger}\}$,根据之前的论述,$U_1$可以由相位门实现,至多差一个global phase,因此$U = U_1V$可以由$O(1)$个Hadamard门和相位门实现,至多差一个global phase。

\subsection{第二步}

\begin{figure}[h]
	\centerline{
		\Qcircuit  @C = 1em @R = 0.8em{
			& \qw & \qw & \ctrl{1} & \gate{H} & \ctrl{1} & \qw\\
			& {/} \qw & \gate{U'} & \gate{g'} & \qw & \gate{g} & \qw
		}
	}
	\caption{证明中构造的量子线路}
	\label{cir}
\end{figure}

考虑\cref{cir} 中的量子线路,其中$U'$定义为满足$U' \ket{\psi} = \sqrt2\bra{0}U(\ket{0}\otimes\ket{\psi})$的算符。这里假设对于某个$g, g' \in \mathcal{P}_n$,有\begin{equation}
\begin{split}
UZ_1U^{\dagger} = X_1 \otimes g\\
UX_1U^{\dagger} = Z_1 \otimes g'
\end{split}
\label{b2}
\end{equation}
注意到两个等式的左侧都是Hermitian的,我们有$g = g^{\dagger}, g' = g'^{\dagger}$。


我们证明这个量子线路实现的算符——姑且称之为$\tilde{U}$——与$U$是等价的,且可以由$O(n^2)$个Clifford门组合实现。首先,我们可以写出\cref{b2}的一些变式\begin{align}
U
&= (X_1 \otimes g)UZ_1\\
&= (Z_1 \otimes g')UX_1\\
&= (Z_1X_1 \otimes g'g)U(Z_1X_1)
\end{align}

将其代入$U' \ket{\psi} = \sqrt2\bra{0}U(\ket{0}\otimes\ket{\psi})$后可以得到
\begin{align}
U' \ket{\psi} 
&= \sqrt2\bra{0}U(\ket{0}\otimes\ket{\psi})\\
&= \sqrt2\bra{1}gU(\ket{0}\otimes\ket{\psi})\\
&= \sqrt2\bra{0}g'U(\ket{1}\otimes\ket{\psi})\\
&= -\sqrt2\bra{1}g'gU(\ket{1}\otimes\ket{\psi})
\end{align}

为了说明$\tilde{U} = U$,我们其实只需要说明对于任意的$\ket{\alpha}, \ket{\beta} \in \{\ket0, \ket1\}$, 都满足$\bra{\alpha}\tilde{U}(\ket{\beta}\otimes\ket{\psi}) = \bra{\alpha}U(\ket{\beta}\otimes\ket{\psi})$,于是
\begin{equation}
\begin{split}
\bra{0} \tilde{U} (\ket{0} \otimes \ket{\psi}) &= \bra{0}(\frac{1}{\sqrt 2}\ket{0} \otimes U'\ket{\psi} + \frac{1}{\sqrt 2}\ket{1}\otimes gU'\ket{\psi})\\
&= \frac{1}{\sqrt 2}U'\ket{\psi}\\
&= \frac{1}{\sqrt 2} \cdot \sqrt2\bra{0}U(\ket{0}\otimes\ket{\psi})\\
&= \bra{0} U (\ket{0} \otimes \ket{\psi})
\end{split}
\end{equation}
\begin{equation}
\begin{split}
\bra{1} \tilde{U} (\ket{0} \otimes \ket{\psi}) &= \bra{1}(\frac{1}{\sqrt 2}\ket{0} \otimes U'\ket{\psi} + \frac{1}{\sqrt 2}\ket{1}\otimes gU'\ket{\psi})\\
&= \frac{1}{\sqrt 2}gU'\ket{\psi}\\
&= \frac{1}{\sqrt 2}g \cdot \sqrt2\bra{1}gU(\ket{0}\otimes\ket{\psi})\\
&= \bra{1} U (\ket{0} \otimes \ket{\psi})
\end{split}
\end{equation}
\begin{equation}
\begin{split}
\bra{0} \tilde{U} (\ket{1} \otimes \ket{\psi}) &= 
\bra{0}(\frac{1}{\sqrt 2}\ket{0}\otimes g'U'\ket{\psi} - \frac{1}{\sqrt 2}\ket{1}\otimes gg'U'\ket{\psi})\\
&= \frac{1}{\sqrt 2}g'U'\ket{\psi}\\
&= \frac{1}{\sqrt 2}g' \cdot \sqrt2\bra{0}g'U(\ket{1}\otimes\ket{\psi})\\
&= \bra{0} U (\ket{1} \otimes \ket{\psi})
\end{split}
\end{equation}
\begin{equation}
\begin{split}
\bra{1} \tilde{U} (\ket{1} \otimes \ket{\psi}) &= 
\bra{1}(\frac{1}{\sqrt 2}\ket{0}\otimes g'U'\ket{\psi} - \frac{1}{\sqrt 2}\ket{1}\otimes gg'U'\ket{\psi})\\
&= -\frac{1}{\sqrt 2}gg'U'\ket{\psi}\\
&= -\frac{1}{\sqrt 2}gg' \cdot -(\sqrt2\bra{1}g'gU(\ket{1}\otimes\ket{\psi}))\\
&= \bra{1} U (\ket{1} \otimes \ket{\psi})\\
\end{split}
\end{equation}

由此我们证明了$\tilde{U} = U$。注意到以上线路除了$U'$外,只包含一个controlled-$g'$门、一个Hadamard门和一个controlled-$g$门,而任意的controlled-$\{X, Y, Z\}$门都是可以由$O(1)$个Clifford门实现的,因此整个线路除了$U'$外,可以通过$O(n) + 1 + O(n) = O(n)$个Clifford门来实现。

结合归纳假设,我们证明了满足\cref{b2}的$U$可以被$O(n^2) + O(n) = O(n^2)$个Clifford门组合实现。

\subsection{第三步}

我们希望把第二步中的结果推广。考虑任意的$U \in \mathcal N(\mathcal P_{n+1})$,记$G = UZ_1U^{\dagger}, G' = UX_1U^{\dagger}$,那么显然有$G, G' \in \mathcal P_{n+1}$。注意到$G$与$G'$必须反对易——因为$Z$与$X$就是反对易的——那么必然存在某个$j \in [1, n+1]$使得$G$与$G'$在第$j$个量子比特上作用了$\{X, Y, Z\}$中不同的门,分别记为$\sigma$与$\sigma'$。额外添加第一个量子比特与第$j$个量子比特之间的交换门\footnote{交换门$\text{SWAP}$可以由被三个$\text{CNOT}$门实现,因此是Clifford算符,自然也是$\mathcal P_{n+1}$的正规化子。},可以得到
\begin{equation}
\begin{split}
(\text{SWAP}_{1, j}U)Z_1(\text{SWAP}_{1, j}U)^{\dagger} &= \text{SWAP}_{1, j} \cdot G \cdot \text{SWAP}_{1, j}^{\dagger} = \sigma \otimes g\\
(\text{SWAP}_{1, j}U)X_1(\text{SWAP}_{1, j}U)^{\dagger} &= \text{SWAP}_{1, j} \cdot G' \cdot \text{SWAP}_{1, j}^{\dagger} = \sigma' \otimes g'
\end{split}
\end{equation}
其中$g, g' \in \mathcal{P}_n$。值得一提的是,对于Pauli矩阵$\sigma, \sigma'$,总会存在由Hadamard门与相位门组成的$V$,满足$V\sigma V^{\dagger} = X, V\sigma' V^{\dagger} = Z$,至多差一个global phase。这个结论可以通过\cref{b2table} 来验证

\begin{table}
	\centering
	\begin{tabular}{c|c|c}
		$\sigma$ & $\sigma'$ & 选取的 $V$ \\
		\hline
		$X$ & $Y$ & $HSH$\\
		%		\hline
		$X$ & $Z$ & $I$\\
		%		\hline
		$Y$ & $X$ & $SH$\\
		%		\hline
		$Y$ & $Z$ & $S$\\
		%		\hline
		$Z$ & $X$ & $H$\\
		%		\hline
		$Z$ & $Y$ & $HS$\\
	\end{tabular}
	\caption{验证不同的Pauli矩阵$\sigma, \sigma'$总可以被$H, S$的乘积共轭变换到$X, Z$}
	\label{b2table}
\end{table}

因此进一步地,我们可以得到
\begin{equation}
\begin{split}
(V_1\text{SWAP}_{1, j}U)Z_1(V_1\text{SWAP}_{1, j}U)^{\dagger} &= V_1\text{SWAP}_{1, j} \cdot G \cdot \text{SWAP}_{1, j}^{\dagger}V_1^{\dagger} = V_1\sigma V_1^\dagger \otimes g = X \otimes g\\
(V_1\text{SWAP}_{1, j}U)X_1(V_1\text{SWAP}_{1, j}U)^{\dagger} &= V_1\text{SWAP}_{1, j} \cdot G' \cdot \text{SWAP}_{1, j}^{\dagger}V_1^{\dagger} = V_1\sigma' V_1^\dagger \otimes g' = Z \otimes g'
\end{split}
\end{equation}
其中$V$的下标表示$V$作用在第一个量子比特上。根据第二步中的结论,我们知道$V_1\text{SWAP}_{1, j}U$可以由$O(n^2)$个Clifford门实现,注意到$(V_1\text{SWAP}_{1, j})^{\dagger}$仅包含$O(1)$个Clifford门,这说明了任意的$U \in \mathcal N(\mathcal P_{n+1})$都可以由$O(n^2)$个Clifford门实现,至多差一个global phase。

\section{\cref{lemma2-1} 的详细证明}\label{proof2-1}
\subsection{充分性证明}
为了方便起见,我们用$\overline{h} = h \oplus 1^n$代替$h$来改写一下\cref{main} 中的\textbf{\textit{C1-C5}}:

\begin{enumerate}
	\renewcommand{\labelenumi}{\textbf{\textit{B\theenumi}} }
	\item 若$h_i = 1, h_j = 1$,则$\Gamma_{i, j} = 0$;
	\item 若$h_i = 0, h_j = 1, \sigma(i) > \sigma(j)$,则$\Gamma_{i, j} = 0$;
	\item 若$h_i = 1, h_j = 1, \sigma(i) > \sigma(j)$,则$\Delta_{i, j} = 0$;
	\item 若$h_i = 0, h_j = 0, \sigma(i) < \sigma(j)$,则$\Delta_{i, j} = 0$;
	\item 若$h_i = 0, h_j = 1$,则$\Delta_{i, j} = 0$。
\end{enumerate}

现在,我们证明对于满足上述条件的$\Gamma, \Delta$,都有$F(O, \Gamma, \Delta) \in \mathcal B(h, \sigma)$即$W^{-1}FW \in \mathcal B_n$。先验证对于$\mathcal B_n$的生成元集$\{X, S, \text{CNOT}^{\downarrow}, \text{CZ}\}$,上述叙述成立:
\begin{enumerate}
	\item $F = X_i$。可以验证此时$W^{-1}FW$一定是Pauli算符(注意到$HX_iH = Z_i$),因此是$\mathcal B_n$中的元素。
	\item $F = S_i$。此时$W^{-1}FW = \begin{cases}
	S_{\sigma(i)}, & h_i = 0\\
	H_{\sigma(i)}S_{\sigma(i)}H_{\sigma(i)}, & h_i = 1
	\end{cases}$,注意到$S_i$的存在表明$\Gamma_{i, i}=1$,从而根据\textbf{\textit{B1}}有$h_i=0$,此时$W^{-1}FW = S_{\sigma(i)} \in \mathcal B(h, \sigma)$。 
	\item $F = \text{CNOT}_{i, j}$,其中$i < j$,此时有$\Delta_{j, i} = 1$。根据$h_i, h_j$的不同取值可以得到如下表格
	\begin{center}
	\begin{tabular}{c|c|c}
		$h_i$ & $h_j$ & $W^{-1}\text{CNOT}_{i, j}W$\\
		\hline
		$0$ & $0$ & $\text{CNOT}_{\sigma(i), \sigma(j)}$\\
		$0$ & $1$ & $\text{CZ}_{\sigma(i), \sigma(j)}$\\
		$1$ & $0$ & $(H \otimes I \cdot \text{CNOT} \cdot H \otimes I)_{\sigma(i), \sigma(j)}$\\
		$1$ & $1$ & $\text{CNOT}_{\sigma(j), \sigma(i)}$\\
	\end{tabular}
	\end{center}
	观察到\textbf{\textit{B1-B5}}在$\Delta_{j,i} = 1$时否定了如下几种情况:$h_i = h_j = 1$且$\sigma(j) > \sigma(i)$;$h_i = h_j = 0$且$\sigma(i) > \sigma(j)$;$h_i = 1$且$h_j = 0$。可以发现除了这些情况外,$W^{-1}FW \in \mathcal B(h, \sigma)$总是成立的。
	\item $F = \text{CZ}_{i,j}$,此时有$\Gamma_{i,j} = \Gamma_{j,i} = 1$根据$h_i, h_j$的不同取值可以得到如下表格
	\begin{center}
	\begin{tabular}{c|c|c}
		$h_i$ & $h_j$ & $W^{-1}\text{CNOT}_{i, j}W$\\
		\hline
		$0$ & $0$ & $\text{CZ}_{\sigma(i), \sigma(j)}$\\
		$0$ & $1$ & $\text{CNOT}_{\sigma(i), \sigma(j)}$\\
		$1$ & $0$ & $\text{CNOT}_{\sigma(j), \sigma(i)}$\\
		$1$ & $1$ & $(H \otimes I \cdot \text{CNOT} \cdot H \otimes I)_{\sigma(i), \sigma(j)}$\\
	\end{tabular}
	\end{center}
	观察到\textbf{\textit{B1-B5}}在$\Gamma_{i,j} = \Gamma_{j,i} = 1$时否定了如下几种情况:$h_i = h_j = 1$;$h_i = 0, h_j = 1$且$\sigma(i) > \sigma(j)$;$h_i = 1, h_j = 0$且$\sigma(j) > \sigma(i)$。可以发现除了这些情况外,$W^{-1}FW \in \mathcal B(h, \sigma)$总是成立的。
\end{enumerate}

对于任意满足上述条件的$F(O, \Gamma, \Delta)$,由于其一定可以分解成$\{X, S, \text{CNOT}^{\downarrow}, \text{CZ}\}$乘积的形式,从而也可以验证$W^{-1}FW \in \mathcal B_n$,这证明了引理的充分性。

\subsection{必要性证明}

我们先验证\cref{I_n}的结论:$I_n(h, \sigma)$是$\overline h, \sigma$通过\textbf{\textit{C1-C5}}给$\Gamma, \Delta$的限制数目,也就等于$h, \sigma$通过\textbf{\textit{B1-B5}}给$\Gamma, \Delta$的限制数目。为了方便我们沿用后者,并且可以显式地把限制数目写下来
\begin{center}
\begin{tabular}{c|c|c}
	& 给$\Gamma, \Delta$的限制数目 & 记号\\
	\hline
	\textbf{\textit{B1}} & $|h| + \sum_{i > j}h_ih_j$ & $|h|\ + \circled{1}$\\
	\textbf{\textit{B2}} & $\sum_{i > j, \sigma(i) > \sigma(j)}\overline{h}_ih_j + \sum_{i > j, \sigma(i) < \sigma(j)}h_i\overline h_j$ & $\circled{2} + \circled{3}$\\
	\textbf{\textit{B3}} & $\sum_{i > j, \sigma(i) > \sigma(j)}h_ih_j$ & $\circled{4}$\\
	\textbf{\textit{B4}} & $\sum_{i > j, \sigma(i) < \sigma(j)}\overline h_i\overline h_j$ & $\circled{5}$\\
	\textbf{\textit{B5}} & $\sum_{i > j}\overline h_ih_j$ & $\circled{6}$\\
\end{tabular}
\end{center}

简单验证一下上面的六项加起来就能够得到\cref{I_n}:
\begin{equation}
\begin{split}
I_n(h, \sigma) &= |h| + \circled 1 + \circled 2 + \circled 3 + \circled 4 + \circled 5 + \circled 6\\
&= |h| + (\circled 1 + \circled 6) + (\circled 2 + \circled 4) + (\circled 3 + \circled 5)\\
&= |h| + \sum_{i > j}h_j + \sum_{i > j, \sigma(i) > \sigma(j)}h_j + \sum_{i > j, \sigma(i) < \sigma(j)}\overline h_j\\
&= |h| + \sum_{i > j}h_j + \sum_{i > j}\overline h_j + \sum_{i > j, \sigma(i) > \sigma(j)}h_j - \overline{h}_j\\
&= \frac{n(n-1)}{2} + |h| + \sum_{1 \le i < j \le n: \sigma(i) < \sigma(j)}(-1)^{1 + h_i}
\end{split}
\end{equation}

根据\cite{paper2},我们知道$|\mathcal C_n| = 2^{n^2 + 2n}\prod\limits_{i=1}^{n}(4^i-1)$,回忆$|\mathcal B_n| = 2^{n^2 + 2n}$,为了完成证明,我们只需要验证如下等式成立
\begin{equation}
F(n) \triangleq \sum_{h \in \{0, 1\}^n} \sum_{\sigma \in \mathcal S_n} 2^{I_n(h, \sigma)} = \prod_{i=1}^{n}(4^i-1)
\label{2-1induction}
\end{equation}

这里的证明手段是归纳。$n = 1$时,$\sigma$一定是单位置换$e$,$I_1(0, e) = 0, I_1(1, e) = 1$而$F(1) = 3 = 2^{I_1(0, e)} + 2^{I_1(1, e)}$,故成立。

假设结论\cref{2-1induction}对于$n$成立。对于任意的$h \in \{0, 1\}^{n+1}$以及$\sigma \in \mathcal S_{n+1}$,将$h$写成$h = (h_1, h')$的形式,其中$h' \in \{0, 1\}^n$;取$\sigma' \in \mathcal S_n$满足$\sigma'(j) = \begin{cases}
\sigma(j+1), & \sigma(j+1) < \sigma(1)\\
\sigma(j+1) - 1, & \sigma(j+1) > \sigma(1)
\end{cases}$(相当于移除$\sigma$的第一位后得到新的$n$阶排列),可以通过验证得到
\begin{equation}
I_{n+1}(h, \sigma) = I_n(h', \sigma') + n + h_1 + (n + 1 - \sigma(1))(-1)^{1 + h_1}
\end{equation}
注意到映射$(h, \sigma) \to (h_1, h', \sigma(1), \sigma')$是一一对应的,因此
\begin{equation}
\begin{split}
F(n+1) &= F(n)\sum_{h_1 \in \{0, 1\}} \sum_{\sigma(1) = 1}^{n+1} 2^{n + h_1 + (n+1-\sigma(1))(-1)^{1 + h_1}}\\
&= F(n)\cdot 2^n\sum_{m=1}^{n+1}(2^{-n-1+m} + 2^{n-m+2})\\
&= F(n)\cdot 2^n\sum_{m=-n}^{n+1}2^m\\
&= F(n)\cdot 2^n(2^{n+2} - 2^{-n})\\
&= (4^{n+1} - 1)F(n)
\end{split}
\end{equation}
从而证明了\cref{2-1induction} 对于任何$n$都成立,同时也完成了引理的必要性证明。
\section{\cref{lemma2-2} 的详细证明}\label{proof2-2}

只需要补充说明所有左陪集$F_j\mathcal B_n(h, \sigma)$是两两不交的即可。假设对于$F_i \neq F_j$有$F_i\mathcal B_n(h, \sigma) \cap F_j\mathcal B_n(h, \sigma) \neq \varnothing$,这就说明$F_j^{-1}F_i \in \mathcal B_n(h, \sigma)$,然而$F_i, F_j \in \mathcal B(\overline h, \sigma)$故$F_j^{-1}F_i \in \mathcal B(\overline h, \sigma)$,根据\cref{lemma2-1} 可知$F_j^{-1}F_i \in \mathcal P_n$。注意到定义指出$F_i, F_j$的Pauli部分都是平凡的,这导致了$F_j^{-1}F_i = I$即$F_i = F_j$,产生矛盾。故左陪集两两不交。

\section{\cref{lemma3} 的详细证明}\label{proof3}

首先考察该算法的正确性,即验证其确实按照$P_n(h, \sigma)$的概率分布来生成了$h$和$\sigma$。考虑对$n$归纳:$n=1$时,有$P_1(0, e) = p(0, 1) = \frac13, P_1(1, e) = p(1, 1) = \frac23$,因此该算法按照$P_1(h, \sigma)$的概率分布生成了$h$和$\sigma$;$n > 1$时,考虑在\cref{proof2-1} 中使用过的分解,即对于$h, \sigma$构造$h', \sigma'$,由于$I_n(h, \sigma) = I_{n-1}(h', \sigma') + (n-1) + h_1 + (n-\sigma(1))(-1)^{1 + h_i}$,故一轮循环实质上按照正确的概率生成了$h_1$和$\sigma(1)$,注意到$(h, \sigma) \to (h_1, h', \sigma(1), \sigma')$的一一对应关系,只需要在接下来的循环中生成$h', \sigma'$即可。结合归纳假设,算法的正确性得证。

关于算法的复杂度,我们证明单轮循环(即单次对$h_1, \sigma(1)$的采样)的复杂度为$\tilde O(1)$。考虑按照一定顺序写下$\{0, 1\} \times [1, m]$中的所有有序对
\begin{equation}
P \triangleq [ p(1, 1), p(1, 2), \cdots, p(1, m), p(0, m), p(0, m-1), \cdots, p(0, 1) ]
\end{equation}
可以验证$P$作为一个数组(也可以视作概率分布)满足
\begin{equation}
P_a = \frac{2^{2m - 1 - a}}{2^{2m} - 1}, a \in [0, 2m - 1]
\end{equation}
因此想要实现对概率分布$P$的随机采样,只需要令
\begin{equation}
a = 2m - \lceil \log_2(r(2^{2m}-1) + 1) \rceil
\end{equation}
其中$r \in [0, 1]$是一个均匀随机分布的实数。因此,一轮循环需要消耗$O(\log n)$个随机比特,同时运行时间为$\tilde O(1)$。

\section{代码实现}\label{code3}

代码实现主要参考自\href[]{https://qiskit.org/documentation/_modules/qiskit/quantum_info/operators/symplectic/random.html#random_clifford}{Qiskit},修改了对\cref{canonical}中算符$F$的采样方式,即严格按照\textbf{\textit{C1-C5}}的限制对$\Gamma, \Delta$进行采样。

\begin{minted}[linenos=true]{python}
import numpy as np
from numpy.random import default_rng
from qiskit.quantum_info import Clifford, StabilizerTable, random_clifford

def my_random_clifford(num_qubits, seed=None):
    """
    Return a random Clifford operator, minimizing the number of random bits used.
    Args:
        num_qubits (int): the number of qubits for the Clifford
        seed (int or np.random.Generator): Optional. Set a fixed seed or generator for RNG.
    Returns:
        Clifford: a random Clifford operator.
    """
    if seed is None:
        rng = np.random.default_rng()
    elif isinstance(seed, np.random.Generator):
        rng = seed
    else:
        rng = default_rng(seed)

    had, perm = _sample_qmallows(num_qubits, rng)
    gamma1, delta1, gamma2, delta2 = _generate_tril(had, perm, rng)

    # For large num_qubits numpy.inv function called below can
    # return invalid output leading to a non-symplectic Clifford
    # being generated. This can be prevented by manually forcing
    # block inversion of the matrix.
    block_inverse_threshold = 50

    # Compute stabilizer table
    zero = np.zeros((num_qubits, num_qubits), dtype=np.int8)
    prod1 = np.matmul(gamma1, delta1) % 2
    prod2 = np.matmul(gamma2, delta2) % 2
    inv1 = _inverse_tril(delta1, block_inverse_threshold).transpose()
    inv2 = _inverse_tril(delta2, block_inverse_threshold).transpose()
    table1 = np.block([[delta1, zero], [prod1, inv1]])
    table2 = np.block([[delta2, zero], [prod2, inv2]])

    # Apply qubit permutation
    table = table2[np.concatenate([perm, num_qubits + perm])]

    # Apply layer of Hadamards
    inds = had * np.arange(1, num_qubits + 1)
    inds = inds[inds > 0] - 1
    lhs_inds = np.concatenate([inds, inds + num_qubits])
    rhs_inds = np.concatenate([inds + num_qubits, inds])
    table[lhs_inds, :] = table[rhs_inds, :]

    # Apply table
    table = np.mod(np.matmul(table1, table), 2).astype(bool)

    # Generate random phases
    phase = rng.integers(2, size=2 * num_qubits).astype(bool)
    return Clifford(StabilizerTable(table, phase))

def _generate_tril(had, perm, rng):
    """
    Return four 01-matrices gamma1, delta1, gamma2, delta2, minimizing the number of random bits used.
    Args:
        had: the Hadamard layer
        perm: the permutation layer
        both of them put some restrictions on gamma1 and delta1
        rng: random number generator
    Returns:
        four 01-matrices gamma1, delta1, gamma2, delta2 for Clifford sampling.
    """
    
    num_qubits = had.size
    
    gamma1 = np.diag(rng.integers(2, size=num_qubits, dtype=np.int8))
    gamma2 = np.diag(rng.integers(2, size=num_qubits, dtype=np.int8))
    delta1 = np.eye(num_qubits, dtype=np.int8)
    delta2 = delta1.copy()
    
    for i in range(num_qubits):
        for j in range(i):
            gamma2[i, j] = gamma2[j, i] = rng.integers(2, dtype=np.int8)
            delta2[i, j] = rng.integers(2, dtype=np.int8)
            
            # restrictions for gamma1
            if had[i] == 1 and had[j] == 1:
                gamma1[i, j] = gamma1[j, i] = rng.integers(2, dtype=np.int8)
            if had[i] == 1 and had[j] == 0 and perm[i] < perm[j]:
                gamma1[i, j] = gamma1[j, i] = rng.integers(2, dtype=np.int8)
            if had[i] == 0 and had[j] == 1 and perm[i] > perm[j]:
                gamma1[i, j] = gamma1[j, i] = rng.integers(2, dtype=np.int8)
            
            # restrictions for delta1
            if had[i] == 0 and had[j] == 1:
                delta1[i, j] = rng.integers(2, dtype=np.int8)
            if had[i] == 1 and had[j] == 1 and perm[i] > perm[j]:
                delta1[i, j] = rng.integers(2, dtype=np.int8)
            if had[i] == 0 and had[j] == 0 and perm[i] < perm[j]:
                delta1[i, j] = rng.integers(2, dtype=np.int8)
    
    return gamma1, delta1, gamma2, delta2

def _sample_qmallows(n, rng=None):
    """Sample from the quantum Mallows distribution."""

    if rng is None:
        rng = np.random.default_rng()

    # Hadmard layer
    had = np.zeros(n, dtype=bool)

    # Permutation layer
    perm = np.zeros(n, dtype=int)

    inds = list(range(n))
    for i in range(n):
        m = n - i
        eps = 4 ** (-m)
        r = rng.uniform(0, 1)
        index = -int(np.ceil(np.log2(r + (1 - r) * eps)))
        had[i] = index < m
        if index < m:
            k = index
        else:
            k = 2 * m - index - 1
        perm[i] = inds[k]
        del inds[k]
    return had, perm

def _inverse_tril(mat, block_inverse_threshold):
    """Invert a lower-triangular matrix with unit diagonal."""
    # Optimized inversion function for low dimensions
    dim = mat.shape[0]

    if dim <= 2:
        return mat

    if dim <= 5:
        inv = mat.copy()
        inv[2, 0] = mat[2, 0] ^ (mat[1, 0] & mat[2, 1])
        if dim > 3:
            inv[3, 1] = mat[3, 1] ^ (mat[2, 1] & mat[3, 2])
            inv[3, 0] = mat[3, 0] ^ (mat[3, 2] & mat[2, 0]) ^ (mat[1, 0] & inv[3, 1])
        if dim > 4:
            inv[4, 2] = (mat[4, 2] ^ (mat[3, 2] & mat[4, 3])) & 1
            inv[4, 1] = mat[4, 1] ^ (mat[4, 3] & mat[3, 1]) ^ (mat[2, 1] & inv[4, 2])
            inv[4, 0] = (
                mat[4, 0]
                ^ (mat[1, 0] & inv[4, 1])
                ^ (mat[2, 0] & inv[4, 2])
                ^ (mat[3, 0] & mat[4, 3])
            )
        return inv % 2

    # For higher dimensions we use Numpy's inverse function
    # however this function tends to fail and result in a non-symplectic
    # final matrix if n is too large.
    if dim <= block_inverse_threshold:
        return np.linalg.inv(mat).astype(np.int8) % 2

    # For very large matrices  we divide the matrix into 4 blocks of
    # roughly equal size and use the analytic formula for the inverse
    # of a block lower-triangular matrix:
    # inv([[A, 0],[C, D]]) = [[inv(A), 0], [inv(D).C.inv(A), inv(D)]]
    # call the inverse function recursively to compute inv(A) and invD

    dim1 = dim // 2
    mat_a = _inverse_tril(mat[0:dim1, 0:dim1], block_inverse_threshold)
    mat_d = _inverse_tril(mat[dim1:dim, dim1:dim], block_inverse_threshold)
    mat_c = np.matmul(np.matmul(mat_d, mat[dim1:dim, 0:dim1]), mat_a)
    inv = np.block([[mat_a, np.zeros((dim1, dim - dim1), dtype=int)], [mat_c, mat_d]])
    return inv % 2

if __name == "__main__":
    circuit = random_clifford(4)
    print(circuit)
    circuit2 = my_random_clifford(4)
    print(circuit2)
\end{minted}

\end{appendices}


\end{document}

