\documentclass[11pt]{article}
\usepackage[UTF8]{ctex}
\usepackage[a4paper]{geometry}
\geometry{left=2.0cm,right=2.0cm,top=2.5cm,bottom=2.5cm}

\usepackage{comment}
\usepackage{booktabs}
\usepackage{graphicx}
\usepackage{subfigure}
\usepackage{diagbox}
\usepackage{amsmath,amsfonts,graphicx,amssymb,bm,amsthm}
\usepackage{algorithm,algorithmicx}
\usepackage[noend]{algpseudocode}
\usepackage{fancyhdr}
\usepackage{tikz}
\usepackage{graphicx}
\usepackage{verbatim}
\usetikzlibrary{arrows,automata}
\usepackage{hyperref}
\usepackage{minted}
\setlength{\headheight}{14pt}
\setlength{\parindent}{0 in}

\newtheorem{theorem}{Theorem}
\newtheorem{lemma}[theorem]{Lemma}
\newtheorem{proposition}[theorem]{Proposition}
\newtheorem{claim}[theorem]{Claim}
\newtheorem{corollary}[theorem]{Corollary}
\newtheorem{definition}[theorem]{Definition}


\newcommand\E{\mathbb{E}}
\newcommand{\hwid}{3}			% 第几次作业
\newcommand{\name}{周书予} 		% 你的名字
\newcommand{\id}{2000013060} 	% 你的学号


\usetikzlibrary{positioning}

\begin{document}

    \pagestyle{fancy}
    \lhead{Peking University}
    \chead{}
    \rhead{Mathematical Foundations for the Information Age, 2021 Fall}

    \begin{center}
        {\LARGE \bf Homework \#\hwid}\\
        {\Large Due: 2020.11.11 23:59}\\
        {\Large \name}\\
        {\Large \id}\\
    \end{center}

  
    
  \section{(25 pts) Orthogonal Approximation}
  Suppose that the singular value decomposition of matrix $A$ is $A = U\Sigma V^\top$, find orthogonal matrix $W$ that minimizes $\|A − W\|_F$.
  \subsection*{answer}
  
  先证明两个引理。
  
  \textbf{引理1}\quad 对于任意$n$阶正交矩阵$B$和$n$阶矩阵$A$,总有$\|A\|_F = \|AB\|_F = \|BA\|_F$。
  
  \textbf{证明}\quad 记$A = \{a_{ij}\}$,$\mathbf b_i$为$B$的第$i$个行向量,注意到$|\mathbf b_i| = 1$且$\{\mathbf b_i\}$两两正交,于是有
  \begin{align*}
  	\|A\|_F^2 &= \sum_{i,j}a_{ij}^2\\
  	\|AB\|_F^2 &= \sum_{i=1}^{n}|\sum_{j=1}^{n}a_{ij}\mathbf{b}_j|^2\\
  	&= \sum_{i=1}^{n}(\sum_{j=1}^{n}a_{ij}^2)\\
   	&= \|A\|_F^2
  \end{align*}
  
  而又有$\|A\|_F^2 = \|A^{\textbf T}\|_F^2$,故$\|BA\|_F^2 = \|A^{\mathbf T}B^{\mathbf T}\|_F^2 = \|A^{\mathbf T}\|_F^2 =\|A\|_F^2$。\hfill $\blacksquare$

	\textbf{引理2}\quad 对于任意对角元均$\ge 0$的对角矩阵$\Sigma$,有$\arg\min\limits_{W^{\mathbf T}W = I}\|\Sigma - W\|_F^2 = I$。
	
	\textbf{证明}\quad
	$$\|\Sigma - W\|_F^2 = \sum_{i,j}(\delta_{ij}\sigma_i - w_{ij})^2 = \sum_{i=1}^{n}\sigma_i^2 + \sum_{i,j}w_{i,j}^2 - 2\sum_{i=1}^n\sigma_iw_{i,i} = \sum_{i=1}^{n}\sigma_i^2 + n - 2\sum_{i=1}^n\sigma_iw_{i,i}$$
	
	由于$\sum\limits_{i=1}^{n}\sigma_i^2+n$是定值,最小化$\|\Sigma - W\|_F^2$等价于最大化$\sum\limits_{i=1}^n\sigma_iw_{i,i}$,注意到$\sigma_i \ge 0$,此时取$w_{i,i} = 1$即可满足最大化条件,于是有$\arg\min\limits_{W^{\mathbf T}W = I}\|\Sigma - W\|_F^2 = I$。\hfill $\blacksquare$

	回到原问题。设答案正交矩阵为$W_0$,可以得到
	\begin{align*}
	W_0 &= \arg\min_{W^{\mathbf T}W = I}\|A - W\|_F^2\\
	&= \arg\min_{W^{\mathbf T}W = I}\|U\Sigma V^{\mathbf T} - W\|_F^2\\
	&= \arg\min_{W^{\mathbf T}W = I}\|\Sigma - U^{\mathbf T}WV\|_F^2\\
	&= U\arg\min_{W^{\mathbf T}W = I}\|\Sigma - W\|_F^2V^{\mathbf T}\\
	&= UV^{\mathbf T}
\end{align*}

其中第四个等号用到了“$U^{\mathbf T}WV$可以为任意正交矩阵”。
  \section{(20 pts) Exercise 3.28}
  Read in a photo and convert to a matrix. Perform a singular value decomposition of the matrix. Reconstruct the photo using only 1,2,4, and 16 singular values.
  \begin{enumerate}
      \item Print the reconstructed photo. How good is the quality of the reconstructed photo?
      \item What percent of the Frobenius norm is captured in each case?
  \end{enumerate}
  \subsection*{answer}

\begin{figure}[htbp]
	\centering
	\subfigure[原图]{
		\includegraphics[scale=0.16]{arima.png} \label{1}
	}
	\quad
	\subfigure[灰度图]{
		\includegraphics[scale=0.16]{arima_gray.png} \label{2} 
	}
	\quad
		\subfigure[前1个奇异值]{
		\includegraphics[scale=0.16]{arima_gray_1.png} \label{3} 
	}
	\\
	\subfigure[前2个奇异值]{
	\includegraphics[scale=0.16]{arima_gray_2.png} \label{4}
}
\quad
\subfigure[前4个奇异值]{
	\includegraphics[scale=0.16]{arima_gray_4.png} \label{5}
}
\quad
\subfigure[前16个奇异值]{
	\includegraphics[scale=0.16]{arima_gray_16.png} \label{6} 
}
\end{figure}

图片质量:前两张图上大致可以看出有马的位置,$4$个奇异值的时候大概能看出人形,最后一张图已经可以提供可分辨的特征了。

\begin{minted}{python}
import cv2
import numpy as np
image = cv2.imread("arima.png")
image_gray = cv2.cvtColor(image, cv2.COLOR_BGR2GRAY)
U, sigma, V = np.linalg.svd(image_gray)
for i in [1, 2, 4, 16]:
    print(np.sqrt(np.sum(sigma[:i] ** 2) / np.sum(sigma ** 2)))
\end{minted}
得到的结果为

\begin{center}
	\begin{tabular}{|c|c|c|c|c|}
		\hline 
		前$k$个奇异值 & $1$ & $2$ & $4$ & $16$ \\ 
		\hline 
		Frobenius norm 所占百分比 & $0.9630$ & $0.9752$ & $0.9865$ & $0.9961$\\
		\hline
	\end{tabular}
\end{center}

  \section{(30 pts) Power Method}
  Use {\bf{power method}} presented in section 3.7.1 of textbook to compute the largest 10 singular values of the random matrix $A$ generated using NumPy as follows:
  
np.random.seed(20211028) //set random seed \\
A = np.random.randn(2000, 1000)\\
  Write down the singular values and attach your code here. You could use any functions provided by the NumPy package.
  If you use other programming languages or other packages to solve this problem, please generate the random matrix in a similar fashion.
  \subsection*{answer}
  \begin{minted}{python}
import numpy as np

def Schimidt(x):
    for i in range(x.shape[1]):
        for j in range(i):
            x[:, i] -= x[:, j] * np.sum(x[:, i] * x[:, j])
        x[:, i] /= np.sqrt(np.sum(x[:, i] ** 2))
    return x

np.random.seed(20211028)
A = np.random.randn(2000, 1000)
B = A.T @ A
x = np.random.randn(1000, 10)
x = Schimidt(x)
for i in range(10000):
    x = Schimidt(B @ x)
x = B @ x
values = []
for i in range(x.shape[1]):
    values.append(np.power(np.sum(x[:, j] ** 2), 0.25))
values.sort(reverse = True)
print(values)
  \end{minted}
得到的结果为
\begin{align*}
\sigma_1 = 76.11359917216343\\
\sigma_2 = 75.63684144555306\\
\sigma_3 = 75.39052805452388\\
\sigma_4 = 75.22870902493426\\
\sigma_5 = 75.20521905044106\\
\sigma_6 = 74.80676731402508\\
\sigma_7 = 74.70959618081706\\
\sigma_8 = 74.50702749968583\\
\sigma_9 = 74.29437233364072\\
\sigma_{10} = 74.06615406206454
\end{align*}
  \section{(25 pts) Exercise 5.4 (Rewrited)} 
 
 Define a labeling $f:\{0,1\}^d\to \{-1,+1\}$:
 $$f(x_1,x_2,x_3,\cdots,x_d):=(-1)^{\sum_{i=1}^d x_i}$$
 
 Show that $\{0,1\}^d$ is not linearly separable if it is labeled by $f$.

\subsection*{answer}

假设$f$是可线性分类的,即存在$\mathbf w = (w_1, w_2, \cdots, w_d) \in \mathbb R^d$以及$t \in \mathbb R$,使得对于任意$\mathbf x = (x_1, x_2, \cdots, x_d) \in \{0, 1\}^d$,都有$$(\mathbf w \cdot \mathbf x - t) f(\mathbf x) > 0$$

分别代入$\mathbf x = (0, 0, 0, \cdots, 0) / (0, 1, 0, \cdots, 0) / (1, 0, 0, \cdots, 0) / (1, 1, 0, \cdots, 0)$可以得到
\begin{align}
-t &> 0\\
w_1 - t &< 0\\
w_2 - t &< 0\\
w_1 + w_2 - t &> 0
\end{align}

由$(1)(3)$可知$w_2 < 0$,这与$(2)(4)$矛盾。故这样的$\mathbf w, t$不存在,即$f$不可被线性分类。
\end{document}
