\documentclass[8pt]{article}
\usepackage[UTF8]{ctex}
\usepackage[a4paper]{geometry}

\usepackage{amsthm,amsmath,amssymb}
\usepackage{graphicx}
\usepackage{subfigure}
\usepackage{amsmath}
\usepackage{tabularx}
\usepackage{color}
\usepackage{hyperref}
\usepackage{ulem}
\usepackage{multirow}
\usepackage[cache=false]{minted}
\hypersetup{
	colorlinks=true,
	linkcolor=blue
}

\usepackage{appendix}
\geometry{a4paper,centering,scale=0.8}
\geometry{left=2.0cm, right=2.0cm, top=2.5cm, bottom=2.5cm}
\usepackage[format=hang,font=small,textfont=it]{caption}
\usepackage[nottoc]{tocbibind}

\usepackage{algorithm}
\usepackage{algorithmicx}
\usepackage{algpseudocode}
\usepackage{amssymb}
\usepackage{extarrows}
\usepackage{qcircuit}
\usepackage{fancyhdr}
\usepackage{cleveref}


\usepackage{tikz}  
\usetikzlibrary{arrows.meta}%画箭头用的包

\makeatletter
\def\@maketitle{%
	\newpage
	\begin{center}%
		\let \footnote \thanks
		{\LARGE \@title \par}%
		\vskip 1.5em%
		{\large
			\lineskip .5em%
			\begin{tabular}[t]{c}%
				\@author
			\end{tabular}\par}%
		\vskip 1em%
		{\large \@date}%
	\end{center}%
	\par
	\vskip 1.5em}
\makeatother

\newtheoremstyle{compact}%
{3pt}{3pt}%
{}{}%
{\bfseries}{\textcolor{red}{.}}%  % Note that final punctuation is omitted.
{.5em}{\mbox{\textcolor{red}{\thmname{#1}\thmnumber{ #2}}\thmnote{ (\textcolor{blue}{#3})}}}
\theoremstyle{compact}
\newtheorem{innercustomgeneric}{\customgenericname}
\providecommand{\customgenericname}{}
\newcommand{\newcustomtheorem}[2]{%
	\newenvironment{#1}[1]
	{%
		\renewcommand\customgenericname{#2}%
		\renewcommand\theinnercustomgeneric{##1}%
		\innercustomgeneric
	}
	{\endinnercustomgeneric}
}

\DeclareMathOperator{\card}{card}

\newtheorem{theorem}{定理}
\newtheorem{lemma}{引理}
\newtheorem{definition}{定义}
\newtheorem{proposition}{命题}
\newtheorem{corollary}{推论}
\newtheorem{remark}{注}
\newtheorem{Proof}{证明}

\def\obj#1{\textbf{\uline{#1}}}
\def\num#1{\textnormal{\textbf{\mbox{\textcolor{blue}{(#1)}}}}}
\def\le{\leqslant}
\def\ge{\geqslant}
\def\im{\text{im }}
\def\Pr#1{\text{Pr}\left[{#1}\right]}
\def\E#1{\mathbb{E}\left[{#1}\right]}
\def\Var#1{\text{Var}\left[{#1}\right]}


\title{\heiti\zihao{1} 代数结构与组合数学\ 第一次作业}
\author{\kaishu\zihao{-3} 周书予\\2000013060@stu.pku.edu.cn}

\CTEXoptions[today=old]
\date{\today}

\begin{document}
\pagestyle{fancy}
\lhead{代数结构与组合数学}
\chead{2022 Spring}
\rhead{第一次作业}


\crefname{theorem}{定理}{定理}
\crefname{lemma}{引理}{引理}
\crefname{figure}{图}{图}
\crefname{table}{表}{表}	
\maketitle

\iffalse
课本p. 238: 9, 11, 14, 16
(以下内容周四会讲到)
p. 238: 18
p. 239: 24(1)(3), 29, 30
\fi

\section*{15.9}

\begin{itemize}
	\item 交换律满足, 因为$\forall x, y \in \mathbb Q, x * y = x + y - xy = y + x - yx = y * x$.
	\item 结合律满足, 因为$\forall x, y, z \in \mathbb Q$, \begin{align*}
		(x * y) * z &= (x + y - xy) * z = x + y + z - xy - xz - yz + xyz\\
		x * (y * z) &= x * (y + z - yz) = x + y + z - xy - xz - yz + xyz\\
	\end{align*}
	\item 幂等律不满足, 比如$5 * 5 = 5 + 5 - 25 = -15 \neq 5$.
	\item 单位元是$0$, 因为$\forall x \in \mathbb Q, x * 0 = x + 0 - 0x = x$.
	\item 零元是$1$,因为$\forall x \in \mathbb Q, x * 1 = x + 1 - 1x = 1$.
	\item 对于$\forall x \neq 1$, 存在逆元$x^{-1} = \frac{x}{x-1}$, 因为$x * \frac{x}{x-1} = x + \frac{x}{x-1} - \frac{x^2}{x-1} = 0$.
\end{itemize}

\section*{15.11}

\begin{itemize}
	\item 交换律不满足, 比如说$\langle 1, 2 \rangle \circ \langle 3, 4 \rangle = \langle 3, 6 \rangle \neq \langle 3, 10 \rangle = \langle 3, 4 \rangle \circ \langle 1, 2 \rangle$.
	\item 结合律满足, 因为$\forall a, b, c, d, e, f \in \mathbb Q$, \begin{align*}
		(\langle a, b \rangle \circ \langle c, d \rangle) \circ \langle e, f \rangle &= \langle ac, ad + b \rangle \circ \langle e, f \rangle = \langle ace, acf + ad + b \rangle\\
		\langle a, b \rangle \circ (\langle c, d \rangle \circ \langle e, f \rangle) &= \langle a, b \rangle \circ \langle ce, cf + d \rangle = \langle ace, acf + ad + b \rangle\\
	\end{align*}
	\item 单位元是 $\langle 1, 0 \rangle$, 因为$\forall x, y \in \mathbb Q$, $\langle 1, 0 \rangle \circ \langle x, y \rangle = \langle x, y \rangle, \langle x, y \rangle \circ \langle 1, 0 \rangle = \langle x, y \rangle$.
	\item 零元不存在, 因为不可能存在 $z_1, z_2 \in \mathbb Q$ 使得 $\langle x, y \rangle \circ \langle z_1, z_2 \rangle = \langle xz_1, xz_2 + y \rangle$ 和 $\langle x, y' \rangle \circ \langle z_1, z_2 \rangle = \langle xz_1, xz_2 + y' \rangle$ 同时等于 $\langle z_1, z_2 \rangle$.
	\item $\forall a, b \in \mathbb Q$其中$a \neq 0$, 存在逆元$\langle a, b \rangle^{-1} = \langle \frac 1a, -\frac{b}{a} \rangle$.
\end{itemize}
\section*{15.14}

子代数有 $\langle \{0, 1, 2, 3, 4, 5\}, \oplus \rangle , \langle \{0, 2, 4\}, \oplus \rangle , \langle \{0, 3\}, \oplus \rangle , \langle \{0\}, \oplus \rangle $, 其中第一者是平凡子代数, 其余是真子代数.

\section*{15.16}
\begin{table}[h]
	\centering
	\begin{tabular}{c|cccccc}
		$\oplus$ & $\langle 0, 0 \rangle$ & $\langle 0, 1 \rangle$ & $\langle 1, 0 \rangle$ & $\langle 1, 1 \rangle$ & $\langle 2, 0 \rangle$ & $\langle 2, 1 \rangle$ \\ \hline
		$\langle 0, 0 \rangle$ & $\langle 0, 0 \rangle$ & $\langle 0, 1 \rangle$ & $\langle 1, 0 \rangle$ & $\langle 1, 1 \rangle$ & $\langle 2, 0 \rangle$ & $\langle 2, 1 \rangle$ \\
		$\langle 0, 1 \rangle$ & $\langle 0, 1 \rangle$ & $\langle 0, 0 \rangle$ & $\langle 1, 1 \rangle$ & $\langle 1, 0 \rangle$ & $\langle 2, 1 \rangle$ & $\langle 2, 0 \rangle$ \\
		$\langle 1, 0 \rangle$ & $\langle 1, 0 \rangle$ & $\langle 1, 1 \rangle$ & $\langle 2, 0 \rangle$ & $\langle 2, 1 \rangle$ & $\langle 0, 0 \rangle$ & $\langle 0, 1 \rangle$ \\
		$\langle 1, 1 \rangle$ & $\langle 1, 1 \rangle$ & $\langle 1, 0 \rangle$ & $\langle 2, 1 \rangle$ & $\langle 2, 0 \rangle$ & $\langle 0, 1 \rangle$ & $\langle 0, 0 \rangle$ \\
		$\langle 2, 0 \rangle$ & $\langle 2, 0 \rangle$ & $\langle 2, 1 \rangle$ & $\langle 0, 0 \rangle$ & $\langle 0, 1 \rangle$ & $\langle 1, 0 \rangle$ & $\langle 1, 1 \rangle$ \\
		$\langle 2, 1 \rangle$ & $\langle 2, 1 \rangle$ & $\langle 2, 0 \rangle$ & $\langle 0, 1 \rangle$ & $\langle 0, 0 \rangle$ & $\langle 1, 1 \rangle$ & $\langle 1, 0 \rangle$ \\
	\end{tabular}
	\caption{$V_1 \times V_2$的运算表}
	\label{tab}
\end{table}


\begin{enumerate}
	\item 见\cref{tab} .
	\item 单位元是 $\langle 0, 0 \rangle$.
			
	$\langle 0, 0 \rangle^{-1} = \langle 0, 0 \rangle$,
 		$\langle 0, 1 \rangle^{-1} = \langle 0, 1 \rangle$,
 		$\langle 1, 0 \rangle^{-1} = \langle 2, 0 \rangle$,
 		$\langle 1, 1 \rangle^{-1} = \langle 2, 1 \rangle$,
 		$\langle 2, 0 \rangle^{-1} = \langle 1, 0 \rangle$,
 		$\langle 2, 1 \rangle^{-1} = \langle 1, 1 \rangle$.

\end{enumerate}

\section*{15.18}
取$f: \mathbb C \to B$满足$f(a + bi) = \begin{pmatrix}
	a&b\\-b&a
\end{pmatrix}\ (\forall a, b \in \mathbb R)$.

容易看出 $f$ 是双射, 只需要再验证 $f$ 保运算:
\begin{align*}
	f((a + bi) + (c + di)) &= f(a + c + (b + d)i) \\&= \begin{pmatrix}
		a+c&b+d\\-b-d&a+c
	\end{pmatrix} = \begin{pmatrix}
		a&b\\-b&a
	\end{pmatrix} + \begin{pmatrix}
		c&d\\-d&c
	\end{pmatrix} \\&= f(a + bi) + f(c + di)\\
	f((a + bi)(c + di)) &= f((ac-bd) + (bc+ad)i) \\&= \begin{pmatrix}
		ac-bd & bc + ad \\ -bc - ad & ac - bd
	\end{pmatrix} = \begin{pmatrix}
		a&b\\-b&a
	\end{pmatrix} \begin{pmatrix}
		c&d\\-d&c
	\end{pmatrix} \\&= f(a + bi)f(c + di)
\end{align*}
故$V_1$ 与 $V_2$ 同构.
\section*{15.24}

\begin{enumerate}
	\item 不是同态.
	\item 是同态, $\im \varphi = \mathbb R_{\ge 0}$.
	\item 是同态, $\im \varphi = \{0\}$.
	\item 不是同态.
\end{enumerate}

\section*{15.29}
\begin{enumerate}
	\item 只需要验证 $\varphi$ 保运算: $$\varphi(\Delta x) = \varphi(x + 1) = (x + 1) \bmod 2 = (x \bmod 2 + 1) \bmod 2 = \overline{\Delta}(\varphi(x))$$
	\item $x \sim y \Leftrightarrow \varphi(x) = \varphi(y) \Leftrightarrow x \bmod 2 = y \bmod 2$, 因此$\varphi$在$V_1$上的划分是 $\mathbb Z = (2\mathbb Z) \sqcup (2\mathbb Z + 1)$.
\end{enumerate}
\section*{15.30}
\begin{enumerate}
	\item 只需要验证 $\varphi$ 保运算: $$\varphi(x + y) = n(x + y) = nx + ny = \varphi(x) + \varphi(y)$$
	\item $x \sim y \Leftrightarrow \varphi(x) = \varphi(y) \Leftrightarrow n(x - y) = 0$.
	
	\begin{itemize}
		\item 当 $n = 0$ 时, $x \sim y$ 对 $\forall x, y \in A_k$ 均成立, 从而$V_1 / \sim = \langle \{A_k\}, \oplus \rangle$, 其中 $A_k \oplus A_k = A_k$.

		\item 当 $n \neq 0$ 时, $x \sim y \Leftrightarrow x = y$, 从而$V_1 / \sim = \langle \{\{x\} | x \in A_k\}, \oplus \rangle$, 其中$\forall x, y \in A_k, \{x\} \oplus \{y\} = \{x + y\}$.
	\end{itemize}
\end{enumerate}

\end{document}
