\documentclass[8pt]{article}
\usepackage[UTF8]{ctex}
\usepackage[a4paper]{geometry}

\usepackage{amsthm,amsmath,amssymb}
\usepackage{graphicx}
\usepackage{subfigure}
\usepackage{amsmath}
\usepackage{tabularx}
\usepackage{color}
\usepackage{hyperref}
\usepackage{ulem}
\usepackage{multirow}
\usepackage[cache=false]{minted}
\hypersetup{
	colorlinks=true,
	linkcolor=blue
}

\usepackage{appendix}
\geometry{a4paper,centering,scale=0.8}
\geometry{left=2.0cm, right=2.0cm, top=2.5cm, bottom=2.5cm}
\usepackage[format=hang,font=small,textfont=it]{caption}
\usepackage[nottoc]{tocbibind}

\usepackage{algorithm}
\usepackage{algorithmicx}
\usepackage{algpseudocode}
\usepackage{amssymb}
\usepackage{extarrows}
\usepackage{qcircuit}
\usepackage{fancyhdr}
\usepackage{cleveref}
\usepackage{totpages}
\usepackage{pgf}
\usepackage{tikz}
\usepackage{bbm}
\usetikzlibrary{arrows,automata}
\usetikzlibrary{arrows.meta}%画箭头用的包

\makeatletter
\def\@maketitle{%
	\newpage
	\begin{center}%
		\let \footnote \thanks
		{\LARGE \@title \par}%
		\vskip 1.5em%
		{\large
			\lineskip .5em%
			\begin{tabular}[t]{c}%
				\@author
			\end{tabular}\par}%
		\vskip 1em%
		{\large \@date}%
	\end{center}%
	\par
	\vskip 1.5em}
\makeatother

\newtheoremstyle{compact}%
{3pt}{3pt}%
{}{}%
{\bfseries}{\textcolor{red}{.}}%  % Note that final punctuation is omitted.
{.5em}{\mbox{\textcolor{red}{\thmname{#1}\thmnumber{ #2}}\thmnote{ (\textcolor{blue}{#3})}}}
\theoremstyle{compact}
\newtheorem{innercustomgeneric}{\customgenericname}
\providecommand{\customgenericname}{}
\newcommand{\newcustomtheorem}[2]{%
	\newenvironment{#1}[1]
	{%
		\renewcommand\customgenericname{#2}%
		\renewcommand\theinnercustomgeneric{##1}%
		\innercustomgeneric
	}
	{\endinnercustomgeneric}
}

\DeclareMathOperator{\card}{card}

\newtheorem{theorem}{定理}
\newtheorem{lemma}{引理}
\newtheorem{definition}{定义}
\newtheorem{proposition}{命题}
\newtheorem{corollary}{推论}
\newtheorem{remark}{注}
\newtheorem{Proof}{证明}

\def\obj#1{\textbf{\uline{#1}}}
\def\num#1{\textnormal{\textbf{\mbox{\textcolor{blue}{(#1)}}}}}
\def\le{\leqslant}
\def\ge{\geqslant}
\def\im{\text{im }}
\def\Pr#1{\text{Pr}\left[{#1}\right]}
\def\E#1{\mathbb{E}\left[{#1}\right]}
\def\Var#1{\text{Var}\left[{#1}\right]}
\def\Enc{\textsf{Enc}}
\def\Dec{\textsf{Dec}}
\def\Gen{\textsf{Gen}}
\def\rep#1{\llcorner{#1}\lrcorner}

\title{\heiti\zihao{1} Fundamentals  of Cryptography \ Homework 3}
\author{\kaishu\zihao{-3} 周书予\\2000013060@stu.pku.edu.cn}

\CTEXoptions[today=old]
\date{\today}

\begin{document}\large
\fancypagestyle{plain}{
	\fancyhf{}
	\lhead{Fundamentals  of Cryptography}
	\chead{2022 Fall}
	\rhead{Homework 3}
	\cfoot{Page \thepage\ of \pageref{TotPages}}
}
\pagestyle{plain}



\crefname{theorem}{定理}{定理}
\crefname{lemma}{引理}{引理}
\crefname{figure}{图}{图}
\crefname{table}{表}{表}	
\maketitle

\def\PrivK#1#2{\textsf{PrivK}_{{#1}}^{\textsf{#2}}}

\section*{Problem 1}

$\textsf{Dec}(k, (r, c)) = F_k^{-1}(c) \oplus r$.

Denote $\Pi = (\textsf{Gen}, \textsf{Enc}, \textsf{Dec})$ as the encryption scheme mentioned in the problem, and $\tilde{\Pi} = (\widetilde{\textsf{Gen}}, \widetilde{\textsf{Enc}}, \widetilde{\textsf{Dec}})$ exactly the same as $\Pi$, except that a truly random permutation $f$ is used in place of $F_k$.

The proof is divided into two parts: \begin{itemize}
	\item In the first part we prove that for any PPT adversary $\mathcal A$, there is some negligible function $\varepsilon(n)$ such that \begin{equation}\left| \text{Pr}\left[\PrivK{\mathcal A, \Pi}{cpa}(n) = 1\right] - \text{Pr}\left[\PrivK{\mathcal A, \tilde{\Pi}}{cpa}(n) = 1\right] \right| < \varepsilon(n)\label{1-1}\end{equation}
	\item In the second part we show that for any PPT adversary $\mathcal A$, \begin{equation}\text{Pr}\left[\PrivK{\mathcal A, \tilde{\Pi}}{cpa}(n) = 1\right] \le \frac12 + \frac{q(n)}{2^n}\label{1-2}\end{equation} for some polynomial $q(n)$.
\end{itemize}

When finished the proof of the two parts mentioned above, one can see that obviously $\text{Pr}\left[\PrivK{\mathcal A, \Pi}{cpa}(n) = 1\right] \le \frac12 + \frac{q(n)}{2^n} + \varepsilon(n)$, which means $\Pi$ is secure under CPA attack.

\subsection*{Proof of \cref{1-1}}

For any PPT adversary $\mathcal A$, a PPT distinguisher $\mathcal D$ can be built, which has access to an oracle $\mathcal O: \{0, 1\}^n \to \{0, 1\}^n$ (here it refers to $F_k$ or $f$) and interacts with $\mathcal A$ like this:
\begin{enumerate}
	\item when $\mathcal A$ queries the ciphertext for message $m \in \{0, 1\}^n$, choose uniformly random $r \in \{0, 1\}^n$ and return $(r, \mathcal O(r \oplus m))$.
	\item when $\mathcal A$ outputs $m_0$ and $m_1$, choose a random bit $b \in \{0, 1\}$ and uniformly random $r \in \{0, 1\}^n$, then return $(r, \mathcal O(r \oplus m_b))$.
	\item continue answering $\mathcal A$'s queries until $\mathcal A$ outputs a bit $b'$, then output $\mathbbm 1[b = b']$.
\end{enumerate}

It is easy to see that \begin{align*}
	\begin{split}
		\text{Pr}\left[\PrivK{\mathcal A, \Pi}{cpa}(n) = 1\right] &= \text{Pr}_{k \gets \{0, 1\}^n}\left[\mathcal D^{F_k}(1^n) = 1\right] \\
		\text{Pr}\left[\PrivK{\mathcal A, \tilde{\Pi}}{cpa}(n) = 1\right] &= \text{Pr}_{f \gets \textsf{Perm}_n}\left[\mathcal D^{f}(1^n) = 1\right]
	\end{split}
\end{align*}
where $\textsf{Perm}_n$ denotes the collection of all permutations over $\{0, 1\}^n$.

Since $F$ is a PRP, by definition we know that $$\left| \text{Pr}_{k \gets \{0, 1\}^n}\left[\mathcal D^{F_k}(1^n) = 1\right] - \text{Pr}_{f \gets \textsf{Perm}_n}\left[\mathcal D^{f}(1^n) = 1\right] \right| < \varepsilon(n)$$ for some negligible $\varepsilon(n)$, so \cref{1-1} is proved as desired.

\subsection*{Proof of \cref{1-2}}
Notice that $\mathcal A$ runs in polynomial time, so it can only queries the ciphertext for polynomially many $m$, say, $q(n)$. Whenever $\mathcal A$ queries $m$ it obtains $f(r \oplus m)$ where $r$ is known to $\mathcal A$ and chosen uniformly random. That is, each query gives $\mathcal A$ a pair $(x, f(x))$ which is a point value of $f$, where $x = r \oplus m$ is chosen uniformly random.

When $\mathcal A$ outputs $m_0, m_1$ and receives $(r^*, f(r^* \oplus m_b))$, it checks out all the recordings from the interaction, and if the point value for $r^* \oplus m_0$ or $r^* \oplus m_1$ is found, it can break the encryption scheme with $100\%$ confidence, otherwise it learns nothing about $f(r^* \oplus m_0)$ and $f(r^* \oplus m_1)$, and probability of outputing the correct answer is exactly $1/2$.

The probability that the point value for $r^* \oplus m_0$ or $r^* \oplus m_1$ can be found equals to the probability of finding out two specific items among $2^n$ during $q(n)$ times of random choosing, which by union bound is not greater than $2q(n) / 2^n$. Thus, $$\text{Pr}\left[\PrivK{\mathcal A, \tilde{\Pi}}{cpa}(n) = 1\right] \le \frac{2q(n)}{2^n} \cdot 1 + \left(1 - \frac{2q(n)}{2^n}\right) \cdot \frac12 = \frac12 + \frac{q(n)}{2^n}$$


\section*{Problem 2}
\subsection*{Part A: $F'$ is a PRF}
First we show that for any PPT distinguisher $\mathcal D$, \begin{equation}
	\left| \text{Pr}_{k \gets \{0, 1\}^n}\left[\mathcal D^{g \circ F_k}(1^n)\right] - \text{Pr}_{f \gets \textsf{Func}_n}\left[\mathcal D^{g \circ f}(1^n)\right] \right| < \textsf{negl}(n)\label{2-A-1}
\end{equation}
(here $f_1 \circ f_2$ denotes the composition of function $f_1$ and $f_2$.)

This can be done by constructing another distinguisher $\mathcal D'$, which always queries the same message $m$ as $\mathcal D$ does except that the oracle used here is $F_k$ or $f$ instead of $g \circ F_k$ or $g \circ f$, and outputs the same as $\mathcal D$ does.

It is easy to see that \begin{align*}
	\begin{split}
		\text{Pr}_{k \gets \{0, 1\}^n}\left[\mathcal D^{g \circ F_k}(1^n)\right] &= \text{Pr}_{k \gets \{0, 1\}^n}\left[\mathcal D'^{F_k}(1^n)\right] \\
		\text{Pr}_{f \gets \textsf{Func}_n}\left[\mathcal D^{g \circ f}(1^n)\right] &= \text{Pr}_{f \gets \textsf{Func}_n}\left[\mathcal D'^{f}(1^n)\right] \\
	\end{split}
\end{align*}

Since $F$ is a PRF, from its definition it is clear to see that \cref{2-A-1} can be proved.

Then we can show that for any PPT distinguisher $\mathcal D$, \begin{equation}
	\left| \text{Pr}_{f \gets \textsf{Func}_n}\left[\mathcal D^{g \circ f}(1^n)\right] - \text{Pr}_{h \gets \textsf{Func}_{n, 2n}}\left[\mathcal D^{h}(1^n)\right] \right| < \textsf{negl}(n)\label{2-A-2}
\end{equation}
(here $\textsf{Func}_{n, 2n}$ is defined as $\{h: \{0, 1\}^n \to \{0, 1\}^{2n}\}$.)

This can be done by using hybird argument: assume $\mathcal D$ interacts with oracle for $p(n)$ rounds and, WLOG, we assume $\mathcal D$ never queries for the same $x$ for encryption (that is obviously suboptimal). Based on $\mathcal D$, distinguisher $\mathcal D'$ can be built, which on input $r \in \{0, 1\}^{2n}$ works as follows:
\begin{itemize}
	\item randomly sample $t$ from $\{1, 2, \cdots, p(n)\}$, and randomly fix some $f \gets \textsf{Func}_n$ and $h \gets \textsf{Func}_{n, 2n}$. ($f$ and $h$ do not need to be fully stored.)
	\item interact with $\mathcal D$. Whenever queried with $x$ in round $i$, return $\begin{cases}
		g(f(x)), & i < t \\
		r, & i = t \\
		h(x), & i > t
	\end{cases}$.
	\item output the same as $\mathcal D$ does.
\end{itemize}

From this construction we know that \begin{align*}
	\begin{split}
		\text{Pr}_{s \gets \{0, 1\}^n}\left[\mathcal D'(g(s)) = 1\right] &- \text{Pr}_{r \gets \{0, 1\}^{2n}}\left[\mathcal D'(r) = 1\right] \\&= \frac{1}{p(n)}\left(\text{Pr}_{f \gets \textsf{Func}_n}\left[\mathcal D^{g \circ f}(1^n)\right] - \text{Pr}_{h \gets \textsf{Func}_{n, 2n}}\left[\mathcal D^{h}(1^n)\right]\right)
	\end{split}
\end{align*}

Since $g$ is a PRG, both sides of the equation are negligible, and \cref{2-A-2} is proved as desired.

From \cref{2-A-1} and \cref{2-A-2} one can draw that \begin{equation}
	\left| \text{Pr}_{k \gets \{0, 1\}^n}\left[\mathcal D^{g \circ F_k}(1^n)\right] - \text{Pr}_{h \gets \textsf{Func}_{n, 2n}}\left[\mathcal D^{h}(1^n)\right] \right| < \textsf{negl}(n)
\end{equation}
which suggests that $F'_k = g \circ F_k$ is a PRF.

\subsection*{Part B: $F'$ may not be a PRF}
Let $g$ be a PRG which drops its first bit of input. It is easy to see that such PRG exists.

Then for any $x \in \{0, 1\}^{n-1}$, $F_k'(0 \| x) = F_k(g(0\|x)) = F_k(g(1\|x)) = F_k'(1\|x)$, which suggests that $F_k'$ is not that "random" and can be easily distinguished from a truly random function.

\section*{Problem 3}
\subsection*{Part A: $F'$ may not be a strong PRP}
3-round Feistel is a PRP but not a strong one.
\subsection*{Part B: $F'$ is a PRF}
For any distinguisher $\mathcal D'$ which tries to distinguish $F'$ from truly random functions, another distinguisher $\mathcal D$ can be built, which \begin{itemize}
	\item simulates $\mathcal D'$, when $\mathcal D'$ queries $x$, use its oracle and get $\mathcal O(x)$, and then return $x \oplus \mathcal O(x)$ to $\mathcal D'$.
	\item outputs whatever $\mathcal D'$ outputs.
\end{itemize}

It is easy to see that \begin{align*}
	\begin{split}
		\text{Pr}_{k \gets \{0, 1\}^n}\left[\mathcal D^{F_k}(1^n)\right] &= \text{Pr}_{k \gets \{0, 1\}^n}\left[\mathcal D'^{F'_k}(1^n)\right] \\
		\text{Pr}_{f \gets \textsf{Func}_n}\left[\mathcal D^{f}(1^n)\right] &= \text{Pr}_{f \gets \textsf{Func}_n}\left[\mathcal D'^{x \oplus f}(1^n)\right] \\
	\end{split}
\end{align*}

Thus $F'$ is a PRF.

\subsection*{Part C: $F'$ is a PRP}

First we can show that for any PPT distinguisher $\mathcal D$, $$\left| \text{Pr}_{k_1 \| k_2 \gets g(\$)}\left[\mathcal D^{F_{k_2} \circ F_{k_1}}(1^n) = 1\right] - \text{Pr}_{k_1, k_2 \gets \$}\left[\mathcal D^{F_{k_2} \circ F_{k_1}}(1^n) = 1\right] \right| < \textsf{negl}(n)$$
This is because $g$ is a PRG and no PPT distinguisher can distinguish $g(\$)$ from $\$ \| \$$ with non-negligible advantage.

Then we can show that for any PPT distinguisher $\mathcal D$, $$\left| \text{Pr}_{k_1, k_2 \gets \$}\left[\mathcal D^{F_{k_2} \circ F_{k_1}}(1^n) = 1\right] - \text{Pr}_{f_1, f_2 \gets \textsf{Perm}_n}\left[\mathcal D^{f_2 \circ f_1}(1^n) = 1\right] \right| < \textsf{negl}(n)$$
This is because $F$ itself is a PRP.

Together it has proved that $F'_k = F_{k_2} \circ F_{k_1}$ is a PRP.

\subsection*{Part D: $F'$ is a PRP}

First we show that for any PPT distinguisher $\mathcal D$, $$\left| \text{Pr}_{k \gets \{0, 1\}^n}\left[\mathcal D^{F_k \circ F_k}(1^n) = 1\right] - \text{Pr}_{f \gets \textsf{Perm}_n}\left[\mathcal D^{f \circ f}(1^n) = 1\right] \right| < \textsf{negl}(n)$$

This is because for any PPT distinguisher $\mathcal D$ which tries to distinguish $F_k \circ F_k$ from $f \circ f$, another distinguisher $\mathcal D'$ can be built, which queries the same as $\mathcal D$ does, uses its oracle twice to get $\mathcal O(\mathcal O(x))$, returns the result to $\mathcal D$ and finally outputs the same bit as $\mathcal D$. Thus \begin{align*}
	\begin{split}
		\text{Pr}_{k \gets \{0, 1\}^n}\left[\mathcal D^{F_k \circ F_k}(1^n)\right] &= \text{Pr}_{k \gets \{0, 1\}^n}\left[\mathcal D'^{F_k}(1^n)\right] \\
		\text{Pr}_{f \gets \textsf{Perm}_n}\left[\mathcal D^{f \circ f}(1^n)\right] &= \text{Pr}_{f \gets \textsf{Perm}_n}\left[\mathcal D'^{f}(1^n)\right] \\
	\end{split}
\end{align*}
since $F_k$ is a PRP, $\mathcal D$ can not have non-negligible advantage.

Then we are going to show that for any PPT distinguisher $\mathcal D$, $$\left|\text{Pr}_{f \gets \textsf{Perm}_n}\left[\mathcal D^{f \circ f}(1^n) = 1\right] - \text{Pr}_{f \gets \textsf{Perm}_n}\left[\mathcal D^{f}(1^n) = 1\right]\right| < \textsf{negl}(n)$$

WLOG we assume that $\mathcal D$ does not query the same $x$ more than once. If all $x^{(i)}, \mathcal O(x^{(i)})$ are distinct, we can assert that no distinguisher can figure out the correct answer with probability more than $1/2$, and the thing only fails with negligible probability, so the inequlity above holds for any PPT distinguisher.

Combining these two statements we can show that $F'_k = F_k \circ F_k$ is a PRP.

\section*{Problem 4}
\subsection*{Part A: $F(k, x \oplus c) = F(k, x) \oplus c$}
A distinguisher $\mathcal D$ simply queries the encryption for $0^n$ and $1^n$ and checks whether $\mathcal O(0^n) \oplus \mathcal O(1^n) = 1^n$, then outputs $1$ if the equation holds and $0$ otherwise.

It is easy to see that \begin{align*}
	\begin{split}
		\text{Pr}_{k \gets \{0, 1\}^n}\left[\mathcal D^{F_k}(1^n) = 1\right] &= 1 \\
		\text{Pr}_{f \gets \textsf{Func}_n}\left[\mathcal D^{f}(1^n) = 1\right] &= 1 / 2^n
	\end{split}
\end{align*}
so $\mathcal D$ breaks $F_k$ as PRF.

\subsection*{Part B: $F(k \oplus c, x) = F(k, x) \oplus c$}
A distinguisher $\mathcal D$ can be built, which \begin{itemize}
	\item calculates $F(0^n, 0^n)$.
	\item queries the encryption for $0^n$, say, $\mathcal O(0^n)$.
	\item learns the "key" $k' = F(0^n, 0^n) \oplus \mathcal O(0^n)$.
	\item checks the "key" is correct or not, that is, randomly choose $x \in \{0, 1\}^n$ and checks whether $F(k', x) = \mathcal O(x)$, then outputs $1$ is key is correct and $0$ otherwise.
\end{itemize}

It is easy to see that \begin{align*}
	\begin{split}
		\text{Pr}_{k \gets \{0, 1\}^n}\left[\mathcal D^{F_k}(1^n) = 1\right] &= 1 \\
		\text{Pr}_{f \gets \textsf{Func}_n}\left[\mathcal D^{f}(1^n) = 1\right] &= 1 / 2^n
	\end{split}
\end{align*}
so $\mathcal D$ breaks $F_k$ as PRF.

\subsection*{Part C: $F(k_1 \oplus k_2, x) = F(k_1, x) \oplus F(k_2, x)$}

Let $\varepsilon_i = 0^{i-1}10^{n-i}$ be the $i$-th "unit vector". For any $k, x \in \{0, 1\}^n$, if $k$ has $1$ on bit $i_1, i_2, \cdots, i_l$, then we have $F(k, x) = \bigoplus\limits_{j=1}^{l}F(\varepsilon_{i_j}, x)$.

A distinguisher $\mathcal D$ first calculates $\alpha_i = F(\varepsilon_i, 0^n)$ and then checks whether the linear span $\textsf{span}\left(\{\alpha_i\}_{i=1}^n\right)$ equals to the whole linear space $\{0, 1\}^n$.

\begin{itemize}
	\item If the answer is yes, then for any $y \in \{0, 1\}^n$ there exists exactly one set $S \subseteq \{1, \cdots, n\}$ such that $y = \bigoplus\limits_{i \in S}\alpha_i$. This means that the distinguisher $\mathcal D$ can learn the key by finding the set $S$ such that $\mathcal O(0^n) = \bigoplus\limits_{i \in S}\alpha_i$ can let key be $k = \bigoplus\limits_{i\in S}\varepsilon_i$.
	
	After learning the key, $\mathcal D$ takes a single test just like the one in \textbf{Part B} does, and then outputs the testing result.

	In this case we can show that \begin{align*}
		\begin{split}
			\text{Pr}_{k \gets \{0, 1\}^n}\left[\mathcal D^{F_k}(1^n) = 1\right] &= 1 \\
			\text{Pr}_{f \gets \textsf{Func}_n}\left[\mathcal D^{f}(1^n) = 1\right] &= 1 / 2^n
		\end{split}
	\end{align*}
	so $\mathcal D$ breaks $F_k$ as PRF.
	\item If the answer is no, then for at least half of $y \in \{0, 1\}^n$, it can not be represented as the linear combination of $\alpha_i$. Then $\mathcal D$ simply checks whether $\mathcal O(0^n)$ can be represented and ouputs the result.
	
	In this case we can show that \begin{align*}
		\begin{split}
			\text{Pr}_{k \gets \{0, 1\}^n}\left[\mathcal D^{F_k}(1^n) = 1\right] &= 1 \\
			\text{Pr}_{f \gets \textsf{Func}_n}\left[\mathcal D^{f}(1^n) = 1\right] &\le 1/2
		\end{split}
	\end{align*}
	so $\mathcal D$ also breaks $F_k$ as PRF.
\end{itemize}

(All linear algebra calculation can be done in polynomial time via algorithms like Gauss-Jordan Elimination, so $\mathcal D$ is indeed a PPT distinguisher.)

\section*{Problem 5}
Assume there is a length-doubling PRG $G: \{0, 1\}^n \to \{0, 1\}^{2n}$, and we use the notation $G_0(s)$ and $G_1(s)$ to describe the former and latter half of $G(s)$ respectively, that is \begin{align*}
	G(s) = G_0(s) \| G_1(s)
\end{align*}
\begin{itemize}
	\item $F$ is constructed as $$F(k, x) = G_{x_1} G_{x_2} \cdots G_{x_n}(k)$$ (this means the composition of $n$ $G_{0/1}$ functions, the notation $\circ$ is omitted for simplicity.) it is easy to see that $F$ runs in polynomial time.
	\item The puncture function is constructed as $$\textsf{puncture}(k, x) = \left\{\left(\overline{x_i} \| x_{i+1 \cdots n}, G_{\overline{x_i}} G_{x_{i+1}} \cdots G_{x_n}(k)\right) \bigg| i = 1, 2, \cdots, n\right\}$$
	\item The eval function $\textsf{eval}(k_{-x}, x')$ first finds one satisfying that $str$ is a suffix of $x'$ among all $(str, key)$ pairs in $k_{-x}$ (we assert that there exsits and only exsits one such pair since $x' \neq x$), and than calculates $\textsf{eval}(k_{-x}, x') = G_{x'_1} G_{x'_2} \cdots G_{x'_{n-|str|}}(key)$.
	
	It is not hard to see that $\textsf{eval}(k_{-x}, x') = F_k(x')$ for all $x' \neq x$.
\end{itemize}

We need to further prove that \begin{itemize}
	\item $F$ is a PRF. By hybird argument, consider $$F_{h_i}(x) = G_{x_1}G_{x_2}\cdots G_{x_i}(h_i(x_{i+1}\cdots x_n))$$ where $h_i: \{0, 1\}^{n-i} \to \{0, 1\}^n$ is a truly random function. Clearly $F_{h_0}$ is actually a truly random function over $\{0, 1\}^n$, and $F_{h_n}$ is equivalent to $F$.
	
	Since $G$ is a PRG, no PPT distinguisher can distinguish $F_{h_i}$ from $F_{h_{i+1}}$ with non-negligible advantage, so $F_{h_0}$ can not be distinguished from $F_{h_n}$ with non-negligible advantage, which means that $F$ is a PRF.
	\item $F$ is puncturable. Here hybird argument is used again, considering such parameterized security game: 
	\begin{itemize}
		\item With parameter $i$, the distinguisher $\mathcal D$ chooses $x$, the challenger samples random $k, u$, computes $k_{-x}$, and sends $\begin{cases}(k_{-x}, G_{x_1}\cdots G_{x_i}(u)),& i < n \\(k_{-x}, G_{x_1}\cdots G_{x_n}(k)), & i = n\end{cases}$ to the distinguisher.
	\end{itemize}

	Obviously the two cases in the problem are equivalent to game $i=n$ and $i=0$ respectively.

	Since $G$ is a PRG, no PPT distinguisher can distinguish security games with adjacent parameter $i$, so the two cases are also indistinguishable, with non-negligible advantage.
\end{itemize}
\section*{Problem 6}
\subsection*{Part A}
The distinguisher $\mathcal D$ can implement the following steps to break 3-round Feistel: \begin{itemize}
	\item Query $\textsf{Dec}(0^n, 0^n) \to (a, b)$.
	\item Query $\textsf{Enc}(0^n, b) \to (c, d)$.
	\item Query $\textsf{Dec}(c, a\oplus d) \to (e, f)$.
	\item Output $1$ is $f = c \oplus b$ and $0$ otherwise.
\end{itemize}

Here's why it works: under 3-round Feistel encryption scheme, the first query reveals \begin{align*}
	\begin{split}
		a &= F_{k_3}(0^n) \oplus F_{k_1}(b) \\
		b &= F_{k_2}(F_{k_3}(0^n))
	\end{split}
\end{align*}
and the second query \begin{align*}
	\begin{split}
		c &= b \oplus F_{k_2}(F_{k_1}(b)) \\
		d &= F_{k_1}(b) \oplus F_{k_3}(c)
	\end{split}
\end{align*}
and so $$a \oplus d = F_{k_3}(0^n) \oplus F_{k_3}(c)$$ 

In the final query, the distinguisher knows that $$f = c \oplus F_{k_2}(F_{k_3}(0^n)) = c \oplus b$$.

So distinguisher $\mathcal D$ outputs $1$ with probability $1$ under 3-round Feistel, but with probability $1/2^n$ under truly random permutations, which means that 3-round Feistel is not a strong PRP.

\subsection*{Part B}
这咋做啊
\end{document}
