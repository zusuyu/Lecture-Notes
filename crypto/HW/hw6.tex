\documentclass[8pt]{article}
\usepackage[UTF8]{ctex}
\usepackage[a4paper]{geometry}

\usepackage{amsthm,amsmath,amssymb}
\usepackage{graphicx}
\usepackage{subfigure}
\usepackage{amsmath}
\usepackage{tabularx}
\usepackage{color}
\usepackage{hyperref}
\usepackage{ulem}
\usepackage{multirow}
\usepackage[linewidth=1pt]{mdframed}
\usepackage[cache=false]{minted}
\hypersetup{
	colorlinks=true,
	linkcolor=blue
}

\usepackage{appendix}
\geometry{a4paper,centering,scale=0.8}
\geometry{left=2.0cm, right=2.0cm, top=2.5cm, bottom=2.5cm}
\usepackage[format=hang,font=small,textfont=it]{caption}
\usepackage[nottoc]{tocbibind}

\usepackage{algorithm}
\usepackage{algorithmicx}
\usepackage{algpseudocode}
\usepackage{amssymb}
\usepackage{extarrows}
\usepackage{qcircuit}
\usepackage{fancyhdr}
\usepackage{cleveref}
\usepackage{totpages}
\usepackage{pgf}
\usepackage{tikz}
\usepackage{bbm}
\usetikzlibrary{arrows,automata}
\usetikzlibrary{arrows.meta}%画箭头用的包

\makeatletter
\def\@maketitle{%
	\newpage
	\begin{center}%
		\let \footnote \thanks
		{\LARGE \@title \par}%
		\vskip 1.5em%
		{\large
			\lineskip .5em%
			\begin{tabular}[t]{c}%
				\@author
			\end{tabular}\par}%
		\vskip 1em%
		{\large \@date}%
	\end{center}%
	\par
	\vskip 1.5em}
\makeatother

\newtheoremstyle{compact}%
{3pt}{3pt}%
{}{}%
{\bfseries}{\textcolor{red}{.}}%  % Note that final punctuation is omitted.
{.5em}{\mbox{\textcolor{red}{\thmname{#1}\thmnumber{ #2}}\thmnote{ (\textcolor{blue}{#3})}}}
\theoremstyle{compact}
\newtheorem{innercustomgeneric}{\customgenericname}
\providecommand{\customgenericname}{}
\newcommand{\newcustomtheorem}[2]{%
	\newenvironment{#1}[1]
	{%
		\renewcommand\customgenericname{#2}%
		\renewcommand\theinnercustomgeneric{##1}%
		\innercustomgeneric
	}
	{\endinnercustomgeneric}
}

\DeclareMathOperator{\card}{card}

\newtheorem{theorem}{定理}
\newtheorem{lemma}{引理}
\newtheorem{definition}{定义}
\newtheorem{proposition}{命题}
\newtheorem{corollary}{推论}
\newtheorem{remark}{注}
\newtheorem{Proof}{证明}

\def\obj#1{\textbf{\uline{#1}}}
\def\num#1{\textnormal{\textbf{\mbox{\textcolor{blue}{(#1)}}}}}
\def\le{\leqslant}
\def\ge{\geqslant}
\def\im{\text{im }}
\def\Pr#1{\text{Pr}\left[{#1}\right]}
\def\E#1{\mathbb{E}\left[{#1}\right]}
\def\Var#1{\text{Var}\left[{#1}\right]}
\def\Enc{\textsf{Enc}}
\def\Dec{\textsf{Dec}}
\def\Gen{\textsf{Gen}}
\def\rep#1{\llcorner{#1}\lrcorner}

\title{\heiti\zihao{1} Fundamentals  of Cryptography \ Homework 6}
\author{\kaishu\zihao{-3} 周书予\\2000013060@stu.pku.edu.cn}

\CTEXoptions[today=old]
\date{\today}

\begin{document}\large
\fancypagestyle{plain}{
	\fancyhf{}
	\lhead{Fundamentals  of Cryptography}
	\chead{2022 Fall}
	\rhead{Homework 6}
	\cfoot{Page \thepage\ of \pageref{TotPages}}
}
\pagestyle{plain}



\crefname{theorem}{定理}{定理}
\crefname{lemma}{引理}{引理}
\crefname{figure}{图}{图}
\crefname{table}{表}{表}	
\maketitle

\def\Gen{\textsf{Gen}}
\def\Enc{\textsf{Enc}}
\def\Dec{\textsf{Dec}}
\def\Pr{\text{Pr}}
\def\poly{\text{poly}}
\def\negl{\text{negl}}

\section*{Problem 1}
\subsection*{Part A}
$$\Dec(sk, ct = (\mathbf t, v)) = \begin{cases}
	0, & v - \mathbf{s}^T\mathbf{t} \in [-\frac q2, \frac q2] \\
	1, & \text{otherwise}
\end{cases}$$
since one can see that $v - \mathbf{s}^T\mathbf{t} = \mathbf{e}^T\mathbf{r} + \lfloor\frac q2\rfloor \cdot b$, so it is "small" if $b = 0$, and "big" if $b = 1$.

Notice that $|\mathbf{e}^T\mathbf{r}| \le Bm$, so to decrypt correctly, we'd like to have the constraint that $Bm \le \frac q4$.
\subsection*{Part B}
\subsubsection*{Hybrid 0 and hybrid 1 are computationally indistinguishable}

FSOC assume PPT distinguisher $\mathcal D$ distinguishes hybrid 0 from hybrid 1 with non-negligible advantage. Then another distinguisher $\mathcal D'$ can be built, which takes $(\mathbf A, \mathbf b^T) = pk$ as input, randomly samples $\mathbf r \gets \{0, 1\}^m$, send $(pk, ct = (\mathbf{Ar}, \mathbf b^T\mathbf r + \lfloor\frac q2\rfloor \cdot b))$ to distinguisher $\mathcal D$, and finally outputs whatever $\mathcal D$ outputs.	

So $\mathcal D'$ distinguishes $(\mathbf A, \mathbf s^T\mathbf A + \mathbf e^T)$ from $(\mathbf A, \mathbf b^T)$, which breaks LWE assumption.

\subsubsection*{Hybrid 1 and hybrid 2 are statistically indistinguishable}

\begin{center}
	\begin{mdframed}

		\includegraphics[scale=0.35]{Leftover.png}
		
	\end{mdframed}
\end{center}

Here each row of matrix $\mathbf A$, $\mathbf a_1, \cdots, \mathbf a_n$, together with $\mathbf b$, can be regarded as $s_1, \cdots, s_m$. $\mathbf r$ can be regarded as hash function key $k$, which means that the hash function $\mathbb Z_q^m \times \mathbb Z_q^m \to \mathbb Z_q$ is defined as "dot product".

Leftover Hash Lemma shows that the statistical difference between $(\mathbf{Ar}, \mathbf{b}^T\mathbf{r})$ and $(\mathbf{a}, v)$ $$\delta \le \frac12(n+1)\sqrt{q \cdot q^{-m}} = \frac{(n+1)q^{-(m-1)/2}}{2}$$

When $\frac{(n+1)q^{-(m-1)/2}}{2} = \negl(n)$, we can say that hybrid 1 and hybrid 2 are statistically indistinguishable.

\section*{Problem 2}
\subsection*{Part A}
For all $0 \le i < N$, we have $$(1 + N)^i = 1 + iN \in \mathbb G_N$$

For any $0 \le i < j < N$, $1 + iN \neq 1 + jN$, which means that $|\{(1 + N)^i | 0 \le i < N\}| = N$, thus $1 + N$ is a generator of $\mathbb G_N$.
\subsection*{Part B}
Every element in $\mathbb G_n$ can be written as $1 + kN$, for some $0 \le k < N$.

Suppose that $g = 1 + xN$ and $g^a = 1 + yN$ (here $x$ and $y$ can be computed efficiently), we know that $(1 + xN)^a = 1 + yN$, thus $ax \equiv y \bmod N$.
\begin{itemize}
	\item If $x$ is invertable module $N$, one can simply calculates $a' = y \cdot x^{-1} \bmod N$.
	\item If $p | x$ and $q \nmid x$, then we must have $p | y$, so one can calculates $a' = \left(\frac yp\right) \cdot \left(\frac xp\right)^{-1} \bmod q$.
	\item If $q | x$ and $p \nmid x$, we must have $q | y$, so one can calculates $a' = \left(\frac yq\right) \cdot \left(\frac xq\right)^{-1} \bmod p$.
	\item If $x = 0$, then $y = 0$, $a'$ can be any integer.
\end{itemize}

\subsection*{Part C}
Uniformly randomly sample $x$ from $\mathbb{QR}_{N^2}$, then output $x^N$.

\begin{itemize}
	\item $x$ can be written as $x = gh$ such that $g \in \mathbb G_N$ and $h \in \mathbb H_N$. Since $|\mathbb G_N| = N$, $g^N = 1$ holds for all $g \in \mathbb G_N$. So $x^N = g^Nh^N = h^N \in \mathbb H_N$.
	\item Since $\mathbb{QR}_{N^2} = \mathbb G_N \times \mathbb H_N$, so uniform $x$ over $\mathbb{QR}_{N^2}$ implies uniform $h$ over $\mathbb H_N$. Notice that $\gcd(N, p'q') = 1$, so $N$ is invertable module $p'q'$, and $h^N$ is also uniform over $\mathbb H_N$.
\end{itemize}

\subsection*{Part D}
$$\Dec(sk, c) = \textsf{discrete-log}(c^{p'q'}) \cdot (p'q')^{-1} \bmod N$$

Notice that $|\mathbb H_N| = p'q'$, so we must have $c^{p'q'} = (h \cdot (1 + N)^m)^{p'q'} = 1 + mp'q'N\in \mathbb G_N$, and $p'q'$ is invertable module $N$, which gives the chance to reveal the message and then make the public-key encryption scheme correct.

FSOC assume PPT adversary $\mathcal A$ breaks this encryption scheme as EAV-secure (notice that EAV-secure is equivalent to CPA-secure under public-key encryption scheme settings). We construct a PPT distinguisher $\mathcal D$ who distinguishes $(N, h)$ from $(N, x)$ for $h \gets \mathbb H_N$ and $x \gets \mathbb{QR}_{N^2}$, and thus breaks DCR assumption.

When $\mathcal D$ receives $(N, y)$, it calls $\mathcal A$ with public key $pk = N$. $\mathcal A$ outputs two distinct messages $m_0, m_1$, and $\mathcal D$ picks $r \in \mathbb H_N$ and $b \in \{0, 1\}$ uniformly at random. Instead of returning $c = h \cdot (1 + N)^{m_b}$, it returns $c = h \cdot y \cdot (1 + N)^{m_b}$. Then $\mathcal A$ outputs a single bit $b'$, and $\mathcal D$ outputs $1$ iff $b' = b$.

Then here are two cases: \begin{itemize}
	\item If $y$ is sampled from $\mathbb H_N$, then the encryption is "right", which means that $\mathcal A$ outputs the correct answer $b$ with probability at least $\frac12 + \frac{1}{\poly(n)}$, so $\mathcal D$ outputs $1$ with probability at least $\frac12 + \frac{1}{\poly(n)}$.
	\item If $y$ is sampled from $\mathbb{QR}_{N^2}$, then the encryption is "corrupt", which means that the ciphertext $c$ output by challenger is the encryption of a random message, and thus $\mathcal A$ can do nothing but random guessing with this random ciphertext, which makes $\mathcal D$ outputs $1$ with probability at most a half.
\end{itemize}

Thus $\mathcal D$ distinguishes $(N, h)$ from $(N, x)$ with non-negligible advantage, which breaks DCR assumption.

\end{document}
