\documentclass[8pt]{article}
\usepackage[UTF8]{ctex}
\usepackage[a4paper]{geometry}

\usepackage{amsthm,amsmath,amssymb}
\usepackage{graphicx}
\usepackage{subfigure}
\usepackage{amsmath}
\usepackage{tabularx}
\usepackage{color}
\usepackage{hyperref}
\usepackage{ulem}
\usepackage{multirow}
\usepackage[cache=false]{minted}
\hypersetup{
	colorlinks=true,
	linkcolor=blue
}

\usepackage{appendix}
\geometry{a4paper,centering,scale=0.8}
\geometry{left=2.0cm, right=2.0cm, top=2.5cm, bottom=2.5cm}
\usepackage[format=hang,font=small,textfont=it]{caption}
\usepackage[nottoc]{tocbibind}

\usepackage{algorithm}
\usepackage{algorithmicx}
\usepackage{algpseudocode}
\usepackage{amssymb}
\usepackage{extarrows}
\usepackage{qcircuit}
\usepackage{fancyhdr}
\usepackage{cleveref}
\usepackage{totpages}
\usepackage{pgf}
\usepackage{tikz}
\usepackage{bbm}
\usetikzlibrary{arrows,automata}
\usetikzlibrary{arrows.meta}%画箭头用的包

\makeatletter
\def\@maketitle{%
	\newpage
	\begin{center}%
		\let \footnote \thanks
		{\LARGE \@title \par}%
		\vskip 1.5em%
		{\large
			\lineskip .5em%
			\begin{tabular}[t]{c}%
				\@author
			\end{tabular}\par}%
		\vskip 1em%
		{\large \@date}%
	\end{center}%
	\par
	\vskip 1.5em}
\makeatother

\newtheoremstyle{compact}%
{3pt}{3pt}%
{}{}%
{\bfseries}{\textcolor{red}{.}}%  % Note that final punctuation is omitted.
{.5em}{\mbox{\textcolor{red}{\thmname{#1}\thmnumber{ #2}}\thmnote{ (\textcolor{blue}{#3})}}}
\theoremstyle{compact}
\newtheorem{innercustomgeneric}{\customgenericname}
\providecommand{\customgenericname}{}
\newcommand{\newcustomtheorem}[2]{%
	\newenvironment{#1}[1]
	{%
		\renewcommand\customgenericname{#2}%
		\renewcommand\theinnercustomgeneric{##1}%
		\innercustomgeneric
	}
	{\endinnercustomgeneric}
}

\DeclareMathOperator{\card}{card}

\newtheorem{theorem}{定理}
\newtheorem{lemma}{引理}
\newtheorem{definition}{定义}
\newtheorem{proposition}{命题}
\newtheorem{corollary}{推论}
\newtheorem{remark}{注}
\newtheorem{Proof}{证明}

\def\obj#1{\textbf{\uline{#1}}}
\def\num#1{\textnormal{\textbf{\mbox{\textcolor{blue}{(#1)}}}}}
\def\le{\leqslant}
\def\ge{\geqslant}
\def\im{\text{im }}
\def\Pr#1{\text{Pr}\left[{#1}\right]}
\def\E#1{\mathbb{E}\left[{#1}\right]}
\def\Var#1{\text{Var}\left[{#1}\right]}
\def\Enc{\textsf{Enc}}
\def\Dec{\textsf{Dec}}
\def\Gen{\textsf{Gen}}
\def\rep#1{\llcorner{#1}\lrcorner}

\title{\heiti\zihao{1} Fundamentals  of Cryptography \ Homework 7}
\author{\kaishu\zihao{-3} 周书予\\2000013060@stu.pku.edu.cn}

\CTEXoptions[today=old]
\date{\today}

\begin{document}\large
\fancypagestyle{plain}{
	\fancyhf{}
	\lhead{Fundamentals  of Cryptography}
	\chead{2022 Fall}
	\rhead{Homework 7}
	\cfoot{Page \thepage\ of \pageref{TotPages}}
}
\pagestyle{plain}



\crefname{theorem}{定理}{定理}
\crefname{lemma}{引理}{引理}
\crefname{figure}{图}{图}
\crefname{table}{表}{表}	
\maketitle

\def\Gen{\textsf{Gen}}
\def\Enc{\textsf{Enc}}
\def\Dec{\textsf{Dec}}
\def\Sign{\textsf{Sign}}
\def\Verify{\textsf{Verify}}
\def\Pr{\text{Pr}}
\def\poly{\text{poly}}

\def\Agg{\textsf{Aggregate}}

\section*{Problem 1}
\begin{align*}
	\begin{split}
		\Agg(m_1, \sigma_1, \cdots, m_{\ell}, \sigma_{\ell}) &= \prod_{i=1}^{\ell}\sigma_i \\
		\Verify(pk, m_1, \cdots, m_{\ell}, \sigma) &= \left[\sigma^{e} \equiv \prod_{i=1}^{\ell}H(m_i) \bmod N\right]
	\end{split}
\end{align*}

Assume there is a PPT adversary (the forger) $\mathcal A$ who braks this aggregate signature scheme, we'll show that we can construct another PPT adversary $\mathcal A'$ who breaks RSA assumption.

$\mathcal A'$ works as follows: \begin{itemize}
	\item $\mathcal A'$ is given $N, y, e$ as described in RSA assumption. The goal of $\mathcal A'$ is to find $x$ such that $x^e \equiv y \bmod N$.
	\item $\mathcal A'$ prepares the answer for the oracle queries that $\mathcal A$ will ask, by choosing at random $r_1, \cdots, r_{p(n)} \in \mathbb Z_n^*$ and an index $j \in \{1, \cdots, p(n)\}$, and let the answer to the $i$-th query to $H$ be $$H(m_i) = \begin{cases}
		y \cdot r_i^e, & i = j \\
		r_i^e, & i \neq j
	\end{cases}$$ where $p(n)$ is some polynomial.
	
	Intuitively, $\mathcal A'$ is betting on the chance that $\mathcal A$ will include its $j$-th oracle query in its output (the forgery).

	\item $\mathcal A'$ prepares the answer for the signature queries that $\mathcal A$ will ask. Typically when asked $\Sign(sk, m_i)$ for some $i \neq j$, $\mathcal A'$ simply returns $r_i$. When asked $\Sign(sk, m_j)$, $\mathcal A'$ fails.
	\item $\mathcal A'$ runs the forger algorithm $\mathcal A$, and answer its both oracle and signature queries.
	\item $\mathcal A'$ retrieves the output $(m_{i_1}, \cdots, m_{i_{\ell}}, \sigma)$ of $\mathcal A$. If $j \notin \{i_1, \cdots, i_{\ell}\}$ it fails, otherwise $\mathcal A'$ knows that w.p. at least $1 / \poly(n)$, $$\sigma^e = \prod_{k=1}^{\ell}H(m_{i_k}) = y \cdot \prod_{k=1}^{\ell}r_{i_k}^e$$ then it simply outputs $x = \sigma \cdot \prod\limits_{k=1}^{\ell}r_{i_k}^{-1}$ such that $x^e = y$.
\end{itemize}

One can see that $$\Pr\left[\mathcal A' \text{ breaks RSA assumption}\right] \ge \frac{1}{p(n)}\Pr\left[\mathcal A \text{ breaks this aggregate signature scheme}\right] = \frac{1}{\poly(n)}$$

So $\mathcal A'$ indeed breaks RSA assumption.

\section*{Problem 2}

\subsection*{Part A}
The signature algorithm $\Sign$ first finds $H(m)^{-1} \bmod \varphi(N)$, which is efficient via 辗转相除. Then it uses the fast power algorithm to calculate $g^{H(m)^{-1}} \bmod N$.

Notice that addition and multiplication operation of $\mathbb Z_N$ elements can be down in $\poly(n)$ time, and only $\poly(n)$ operations are needed in the above process, thus all thins can be done in $\poly(n)$-time.

\subsection*{Part B}
Assume there is a PPT adversary $\mathcal A$ who braks Gennaro-Halevi-Rabin signature scheme, we'll show that we can construct another PPT adversary $\mathcal A'$ who breaks RSA assumption.

$\mathcal A'$ works as follows: \begin{itemize}
	\item $\mathcal A'$ is given $N, y, e$ as described in RSA assumption. The goal of $\mathcal A'$ is to find $x$ such that $x^e \equiv y \bmod N$.
	\item $\mathcal A'$ prepares the answer for the oracle queries that $\mathcal A$ will ask, by choosing at random $e_1, \cdots, e_{p(n)} \in \mathbb Z_n^*$ (except $e_j$, it sets $e_j = e$ without sampling) and an index $j \in \{1, \cdots, p(n)\}$, and let the answer to the $i$-th query to $H$ be $H(m_i) = e_i$. Let $E = \prod\limits_{i \neq j}e_i$, $\mathcal A'$ sets $g = y^E \bmod N$ as public key and sends it to $\mathcal A$.
	
	For simplicity, we suppose that $\mathcal A'$ chooses such $e_i$-s that $\gcd(e_i, e) = 1$ for all $i \neq j$, which means that $\gcd(E, e) = 1$. In the real world, this succeeds with probability at least $1 / \poly(n)$.

	Intuitively, $\mathcal A'$ is betting on the chance that $\mathcal A$ will use its $j$-th oracle query as its output (the forgery).

	\item $\mathcal A'$ prepares the answer for the signature queries that $\mathcal A$ will ask. Typically when asked $\Sign(sk, m_i)$ for some $i \neq j$, $\mathcal A'$ simply returns $y^{E / e_i}$ (notice that $E / e_i$ is an integer when $i \neq j$). When asked $\Sign(sk, m_j)$, $\mathcal A'$ fails.

	\item $\mathcal A'$ runs the forger algorithm $\mathcal A$, and answer its both oracle and signature queries.

	\item $\mathcal A'$ retrieves the output $(m, \sigma)$ of $\mathcal A$. If $m \neq m_j$ it fails, otherwise $\mathcal A'$ knows that w.p. at least $1 / \poly(n)$, $$\sigma^{e} = y^{E}$$ now $\mathcal A'$ computes two integer $a, b$ such that $ae + bE = 1$ by extended Euclidean algorithm, and outputs $$x = \sigma^b \cdot y^a$$ such that $$x^e = (\sigma^{e})^b\cdot y^{ae} = y^{bE}\cdot y^{ae} = y$$
\end{itemize}

\subsection*{Part C}

$$\Verify'(pk, m_1, \cdots, m_{\ell}, \sigma) = \left[\sigma^{\prod_{i}H(m_i)} = g^{\sum_i\prod_{j \neq i}H(m_j)}\right]$$

\subsection*{Part D}

(We assume that $m_1, \cdots, m_{\ell}$ be a sequence of distinct messages such that all $H(m_i)$-s are pairwise-coprime.)

With aggregate signature $\sigma$ and messages $m_1, \cdots, m_{\ell}$ given, one can calculate $\sigma^{\prod_{i \neq 1}H(m_i)}$ which gives the value $$v_1 = \sigma^{\prod_{i \neq 1}H(m_i)} = g^{\frac{\prod_{i \neq 1}H(m_i)}{H(m_1)} + \sum_{j=2}^{\ell}\prod_{i \neq 1, i \neq j}H(m_i)}$$

Since for all $j \in \{2, \cdots, \ell\}$, the value $g^{\prod_{i \neq 1, i \neq j}H(m_i)}$ can be calculated efficiently, one can further obtain the value $$v_2 = v_1 / \prod_{j=2}^{\ell}g^{\prod_{i \neq 1, i \neq j}H(m_i)} = g^{\frac{\prod_{i \neq 1}H(m_i)}{H(m_1)}}$$

Notice that by our assumption, $\gcd\left(H(m_1), \prod_{i \neq 1} H(m_i)\right) = 1$, thus we can compute two integer $a, b$ such that $aH(m_1) + b\prod_{i \neq 1} H(m_i) = 1$, and it follows $$v_3 = g^{a}v_2^{b} = g^{\frac{aH(m_1) + b\prod_{i \neq 1} H(m_i)}{H(m_1)}} = g^{1 / H(m_1)}$$ which gives the signature of message $m_1$.

\section*{Problem 3}
Notice that the equation $$x^e = g \bmod N$$ has and only has one solution $$x = g^d \bmod N$$ for some $g \in \mathbb Z_N^*$, and $ed = 1 \bmod \varphi(N)$. This suggests that \Verify-s in the RSA signature and the Gennaro-Halevi-Rabin Signature are both canonical.

And since both \Sign-s are deterministic, existential unforgeability implies strong unforgeability.


\end{document}
