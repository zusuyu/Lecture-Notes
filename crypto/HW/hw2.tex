\documentclass[8pt]{article}
\usepackage[UTF8]{ctex}
\usepackage[a4paper]{geometry}

\usepackage{amsthm,amsmath,amssymb}
\usepackage{graphicx}
\usepackage{subfigure}
\usepackage{amsmath}
\usepackage{tabularx}
\usepackage{color}
\usepackage{hyperref}
\usepackage{ulem}
\usepackage{multirow}
\usepackage[cache=false]{minted}
\hypersetup{
	colorlinks=true,
	linkcolor=blue
}

\usepackage{appendix}
\geometry{a4paper,centering,scale=0.8}
\geometry{left=2.0cm, right=2.0cm, top=2.5cm, bottom=2.5cm}
\usepackage[format=hang,font=small,textfont=it]{caption}
\usepackage[nottoc]{tocbibind}

\usepackage{algorithm}
\usepackage{algorithmicx}
\usepackage{algpseudocode}
\usepackage{amssymb}
\usepackage{extarrows}
\usepackage{qcircuit}
\usepackage{fancyhdr}
\usepackage{cleveref}
\usepackage{totpages}
\usepackage{pgf}
\usepackage{tikz}
\usepackage{bbm}
\usetikzlibrary{arrows,automata}
\usetikzlibrary{arrows.meta}%画箭头用的包

\makeatletter
\def\@maketitle{%
	\newpage
	\begin{center}%
		\let \footnote \thanks
		{\LARGE \@title \par}%
		\vskip 1.5em%
		{\large
			\lineskip .5em%
			\begin{tabular}[t]{c}%
				\@author
			\end{tabular}\par}%
		\vskip 1em%
		{\large \@date}%
	\end{center}%
	\par
	\vskip 1.5em}
\makeatother

\newtheoremstyle{compact}%
{3pt}{3pt}%
{}{}%
{\bfseries}{\textcolor{red}{.}}%  % Note that final punctuation is omitted.
{.5em}{\mbox{\textcolor{red}{\thmname{#1}\thmnumber{ #2}}\thmnote{ (\textcolor{blue}{#3})}}}
\theoremstyle{compact}
\newtheorem{innercustomgeneric}{\customgenericname}
\providecommand{\customgenericname}{}
\newcommand{\newcustomtheorem}[2]{%
	\newenvironment{#1}[1]
	{%
		\renewcommand\customgenericname{#2}%
		\renewcommand\theinnercustomgeneric{##1}%
		\innercustomgeneric
	}
	{\endinnercustomgeneric}
}

\DeclareMathOperator{\card}{card}

\newtheorem{theorem}{定理}
\newtheorem{lemma}{引理}
\newtheorem{definition}{定义}
\newtheorem{proposition}{命题}
\newtheorem{corollary}{推论}
\newtheorem{remark}{注}
\newtheorem{Proof}{证明}

\def\obj#1{\textbf{\uline{#1}}}
\def\num#1{\textnormal{\textbf{\mbox{\textcolor{blue}{(#1)}}}}}
\def\le{\leqslant}
\def\ge{\geqslant}
\def\im{\text{im }}
\def\Pr#1{\text{Pr}\left[{#1}\right]}
\def\E#1{\mathbb{E}\left[{#1}\right]}
\def\Var#1{\text{Var}\left[{#1}\right]}
\def\Enc{\textsf{Enc}}
\def\Dec{\textsf{Dec}}
\def\Gen{\textsf{Gen}}
\def\rep#1{\llcorner{#1}\lrcorner}

\title{\heiti\zihao{1} Fundamentals  of Cryptography \ Homework 2}
\author{\kaishu\zihao{-3} 周书予\\2000013060@stu.pku.edu.cn}

\CTEXoptions[today=old]
\date{\today}

\begin{document}\large
\fancypagestyle{plain}{
	\fancyhf{}
	\lhead{Fundamentals  of Cryptography}
	\chead{2022 Fall}
	\rhead{Homework 2}
	\cfoot{Page \thepage\ of \pageref{TotPages}}
}
\pagestyle{plain}



\crefname{theorem}{定理}{定理}
\crefname{lemma}{引理}{引理}
\crefname{figure}{图}{图}
\crefname{table}{表}{表}	
\maketitle

\section*{Problem 1}

We construct PRG $G'$ like this: on input $x$ of length $n$,
\begin{enumerate}
	\item $G'$ first finds the largest $m$ such that $m \le n$ and $m \in \mathcal I$. If no such $m$ exists, $G'$ just simply outputs $0^{n+1}$ (an arbitrary string of length $n + 1$).
	\item Then it truncates $x$ to $m$ bits and stretches the input seed into length $n + 1$ by repetitively replacing the last $m$ bits (say $x'$) with $G(x')$.
\end{enumerate}

According to the second property of polynomial-time-enumerable set, $n = \text{poly}(m)$, which means any negligible function of $m$ is also negligible of $n$. Thus the stretching procedure preserves pseudorandomness, which can be formally described as, for all PPT distinguisher $D$, there is a negligible function $\varepsilon(m) = \text{negl}(m)$, such that $$\left|\text{Pr}_{s \gets \{0, 1\}^m}[D(\textsf{STRETCH}(s)) = 1] - \text{Pr}_{r \gets \{0, 1\}^{n+1}}[D(r) = 1]\right| \le \varepsilon(m)$$
Notice that $$\text{Pr}_{s \gets \{0, 1\}^m}[D(\textsf{STRETCH}(s)) = 1] = \text{Pr}_{s \gets \{0, 1\}^n}[D(G'(s)) = 1]$$
thus, $$\left|\text{Pr}_{s \gets \{0, 1\}^n}[D(G'(s)) = 1] - \text{Pr}_{r \gets \{0, 1\}^{n+1}}[D(r) = 1]\right| \le \varepsilon(m) = \varepsilon'(n)$$
which finish the prove that $G'$ is a PRG of stretch $\ell(n) = n + 1$.

\section*{Problem 2}

Assume $|G(s)| = |G_0(s)| = |G_1(s)| = \ell(|s|)$.

\subsection*{Part A}
$G'$ is a PRG of stretch $\ell(n/2)$.

For any fixed PPT distinguisher $D$, define $\varepsilon: \mathbb N \to \mathbb R$ as $$\varepsilon(n) = \left| \text{Pr}_{s \gets \{0, 1\}^n}[D(G'(s)) = 1] - \text{Pr}_{r \gets \{0, 1\}^{\ell(n/2)}}[D(r) = 1]\right|$$

By the definition of $G'$, we know that $$\text{Pr}_{s \gets \{0, 1\}^n}[D(G'(s)) = 1] = \text{Pr}_{s \gets \{0, 1\}^{n/2}}[D(G(s)) = 1]$$
thus $$\varepsilon(n) = \left| \text{Pr}_{s \gets \{0, 1\}^{n/2}}[D(G(s)) = 1] - \text{Pr}_{r \gets \{0, 1\}^{\ell(n/2)}}[D(r) = 1]\right| \le \varepsilon'(n/2)$$
where $\varepsilon'$ is a negligible function since $G$ is assumed to be a secure PRG. $\varepsilon(n) \le \varepsilon'(n/2)$ means $\varepsilon$ is also negligible, thus $G'$ is a PRG.

\subsection*{Part B}
$G'$ may not be a PRG.

Let $H$ be a PRG of stretch $\ell(n)$, and let $G(s) = s_1 \| H(s_2, \cdots, s_n)$. Obviously $G$ is a PRG whose first bit of input and output are always the same.

When $G'$ is constructed based on $G$, $G'$ always outputs $0$ as its first bit, which can be apparently distinguished from the uniform distribution over $\{0, 1\}^{\ell(n + 1)}$.

\subsection*{Part C}
$G'$ may not be a PRG.

Let $G$ be a PRG who ignores its last bit of input, and consider a PPT distinguisher $D$ who examines whether the two halves of the output of $G'$ are the same.

With probability $1/2$, the last bit of input $s$ is $0$, which indicates the output of $G(s)$ and $G(s + 1)$ are totally the same. Thus, $$\text{Pr}_{s \gets \{0, 1\}^n}[D(G'(s)) = 1] \ge 1/2$$meanwhile, $$\text{Pr}_{r \gets \{0, 1\}^{2\ell(n)}}[D(r) = 1] = 2^{-\ell(n)}$$ $G'$ is not a PRG in this case.

\subsection*{Part D}
$G'$ is a PRG of stretch $\ell(n/2)$.

We will use a \textit{hybird argument} to show this conclusion.

For any PPT distinguisher $D$ of $G'$, consider two distinguishers $D_0$ and $D_1$, corresponding to $G_0$ and $G_1$, respectively:
\begin{itemize}
	\item $D_0$ takes input $r_0$ of length $\ell(n/2)$, randomly samples $r_1 \gets \{0, 1\}^{\ell(n/2)}$, and then outputs $D(r_0 \oplus r_1)$.
	\item $D_1$ takes input $r_1$ of length $\ell(n/2)$, randomly samples $s_0 \gets \{0, 1\}^{n/2}$, and then outputs $D(G_0(s_0) \oplus r_1)$.
\end{itemize}

Since $G_0$ and $G_1$ are both PRGs, there is two negligible functions $\varepsilon_0(n), \varepsilon_1(n)$ such that \begin{align*}
	\begin{split}
		\left| \text{Pr}_{s_0 \gets \{0, 1\}^{n/2}}[D_0(G_0(s_0)) = 1] - \text{Pr}_{r_0 \gets \{0, 1\}^{\ell(n/2)}}[D_0(r_0) = 1] \right| \le \varepsilon_0(n) \\
		\left| \text{Pr}_{s_1 \gets \{0, 1\}^{n/2}}[D_1(G_1(s_1)) = 1] - \text{Pr}_{r_1 \gets \{0, 1\}^{\ell(n/2)}}[D_1(r_1) = 1] \right| \le \varepsilon_1(n) \\
	\end{split}
\end{align*}
and notice that \begin{align*}
	\begin{split}
		\text{Pr}_{s_1 \gets \{0, 1\}^{n/2}}[D_1(G_1(s_1)) = 1] &= \text{Pr}_{\substack{s_0 \gets \{0, 1\}^{n/2} \\ s_1 \gets \{0, 1\}^{n/2}}}[D(G_0(s_0) \oplus G_1(s_1)) = 1] \\
		\text{Pr}_{s_0 \gets \{0, 1\}^{n/2}}[D_0(G_0(s_0)) = 1] &= \text{Pr}_{\substack{s_0 \gets \{0, 1\}^{n/2} \\ r_1 \gets \{0, 1\}^{\ell(n/2)}}}[D(G_0(s_0) \oplus r_1) = 1] \\ 
		\text{Pr}_{r_1 \gets \{0, 1\}^{\ell(n/2)}}[D_1(r_1) = 1] &= \text{Pr}_{\substack{s_0 \gets \{0, 1\}^{n/2} \\ r_1 \gets \{0, 1\}^{\ell(n/2)}}}[D(G_0(s_0) \oplus r_1) = 1] \\ 
		\text{Pr}_{r_0 \gets \{0, 1\}^{\ell(n/2)}}[D_0(r_0) = 1] &= \text{Pr}_{\substack{r_0 \gets \{0, 1\}^{\ell(n/2)} \\ r_1 \gets \{0, 1\}^{\ell(n/2)}}}[D(r_0 \oplus r_1) = 1]\\
	\end{split}
\end{align*}
which indicates that $$\left| \text{Pr}_{\substack{s_0 \gets \{0, 1\}^{n/2} \\ s_1 \gets \{0, 1\}^{n/2}}}[D(G_0(s_0) \oplus G_1(s_1)) = 1] - \text{Pr}_{\substack{r_0 \gets \{0, 1\}^{\ell(n/2)} \\ r_1 \gets \{0, 1\}^{\ell(n/2)}}}[D(r_0 \oplus r_1) = 1] \right| \le \varepsilon_0(n) + \varepsilon_1(n)$$

The choice of distinguisher $D$ is arbitrary, which means $G'(s) = G_0(s_0) \| G_1(s_1)$ is a PRG.

\subsection*{Part E}
$G'$ may not be a PRG.

By using the similar argument in \textbf{Part B}, we let $G_0$ be a PRG who has its first bit of output always equal to its first bit of input, and let $G_1$ be just the opposite, i.e. has its first bit of output always differ from its first bit of input.

We can find that $G'(s) = G_{s_1}(s)$ always outputs 0 as its first bit, which can be easily distinguished.

\subsection*{Part F}
$G'$ is a PRG of stretch $\ell(n-1)+1$.

For any PPT distinguisher $D$ of $G'$, consider two distinguishers $D_0$ and $D_1$, corresponding to $G_0$ and $G_1$, respectively:
\begin{itemize}
	\item $D_0$ takes input $r$ of length $\ell(n-1)$ and outputs $D(0 \| r)$.
	\item $D_1$ takes input $r$ of length $\ell(n-1)$ and outputs $D(1 \| r)$.
\end{itemize}

Since $G_0$ and $G_1$ are both PRGs, there is two negligible functions $\varepsilon_0(n), \varepsilon_1(n)$ such that \begin{align*}
	\begin{split}
		\left| \text{Pr}_{s \gets \{0, 1\}^{n-1}}[D_0(G_0(s)) = 1] - \text{Pr}_{r \gets \{0, 1\}^{\ell(n-1)}}[D_0(r) = 1] \right| \le \varepsilon_0(n) \\
		\left| \text{Pr}_{s \gets \{0, 1\}^{n-1}}[D_1(G_1(s)) = 1] - \text{Pr}_{r \gets \{0, 1\}^{\ell(n-1)}}[D_1(r) = 1] \right| \le \varepsilon_1(n) \\
	\end{split}
\end{align*}
notice that \begin{align*}
	\begin{split}
		\text{Pr}_{s \gets \{0, 1\}^n}[D(G'(s)) = 1] &= \frac12 \left(\text{Pr}_{s \gets \{0, 1\}^{n-1}}[D_0(G_0(s)) = 1] + \text{Pr}_{s \gets \{0, 1\}^{n-1}}[D_1(G_1(s)) = 1]\right) \\
		\text{Pr}_{r \gets \{0, 1\}^{\ell(n-1)+1}}[D(r) = 1] &= \frac12 \left(\text{Pr}_{r \gets \{0, 1\}^{\ell(n-1)}}[D_0(r) = 1] + \text{Pr}_{r \gets \{0, 1\}^{\ell(n-1)}}[D_1(r) = 1]\right)
	\end{split}
\end{align*}
thus $$\left| \text{Pr}_{s \gets \{0, 1\}^n}[D(G'(s)) = 1] - \text{Pr}_{r \gets \{0, 1\}^{\ell(n-1)+1}}[D(r) = 1]\right| \le \frac{\varepsilon_0(n) + \varepsilon_1(n)}{2}$$
which finish the proof that $G'$ is a PRG.

\section*{Problem 5}

\subsection*{Part A}
$f'$ is an OWF.

FSOC we assume PPT adversary $\mathcal A'$ breaks $f'$ as OWF,which means that $$\text{Pr}_{x \gets \{0, 1\}^n}\left[\mathcal A'(f(x) \| f(f(x))) \to x': f(x) \| f(f(x)) = f(x') \| f(f(x'))\right] \ge \frac{1}{\text{poly}(n)}$$ then we can construct adversary $\mathcal A$ which breaks $f$ as OWF: on input $y$, call $\mathcal A'$ with input $y \| f(y)$, and output whatever $\mathcal A'$ outputs. It's clear that $\mathcal A$ runs in poly-time, and $$\text{Pr}_{x \gets \{0, 1\}^n}\left[\mathcal A(f(x)) \to x': f(x) = f(x') \right] \ge \frac{1}{\text{poly}(n)}$$
which indicates that $f$ is not an OWF. A contradiction.
\subsection*{Part B}
$f'$ may not be an OWF.

Assume that there is a length-perserving OWF $h$, we construct $f$ as $f(x_1 \| x_2) = 0^{n}h(x_1)$ where $|x_1| = |x_2| = n$. First we prove briefly that $f$ is also a length-perserving OWF.

FSOC PPT adversary $\mathcal A$ breaks $f$ as OWF, which on input $0^n \| h(x_1)$ outputs $x_1' \| x_2'$ such that $0^n \| h(x_1') = 0^n \| h(x_1)$, with probability at least $\frac{1}{\text{poly(n)}}$. Then an adversary for $h$ can be built, which simply concatenate $0^n$ before its input and then call $\mathcal A$, and outputs the first half of $\mathcal A$'s output.

By the construction of $f'$, $f'(x_1 \| x_2) = x_1 \| (x_2 \oplus h(x_1))$. An adversary $\mathcal A$, with $\mathcal A(y_1 \| y_2) = y_1 \| (y_2 \oplus h(y_1))$, runs in poly-time and breaks $f$ as OWF. 
\subsection*{Part C}
$f'$ may not be an OWF.

Assume $h$ is a length-perserving OWF, we construct $f$ as $$f(x_1 \| x_2) = \begin{cases}
	x_1 \| h(x_2), & x_1 \text{ starts with } 0\\
	h(x_1) \| x_2, & x_1 \text{ starts with } 1
\end{cases}$$

First we prove that $f$ is also an OWF: with adversary $\mathcal A'$ which breaks $f$, one can construct $\mathcal A$, on input $y$ it calls $\mathcal A'$ with input $0^n\|y$ and outputs the second half of the output of $\mathcal A'$. Then $A$ breaks $h$ as OWF, an contradiction.

Depending on the first bit of input, the output of $f(x_1 \| x_2)$ must be in the form of either $x_1 \| h(x_2) \| h(\overline{x_1}) \| \overline{x_2}$ or $h(x_1) \| x_2 \| \overline{x_1} \| h(\overline{x_2})$, so an adversary can be constructed, which on input $y_1 \| y_2 \| y_3 \| y_4$, simply tries $y_1 \| \overline{y_4}$ and $\overline{y_3} \| y_2$ as answers and outputs the correct one (if there is). So $f'$ here is not a OWF.
\subsection*{Part D}
$f'$ is an OWF.

If $f'$ is not an OWF, a.k.a. there exists adversary $\mathcal A'$ such that $$\text{Pr}_{x \gets \{0, 1\}^n} [\mathcal A'(f(G(x))) \to x': f(G(x')) = f(G(x))] \ge \frac{1}{\text{poly}(n)}$$

We construct a PPT distinguisher $\mathcal D$ that breaks $G$ as PRG using $\mathcal A'$: on input $r$, it outputs $\mathbbm 1[f(r) = f(G(\mathcal A'(f(r))))]$. We use the notation $U_n$ to indicate the uniform distribution over $\{0, 1\}^n$ in the following statement.

\begin{itemize}
	\item When $r$ is sampled from $G(U_n)$, by the assumption above $\mathcal D$ outputs $1$ with probability at least $\frac{1}{\text{poly}(n)}$.
	\item When $r$ is sampled from $U_{n+1}$, recall that $f$ is an OWF, so by definition, for the adversary $G \circ \mathcal A'$, we have $$\text{Pr}_{x \gets \{0, 1\}^{n+1}}[G \circ \mathcal A'(f(y)) \to y': f(y') = f(y)] < \varepsilon(n)$$ which means that $\mathcal D$ outputs $1$ with probability less than $\varepsilon(n)$.
\end{itemize}

Thus $\mathcal D$ breaks $G$ as a PRG. An contradiction.
\subsection*{Part E}
$f'$ may not be an OWF.

Let $f$ be the OWF mentioned in \textbf{Part B} ($f(x_1 \| x_2) = 0^n \| h(x_2)$) and $G$ be the PRG in \textbf{Problem 2 Part A} ($G(s) = G'(s_1, \cdots, s_{n/2})$).

Now $f \circ G \equiv f(G(0^n))$ is constant, and is obviously not OWF. 
\subsection*{Part F}
$f'$ is an OWF.

FSOC assume that adversary $A$ breaks $f'$ as OWF, which is, for some polynomial $p(n)$, $$\text{Pr}_{x \gets \{0, 1\}^{n}}[\mathcal A(f(x \| 0^{\log n})) \to x': f(x' \| 0^{\log n}) = f(x \| 0^{\log n})] \ge \frac{1}{p(n)}$$

We assert that $\mathcal A$ is also an adversary which breaks $f$ as OWF: \begin{align*}
	\begin{split}
		&\text{Pr}_{x \gets \{0, 1\}^{n+\log n}}[\mathcal A(f(x)) \to x': f(x') = f(x)] \\\ge\ & \text{Pr}_{x \gets \{0, 1\}^{n+\log n}}[\mathcal A(f(x)) \to x': f(x') = f(x) \wedge \text{ last } \log n \text{ bits of } x \text{ are all } 0\text{s}] \\=\ & \text{Pr}_{x \gets \{0, 1\}^{n}}[\mathcal A(f(x \| 0^{\log n})) \to x': f(x' \| 0^{\log n}) = f(x \| 0^{\log n})] \cdot \frac{1}{2^{\log n}} \\\ge\ &\frac{1}{np(n)}
	\end{split}
\end{align*}

$np(n)$ is also a polynomial, so $\mathcal A$ breaks $f$ as OWF, which causes a contradiction.
\subsection*{Part G}
咋做啊
\section*{Problem 6}
\subsection*{Part A}
\begin{align*}
	\begin{split}
		&\text{Pr}[f(\hat x_1) = f(x_1), \cdots, f(\hat x_m) = f(x_m)] \\= \ & \text{Pr}[f(\hat x_1) = f(x_1), \cdots, f(\hat x_m) = f(x_m) \wedge \text{ some } x_i \text{ is bad}] \\&+ \text{Pr}[f(\hat x_1) = f(x_1), \cdots, f(\hat x_m) = f(x_m) \wedge \text{ all } x_i \text{ are good}] \\ \le\ & \sum_{j=1}^{m}\text{Pr}[f(\hat x_1) = f(x_1), \cdots, f(\hat x_m) = f(x_m) \wedge x_j \text{ is bad}] + \left(\text{Pr}[x \text{ is good}]\right)^m \\ \le\ & \sum_{j=1}^{m}\text{Pr}[f(\hat x_1) = f(x_1), \cdots, f(\hat x_m) = f(x_m) | x_j \text{ is bad}] + \left(\text{Pr}[x \text{ is good}]\right)^m \\ \le\ & \sum_{j=1}^{m}m \cdot \text{Pr}[\mathcal A(f(x_j)) \to x': f(x') = f(x_j) | x_j \text{ is bad}] + \left(\text{Pr}[x \text{ is good}]\right)^m \\ < \ & \frac{m^2}{r(n)} + \left(\text{Pr}[x \text{ is good}]\right)^m
	\end{split}
\end{align*}
\subsection*{Part B}
Let $m(n) = 2nq(n)$ and $r(n) = 2m(n)^2p(n)$.

From \textbf{Part A} we know that $$\frac{1}{p(n)} < \frac{m(n)^2}{r(n)} + \left(\text{Pr}[x \text{ is good}]\right)^m \Rightarrow \left(\text{Pr}[x \text{ is good}]\right)^m > \frac{1}{2p(n)}$$
which implies $\text{Pr}[x \text{ is good}] \ge 1 - \frac{1}{2q(n)}$ since otherwise we have $$\frac{1}{2p(n)} < \left(\text{Pr}[x \text{ is good}]\right)^m < \left(1 - \frac{1}{2q(n)}\right)^{2nq(n)} \approx \exp(-n)$$ RHS of this inequility is negligible while LHS is non-negligible, which shows a contradiction.

So $\text{Pr}[x \text{ is good}] \ge 1 - \frac{1}{2q(n)}$ and $\text{Pr}[x \text{ is bad}] \le \frac{1}{2q(n)}$.

\subsection*{Part C}
\def\AREP{\mathcal{A}_{\text{repeat}}}
We show that $\AREP$ breaks $f$ as weak-OWF.
\begin{align*}
	\begin{split}
		&\text{Pr}[\AREP(f(x)) \to x': f(x') \neq f(x)] \\
		= \ & \text{Pr}[\AREP(f(x)) \to x': f(x') \neq f(x) \wedge x \text{ is good}] \\ &+ \text{Pr}[\AREP(f(x)) \to x': f(x') \neq f(x) \wedge x \text{ is bad}] \\
		\le \ & \text{Pr}[\AREP(f(x)) \to x': f(x') \neq f(x) | x \text{ is good}] + \text{Pr}[x \text{ is bad}] \\
		\le \ & \left(1 - \frac{1}{r(n)}\right)^{n \cdot r(n)} + \frac{1}{2q(n)} \\
		\approx\ & \exp(-n) + \frac{1}{2q(n)} \\
		< \ & \frac{1}{q(n)}
	\end{split}
\end{align*}
which violates the definition ($q(n)$-weakness) of $f$.

So we have already finished the proof that $f'$ is an OWF.

\end{document}
