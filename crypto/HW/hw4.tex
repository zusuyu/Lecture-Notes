\documentclass[8pt]{article}
\usepackage[UTF8]{ctex}
\usepackage[a4paper]{geometry}

\usepackage{amsthm,amsmath,amssymb}
\usepackage{graphicx}
\usepackage{subfigure}
\usepackage{amsmath}
\usepackage{tabularx}
\usepackage{color}
\usepackage{hyperref}
\usepackage{ulem}
\usepackage{multirow}
\usepackage[cache=false]{minted}
\hypersetup{
	colorlinks=true,
	linkcolor=blue
}

\usepackage{appendix}
\geometry{a4paper,centering,scale=0.8}
\geometry{left=2.0cm, right=2.0cm, top=2.5cm, bottom=2.5cm}
\usepackage[format=hang,font=small,textfont=it]{caption}
\usepackage[nottoc]{tocbibind}

\usepackage{algorithm}
\usepackage{algorithmicx}
\usepackage{algpseudocode}
\usepackage{amssymb}
\usepackage{extarrows}
\usepackage{qcircuit}
\usepackage{fancyhdr}
\usepackage{cleveref}
\usepackage{totpages}
\usepackage{pgf}
\usepackage{tikz}
\usepackage{bbm}
\usetikzlibrary{arrows,automata}
\usetikzlibrary{arrows.meta}%画箭头用的包

\makeatletter
\def\@maketitle{%
	\newpage
	\begin{center}%
		\let \footnote \thanks
		{\LARGE \@title \par}%
		\vskip 1.5em%
		{\large
			\lineskip .5em%
			\begin{tabular}[t]{c}%
				\@author
			\end{tabular}\par}%
		\vskip 1em%
		{\large \@date}%
	\end{center}%
	\par
	\vskip 1.5em}
\makeatother

\newtheoremstyle{compact}%
{3pt}{3pt}%
{}{}%
{\bfseries}{\textcolor{red}{.}}%  % Note that final punctuation is omitted.
{.5em}{\mbox{\textcolor{red}{\thmname{#1}\thmnumber{ #2}}\thmnote{ (\textcolor{blue}{#3})}}}
\theoremstyle{compact}
\newtheorem{innercustomgeneric}{\customgenericname}
\providecommand{\customgenericname}{}
\newcommand{\newcustomtheorem}[2]{%
	\newenvironment{#1}[1]
	{%
		\renewcommand\customgenericname{#2}%
		\renewcommand\theinnercustomgeneric{##1}%
		\innercustomgeneric
	}
	{\endinnercustomgeneric}
}

\DeclareMathOperator{\card}{card}

\newtheorem{theorem}{定理}
\newtheorem{lemma}{引理}
\newtheorem{definition}{定义}
\newtheorem{proposition}{命题}
\newtheorem{corollary}{推论}
\newtheorem{remark}{注}
\newtheorem{Proof}{证明}

\def\obj#1{\textbf{\uline{#1}}}
\def\num#1{\textnormal{\textbf{\mbox{\textcolor{blue}{(#1)}}}}}
\def\le{\leqslant}
\def\ge{\geqslant}
\def\im{\text{im }}
\def\Pr#1{\text{Pr}\left[{#1}\right]}
\def\E#1{\mathbb{E}\left[{#1}\right]}
\def\Var#1{\text{Var}\left[{#1}\right]}
\def\Enc{\textsf{Enc}}
\def\Dec{\textsf{Dec}}
\def\Gen{\textsf{Gen}}
\def\rep#1{\llcorner{#1}\lrcorner}

\title{\heiti\zihao{1} Fundamentals  of Cryptography \ Homework 4}
\author{\kaishu\zihao{-3} 周书予\\2000013060@stu.pku.edu.cn}

\CTEXoptions[today=old]
\date{\today}

\begin{document}\large
\fancypagestyle{plain}{
	\fancyhf{}
	\lhead{Fundamentals  of Cryptography}
	\chead{2022 Fall}
	\rhead{Homework 4}
	\cfoot{Page \thepage\ of \pageref{TotPages}}
}
\pagestyle{plain}



\crefname{theorem}{定理}{定理}
\crefname{lemma}{引理}{引理}
\crefname{figure}{图}{图}
\crefname{table}{表}{表}	
\maketitle

\section*{Problem 1}
\subsection*{Part A}
Obviously $E$ is poly-time computable.

An decoding algorithm counts the number of leading zeros and learns the length of $x$, and finally outputs the last $n$ bits as $x$.

For any $x, x' \in \mathcal X^*$, \begin{itemize}
	\item if $|x| = |x'|$, because $E(x) \neq E(x')$ and $|E(x)| = |E(x')|$, they are not prefix of each other.
	\item if $|x| < |x'|$ then $|E(x)| < |E(x')|$, notice that the $|x| + 1$ bit in $E(x)$ is $1$ while the counterpart is $0$ in $E(x')$, thus $E(x)$ is not a prefix of $E(x')$.
\end{itemize}

\subsection*{Part B}
Write $|x|$ as a binary string of length $\log(|x|)$, then insert a $1$ between each adjacent bits, and concatenate with $0 \| x$. That is, $$E(x) = \textsf{insert-one}(x) \| 0 \| x$$ for example, we have $E(x) = 11010 \| 0\|x$ for $|x| = 4$ and $E(x) = 1101111 \| 0 \| x$ for $|x| = 11$.

Obviously $E$ is poly-time computable.

An decoding algorithm can learn from the \textsf{insert-one} part (which ends by a $0$ on even bits) the length of $x$, and output the last $n$ bits as $x$.

Apparently $E(x) \neq E(x')$ holds for any $x \neq x'$ with the same length. As for $|x| < |x'|$, the \textsf{insert-one} part of encoding makes $E(x)$ and $E(x')$ different in the first $2\log(|x|)$ bits, and thus $E(x)$ is not a prefix of $E(x')$.

\subsection*{Part D}
$$E(x) = \begin{cases}
	\textsf{int2str}(|x|, n) \| 0^{-|x| \bmod n} \| x, & |x| < 2^n - 1 \\
	1^n \| 0^{|x|} \| 1 \| x, & |x| \ge 2^n - 1
\end{cases}$$
where function $\textsf{int2str}(m, n)$ converts integer $m$ into a binary string of length $n$.

It's not hard to see that $E$ is a prefix-free encoding, and it suffices $|E(x)| < |x| + 2n$ and $n \mid |E(x)|$ for every $x \in \mathcal X^{< 2^n-1}$.

\section*{Problem 2}
\subsection*{Part A}
Since $F_{CBC}$ is parameterized with a keyed secure PRF $F$, it is not hard to see (we have proven it multiple times) that $$\left| \text{Pr}\left[\mathcal D^{F_{CBC}}(1^n) = 1\right] - \text{Pr}\left[\mathcal D^{G_{CBC}}(1^n) = 1\right] \right| < \textsf{negl}(n)$$
where $G_{CBC}$ is exactly the same as $F_{CBC}$, except a trult random function $g: \{0, 1\}^n \to \{0, 1\}^n$ is used instead of the PRF $F_k$.

Then we're going to show that $$\left| \text{Pr}\left[\mathcal D^{G_{CBC}}(1^n) = 1\right] - \text{Pr}\left[\mathcal D^{f}(1^n) = 1\right] \right| < \textsf{negl}(n)$$
(where $f$ is a truly random function which maps $(\{0, 1\}^n)^*$ to $\{0, 1\}^n$) For any prefix-free set $X_1, \cdots, X_q \in (\{0, 1\}^{n})^{*}$ queried by the distinguisher and $t_1, \cdots, t_n \in \{0, 1\}^n$ the output of the oracle (which is $G_{CBC}$ or $f$), we want to show that \begin{equation}
	\text{Pr}\left[\forall i, G_{CBC}(X_i) = t_i\right] \ge (1 - \textsf{negl}(n))\text{Pr}\left[\forall i, f(X_i) = t_i\right]
	\label{lalala}
\end{equation}

If so, for any case where $\mathcal D^{f}$ outputs $1$, $\mathcal D^{G_{CBC}}$ outputs $1$ with probability at least $1 - \textsf{negl}(n)$, and the same when outputing $0$. Which means that $\mathcal D^{G_{CBC}}$ outputs the same as $\mathcal D^{f}$ with probability at least $1 - \textsf{negl}(n)$, so they are indistinguishable.

Now we prove \cref{lalala}. For $X_i \in (\{0, 1\}^n)^{\ell}$, we denote $(I_1, I_2, \cdots, I_{\ell})$ as $(x_{i, 1}, G_{CBC}(x_{i, 1}) \oplus x_{i_2}, \cdots, G_{CBC}(x_{i, 1}, \cdots, x_{i, \ell-1}) \oplus x_{i, \ell})$, which is all the input to $G_{CBC}$. If there is an $I_i$ in $X$ coincides with another $I'_j$ in $X'$ with different prefix $(x_1, \cdots, x_i) \neq (x'_1, \cdots, x'_j)$, we say it is a \textbf{collision}.

It can be proved that \begin{itemize}
	\item if there is no collision occurred, then $$\text{Pr}\left[\forall i, G_{CBC}(X_i) = t_i\right] = \text{Pr}\left[\forall i, f(X_i) = t_i\right]$$
	\item collision only occurred with negligible probability
\end{itemize}

These two conclusions are quite intuitive. The first one holds because $g$ is a truly random function, so it outputs identical independent distribution on different inputs. The second one holds because for each pair of $(I_i, I'_j)$, they coincides with negligible probability, which the length and number of $X$ and both polynomial, by union bound we know that even one collision happens with negligible probability.

\subsection*{Part B}

$F_{CBC}$ is a secure prefix-free PRF, which means that $$\left| \text{Pr}\left[\textsf{Mac-forge}_{\mathcal A. \Pi}(n) = 1\right] - \text{Pr}\left[\textsf{Mac-forge}_{\mathcal A. \Pi'}(n) = 1\right] \right| < \textsf{negl}(n)$$
where $\Pi = (\textsf{Gen}, \textsf{Mac}, \textsf{Vrf})$ is the MAC mentioned in the problem, and $\Pi'$ exactly the same as $\Pi$, except a truly random function $f(\cdot)$ is used instead of $F_{CBC}(k, \cdot)$.

we also have
$$\text{Pr}\left[\textsf{Mac-forge}_{\mathcal A. \Pi'}(n) = 1\right] = 2^{-n}$$ which is clear since $E(\cdot)$ outputs different $E(x)$ for different $x$, and any adversary can only make random guessing and has success probability at most $2^{-n}$ in front of the truly random function $f$.

\section*{Problem 3}
We define that all operations are over the $\mathbb F_{2^n}$ field.

\subsection*{Part A}
\begin{itemize}
	\item If $i \neq i'$, $m_{j, i} + ik_1 \neq m_{j', i'} + i'k_1$ is equivalent to $k_1 \neq (m_{j, i} - m_{j', i'}) \cdot (i' - i)^{-1}$, which is of probability $1 - \textsf{negl}(n)$ since $k_1$ is chosen at uniformly random.
	\item If $i = i'$ then $m_{j,i} \neq m_{j',i'}$, thus $m_{j, i} + ik_1 \neq m_{j', i'} + i'k_1$ must holds.
\end{itemize}

\subsection*{Part B}
Consider a distinguisher $\mathcal D$ which compares $\sum\limits_{i=1}^{\ell_j}\mathcal O(m_{j,i} + ik_1)$ with $\sum\limits_{i'=1}^{\ell_{j'}}\mathcal O(m_{j',i'} + i'k_1)$, and outputs $1$ if equal and $0$ otherwise.

For truly random function $f$, it can be seen that $\text{Pr}[\mathcal D^{f(\cdot)}(1^n) = 1] = 1 - \textsf{negl}(n)$ since with probability at least $1 - \textsf{negl}(n)$, the set $\{m_{j, i} + ik_1\}$ is not equal to $\{m_{j', i'} + i'k_1\}$.

Notice that $F(k_2, \cdot)$ is a PRF, so we also have $\text{Pr}[\mathcal D^{F(k_2, \cdot)}(1^n) = 1] = 1 - \textsf{negl}(n)$, which means that with probability $1 - \textsf{negl}(n)$, $$(m_{j,1}, \cdots, m_{j,\ell_j}) \neq (m_{j',1}, \cdots, m_{j',\ell_{j'}}) \Rightarrow \sum\limits_{i=1}^{\ell_j}F(k_2, m_{j,i} + ik_1) \neq \sum\limits_{i'=1}^{\ell_{j'}}F(k_2, m_{j',i'} + i'k_1)$$

\subsection*{Part C}

By replacing $F(k_3, \cdot)$ in $F_{\text{PMAC}}$ with a truly random function $f$, we obtain another function $G_{\text{PMAC}} = g\left(\sum\limits_{i=1}^{\ell}F(k_2, m_i + ik_1)\right)$.

Since $F(k_3, \cdot)$ is a secure PRF, it can be shown that $$\left| \text{Pr}\left[\mathcal D^{F_{\text{PMAX}}}(1^n)=1\right] - \text{Pr}\left[\mathcal D^{G_{\text{PMAC}}}(1^n)=1\right] \right| < \textsf{negl}(n)$$

With probability $1 - \textsf{negl}(n)$, the outputs of $F(k_2, \cdot)$ are all distinct, which means $G_{\text{PMAC}}$ indistinguishable from a truly random function $g$.

Thus, $F_{\text{PMAC}}$ is a secure PRF.

\section*{Problem 4}

\subsection*{Part A}
Regard $H(k, (m_1, \cdots, m_{\ell}))$ as a degree-$\ell$ polynomial of $k$ over $\mathbb F_{2^n}$ field.

For distinct messages $m, m'$ with length $\le \ell n$, $H(k, m) - H(k, m')$ is a nonzero polynomial of degree at most $\ell$, which has at most $\ell$ zero points over $\mathbb F_{2^n}$.

$\ell$ is polynomial on $n$, which draws the conclusion that $$\text{Pr}_{k \gets \{0, 1\}^n} \left[H(k, m) = H(k, m')\right] \le \frac{\ell}{2^n} = \textsf{negl}(n)$$

\subsection*{Part B}
Let $\Pi = (\textsf{Gen}, \textsf{Mac}, \textsf{Vrfy})$ be the MAC mentioned in the problem, and $\Pi' = (\textsf{Gen}', \textsf{Mac}', \textsf{Vrfy}')$ exactly the same as $\Pi$, except that a truly random function $f$ is used instead of the pseudorandom function $F_{k_2}$. Consider that $$\text{Pr}\left[\textsf{Mac-forge}_{\mathcal A. \Pi'}(n) = 1\right] \le p(n)\textsf{negl}(n) + \left(1 - p(n)\textsf{negl}(n)\right)2^{-n}$$ for some polynomial $p(n)$.

This is because when adversary $\mathcal A$ gives $m'$, there is probability at most $p(n)\textsf{negl}(n)$ such that $H(k, m')$ coincides with some $H(k, m^{(i)})$ asked before, and if there is not, the adversary can only make random guessing and has success probability at most $2^{-n}$.

We can further prove that $$\left| \text{Pr}\left[\textsf{Mac-forge}_{\mathcal A. \Pi}(n) = 1\right] - \text{Pr}\left[\textsf{Mac-forge}_{\mathcal A. \Pi'}(n) = 1\right] \right| < \textsf{negl}(n)$$

Consider a distinguisher $\mathcal D$ which emulates the message authentication experiment for $\mathcal A$ and observes whether $\mathcal A$ succeeds in outputting a valid tag on a “new” message. If so, $\mathcal D$ guesses that its oracle is a pseudorandom function; otherwise, it guesses that its oracle is a truly random function.

We have \begin{align*}
	\begin{split}
		\text{Pr}\left[\mathcal D^{F_{k_2}}(1^n)=1\right] &= \text{Pr}\left[\textsf{Mac-forge}_{\mathcal A. \Pi}(n) = 1\right] \\
		\text{Pr}\left[\mathcal D^{f}(1^n)=1\right] &= \text{Pr}\left[\textsf{Mac-forge}_{\mathcal A. \Pi'}(n) = 1\right]
	\end{split}
\end{align*}
and since $F$ is a secure PRF and $\mathcal D$ runs in polynomial time, we also have $$\left| \text{Pr}\left[\mathcal D^{F_{k_2}}(1^n)=1\right] - \text{Pr}\left[\mathcal D^{f}(1^n)=1\right] \right| < \textsf{negl}(n)$$

So finally we draw the conclu that $$ \text{Pr}\left[\textsf{Mac-forge}_{\mathcal A. \Pi}(n) = 1\right] \le \textsf{negl}(n)$$
which means that $\Pi$ is a (strongly) secure MAC.

\subsection*{Part C}

If the item $k^{\ell}$ in $H(k, m)$ is lost, then $H(k, m) = H(k, 0^n \| m)$ holds for all $m$, which easily makes this MAC insecure.

\section*{Problem 5}
\subsection*{Part A}
Suppose $F'$ is a PRF, then $F(k, x) = \begin{cases}
	F'(x, 0), & k = 0^n \\
	F'(k, x), & k \neq 0^n
\end{cases}$ is also a PRF since it performs differently with $F'$ with only negligible probability.

Thus we have $\hat{F}(k, 0^n) = F(k, 0^n) \oplus F(0^n, k) = \begin{cases}
	F'(0^n, 0^n) \oplus F'(0^n, 0^n) & k = 0^n \\
	F'(k, 0^n) \oplus F'(k, 0^n) & k \neq 0^n
\end{cases} = 0^n$ to be a deterministic value, which makes $\hat{F}$ obviously not a PRF.
\subsection*{Part B}
$\hat{F}(x, y)$ is obviously symmetric, so it is sufficed to prove that $\hat{F}(k, \cdot)$ is a PRF.

For any distinguisher which queries encryption for message $m_1, \cdots, m_{p(n)}$ towards its oracle, there is $1 - \textsf{negl}(n)$ probability that all $g_1(m_i)$-s are distinct since we assume $g_1$ to be collision resistant. And also $F(g_0(k), g_1(m_i))$-s are distinct with $1 - \textsf{negl}(n)$ probability, since $F$ itself is a PRF.

Since $g$ is a PRG, we know that $g_0(k), g_1(m_i)$ should be nearly independent with $g_1(k), g_0(m_i)$, which makes $F(g_0(k), g_1(m_i))$ nearly independent with $F(g_0(m_i), g_1(k))$. In summary, with probability $1 - \textsf{negl}(n)$, all $\hat{F}(k, m_i)$-s are distinct, which gives no information to the distinguisher, and thus it can not distinguish $\hat{F}(k, \cdot)$ from a truly random function.

\section*{Problem 6}
\subsection*{Part A}
We prove that $\textsf{Enc}_1$ is DKMA-secure when $F$ is modeled as an ideal cipher.

For any PPT adversary $\mathcal A$, a PPT distinguisher $\mathcal D$ can be built, which simulates the interaction between $\mathcal A$ and the challenger, except that when $c_i = \textsf{Enc}(k, x_i)$ is returned, it uses an oracle $\mathcal O$ and returns $c_i = (r, \mathcal O(r \oplus m))$. Finally, $\mathcal D$ outputs $1$ if $\mathcal A$ wins, and $0$ otherwise.

On the one hand, we have the equlity that $$\text{Pr}\left[\textsf{DKMA}_{\mathcal A, \textsf{Enc}_1}(n) = 1\right] = \text{Pr}\left[\mathcal D^{F(k, \cdot)}(1^n) = 1\right]$$ on the other hand, we denote $\textsf{Enc}_1'$ exactly the same as $\textsf{Enc}_1$, except a truly random permutation $f(\cdot)$ is in place of $F(k, \cdot)$. When adversary plays DKMA security game under this encryption scheme, the distribution of $\textsf{Enc}'_1(k, f_{i, 0}(k))$ and $\textsf{Enc}'_1(k, f_{i, 1}(k))$ are totally the same, which makes them indistinguishable, so we have $$\text{Pr}\left[\textsf{DKMA}_{\mathcal A, \textsf{Enc}'_1}(n) = 1\right] = \text{Pr}\left[\mathcal D^{f(\cdot)}(1^n) = 1\right] = \frac12$$

Since $F(k, \cdot)$ is a ideal cipher, $\mathcal A$ can have at most negligible advantage to win this game.
\subsection*{Part B}
For fixed $k_1, k_2$, there is a bijection between permutation $P(m)$ and $Q(m) := F((k_1, k_2), m) = P(m \oplus k_1) \oplus k_2$, which means that the distribution of $P(\cdot)$ and $F((k_1, k_2), \cdot)$ are totally the same, making them indistinguishable.

So $P$ is random permutation implies that $F$ is a strong PRP.

\subsection*{Part C}

We prove that $\textsf{Enc}_1$ is not DKMA-secure as $F$ defined in \textbf{Part B}, which is, $\textsf{Enc}_1(k, x) = (r, P(r \oplus x \oplus k_1) \oplus k_2)$.

First adversary chooses function $f_{1, 0}, f_{1, 1}$ such that for all $k = k_1 \| k_2$, $f_{1, 0}(k) = f_{1, 1}(k) = k_1$, which makes $c_1 = (r, P(r \oplus k_1 \oplus k_1) \oplus k_2) := (c_{1, 0}, c_{1, 1})$, and adversary can learn $k_2$ by calculating $k_2 = c_{1, 1} \oplus P(c_{1, 0})$.

Secondly it chooses $f_{2, 0}(k) = f_{2, 1}(k) = 0^n$, which makes $c_2 = (r, P(r \oplus k_1) \oplus k_2) := (c_{2, 0}, c_{2, 1})$, and can learn $k_1$ by calculating $k_1 = P^{-1}(c_{2, 1} \oplus k_2) \oplus c_{2, 0}$.

By now adversary has already learned all information about the key, thus it can easily wins the security game, which means that $\textsf{Enc}_1$ here is not DKMA-secure.



\end{document}
