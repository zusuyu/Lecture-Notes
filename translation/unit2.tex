\documentclass[8pt]{article}
\usepackage[UTF8]{ctex}
\setCJKmainfont[ItalicFont=Noto Serif CJK SC Bold, BoldFont=Noto Serif CJK SC Black]{Noto Serif CJK SC}

\usepackage[a4paper]{geometry}

\usepackage{graphicx}
\usepackage{subfigure}
\usepackage{amsmath}
\usepackage{tabularx}
\usepackage{color}
\usepackage{hyperref}
\usepackage{ulem}
\usepackage{multirow}
\usepackage[cache=false]{minted}
\hypersetup{
	colorlinks=true,
	linkcolor=blue
}

\usepackage{appendix}
\geometry{a4paper,centering,scale=0.8}
\geometry{left=2.0cm, right=2.0cm, top=2.5cm, bottom=2.5cm}
\usepackage[format=hang,font=small,textfont=it]{caption}
\usepackage[nottoc]{tocbibind}

\usepackage{algorithm}
\usepackage{algorithmicx}
\usepackage{algpseudocode}
\usepackage{amssymb}
\usepackage{qcircuit}
\usepackage{fancyhdr}
\usepackage{cleveref}


\usepackage{tikz}  
\usetikzlibrary{arrows.meta}%画箭头用的包

\makeatletter
\def\@maketitle{%
	\newpage
	\begin{center}%
		\let \footnote \thanks
		{\LARGE \@title \par}%
		\vskip 1.5em%
		{\large
			\lineskip .5em%
			\begin{tabular}[t]{c}%
				\@author
			\end{tabular}\par}%
		\vskip 1em%
		{\large \@date}%
	\end{center}%
	\par
	\vskip 1.5em}
\makeatother

\newtheorem{theorem}{定理}
\newtheorem{lemma}{引理}
\newtheorem{definition}{定义}
\newtheorem{proposition}{命题}
\newtheorem{corollary}{推论}


\title{\heiti\zihao{1} Unit 2 Homework}
\author{\kaishu\zihao{-3} 周书予\\2000013060@stu.pku.edu.cn}

\CTEXoptions[today=old]
\date{\today}

\begin{document}\large
\pagestyle{fancy}
\lhead{汉英翻译理论与实践}
\chead{2022 Spring}
\rhead{E-C Translation: Theories and Practice}


\crefname{theorem}{定理}{定理}
\crefname{lemma}{引理}{引理}
\crefname{figure}{图}{图}
\crefname{table}{表}{表}	
\maketitle

\section{E2C Translation}

	我们建立了一座纪念碑来悼念一个物种的灭绝。它象征着我们的悲痛。再也没有人能在三月的天空中看见疾行的鸟群,它们在威斯康星州的每一片森林和草原上赶走战败的冬,并用胜利的姿态迎接春的到来。
	
	年轻时记得鸽子的人们依然在世,正如年轻时被风拂过的树也一样。但在十年后,只有最古老的橡树还会记得。最终,只有山峦会记得。
	
	书籍和博物馆里永远都还有鸽子,但都是些雕塑和影像,麻木于一切的苦痛和快乐。书本上的鸽子不能从云层中窜出,惊动鹿跑着躲避,也不能在满是桅杆的树林中拍打翅膀发出雷鸣般的声响。书本上的鸽子不能在明尼苏达州晨食新麦,在加拿大夕品蓝莓。它们感受不到季节的变动,感受不到太阳的亲吻和风雨的拍打。它们永远活着,也从没有活过。

	我们的祖先在住房和吃穿上都远不及我们。他们为了改善自身命运的不懈奋斗,成为了我们失去鸽子的原因。或许我们现在悲伤的原因,只是我们心里不太确定我们是否真的从这种交换中得到了好处。工业设施相比鸽子给我们带来了更多的舒适便捷,但两者对于春的贡献就难下定论了。

	自达尔文向我们揭示物种起源以来已经过去了一个世纪。我们掌握了前人无法企及的真相:在漫长的进化征程中,人类只是其他物种的同行者。我们应当从中学到些什么:与其他物种的亲近感,与之和谐共处的愿想,以及对生命在时空上的广阔维度的感慨。

	总的来说,在达尔文的时代之后,我们应该理解到,人类,尽管是生物进化这艘冒险船的船长,但却不是唯一的主体。此外,达尔文提出这种假说的初衷只是为了“在黑暗中吹口哨”这种简单的理由——他为人类带来了一次思想启蒙。

	如今的这种事情已然降临到了我们身边。我担心这种事情对很多人来说还尚未发生。

	对于一个物种来说,哀悼另一个物种的灭绝是阳光下的一件新鲜事。杀死最后一头猛犸象的克罗马尼翁人一心想着肉排,射杀最后一只鸽子的运动员汲汲于能力提升,而打死最后一只海鸥的水手,根本就没有什么心理活动。但是,我们这些失去鸽子的人,却为鸽子的离去感到悲伤。如果这是我们的葬礼,鸽子们几乎不会为我们哀悼。基于这个事实——而不是杜邦先生的尼龙袜或者万尼瓦尔·布什先生的炸弹——客观地证明了我们之于野兽的优越性。

	这座纪念碑如鸭鹰般地栖息在这片悬崖上,将在数不清的年岁里静静地注视这片宽阔的峡谷。三月里,它将盼着大雁飞过,向河流描述苔原上更清晰、寒冷、孤寂的水。它将见证四月里紫荆花开花落,和五月里漫山橡树花的绚烂。觅食的木鸭会在椴树林中寻找空心的树枝,金色的原野鸟会在河畔的柳树上摇下金色的花粉。白鹭会在八月的沼泽地上尽展风姿,鸻鹬会在九月的天空中高声呼啸,山核桃会在十月的落叶间坠落,而冰雹会在十一月的树林里起舞。但鸽子不会经过,因为不再有鸽子了,除了这只不会飞的,用青铜刻在岩石上的鸽子。游客们会读到这段铭文,但他们的思想无法展翅——就像鸽子那样。

	经济道德学家告诉我们,对鸽子的哀悼不过是一种怀旧;如果养鸽子的人不把它们赶走,农民最终也会出于自卫而不得不这样做。
	
	这是一条奇特的真理,言之有理,却不是出于所声称的理由。

	鸽子是一场生物风暴,是在土壤的肥力与空气的氧气这两种势不两立的强大能量之间迸发出的闪电。这股羽毛风暴的咆哮每年都将遍及整片大陆,它们吸食森林和草原上的果实,用在生命的旅途中充当燃料。就如同其他连锁反应一样,鸽子无法在自身狂暴强度降低的情况下幸存下来。当连续供给的燃料减少时,它们的火焰熄灭了,几乎没有溅射,甚至没有一丝烟雾。

	今天,橡树仍在向天空炫耀它们的负担,但羽毛状的闪电已不复存在。虫子和象鼻虫必须缓慢而无声地执行那个曾经从苍穹引来雷鸣的生物任务。

	令人惊奇的不是鸽子的消失,而是它们曾经在前巴比伦时代后的数千年中生存下来。

	鸽子热爱它们的土地:它们靠着对丰硕的葡萄和爆裂的蜂巢的强烈欲望,以及对距离与季节的蔑视而生存。凡是在今天威斯康星州没有无偿提供给它们的东西,它们都会在明天去密歇根州、拉布拉多州或田纳西州搜寻并得到。它们热爱的是眼前的事物,而这些事物就存在于某个地方;要想找到它们,只需要自由的天空,以及展翅高飞的意志。

	沉溺于过往也是阳光下的新鲜事,这对于大多数人和所有的鸽子来说都是未曾设想的。把美国看成历史,把命运设想成一种必然,在静止的岁月中嗅到山核桃树的味道——所有这些事情对我们来说都是可能的,而要实现它们,只需要自由的天空,以及展翅高飞的意志。综上——而不是杜邦先生的尼龙袜或者万尼瓦尔·布什先生的炸弹——客观地证明了我们之于野兽的优越性。

\section{C2E Translation}

	Chagan Lake, which means white and holy lake, is located in the notrhwest part of Jilin Province, lies in the water network area of the confluence of the Nen River and the Huolin River. It is one of the ten largest freshwater lake in China, and also the largest inland lake in Jilin Province. With beautiful environment, fascinating scenery and abundant resources, Chagan Lake is rich in carp, silver carp, bighead carp and so on, making it a national nature reserve.

	In Summer, Chagan Lake has a refreshing sight with its blue waves and the blurred boundary between lake and sky. The summer of Chagan Lake is famous for its fresh grassland, pure lake water, cool mountain breeze and rich ethnic customs. In Winter, Chagan Lake is wrapped in sliver and has a unique style.


\end{document}
