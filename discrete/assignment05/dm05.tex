\documentclass[UTF-8]{ctexart}

\usepackage{graphicx}
\usepackage{subfigure}
\usepackage{amsmath}
\usepackage{tabularx}
\usepackage{color}
\usepackage{hyperref}
\usepackage{ulem}
\usepackage{multirow}
\usepackage[cache=false]{minted}
\hypersetup{
	colorlinks=true,
	linkcolor=black
}

\usepackage{geometry}
\geometry{a4paper,centering,scale=0.8}
\geometry{left=2.0cm, right=2.0cm, top=2.5cm, bottom=2.5cm}
\usepackage[format=hang,font=small,textfont=it]{caption}
\usepackage[nottoc]{tocbibind}

\usepackage{algorithm}
\usepackage{algorithmicx}
\usepackage{algpseudocode}
\usepackage{amssymb}


\usepackage{tikz}  
\usetikzlibrary{arrows.meta}%画箭头用的包

\makeatletter
\def\@maketitle{%
	\newpage
	\begin{center}%
		\let \footnote \thanks
		{\LARGE \@title \par}%
		\vskip 1.5em%
		{\large
			\lineskip .5em%
			\begin{tabular}[t]{c}%
				\@author
			\end{tabular}\par}%
		\vskip 1em%
		{\large \@date}%
	\end{center}%
	\par
	\vskip 1.5em}
\makeatother


\title{\heiti\zihao{1} 群习题}
\author{\kaishu\zihao{-3} 周书予\\2000013060@stu.pku.edu.cn}

\date{\today}

\begin{document}
\maketitle

\section{}
\subsection{description}
设$G$是有限半群, 证明: 若消去律存在, 即任意$ax = ay$或者$xa = ya$都可以推出$x = y$, 则$G$是群.
\subsection{solution}
只需要证明$G$中存在幺元以及任意元素都存在逆元.

任取$a \in G$, 构造映射$f: G \to G$满足$f(x) = ax$, 由消去律可知该映射为单射, 由$G$有限可知该映射为双射, 注意到$a$的任意性, 这说明了对于$\forall a, b \in G$, 都存在唯一的$x \in G$使得$ax = b$. 同理, 构造映射$g: G \to G$满足$g(y) = ya$, 可以说明对于$\forall a, b \in G$, 都存在唯一的$y \in G$使得$ya = b$.

对于$a \in G$, 分别记满足$ya = a$和$ax = a$的$x, y$为$e_a, e_a'$即$a$的左幺元与右幺元, 两边分别相乘得到$a(e_aa) = (ae_a')a$, 根据结合律与消去律可知$e_a = e_a'$, 即$a$的左右幺元相等, 以下简称为$a$的幺元. 即$e_b$为$b$的幺元, 那么有$(ae_a)b = a(e_bb)$, 可以得到$e_a = e_b$, 即所有元素的幺元均相等, 因此$G$中存在幺元, 记为$e$.

由于$ax = e, ya = e$有唯一解, 这说明$G$中元素均存在左右逆元. 类似上面证明左右幺元相等的方法, 也可以证明左右逆元相等. 故$G$是一个群.

\section{}
\subsection{description}
在$S_5$中, 设
$$\sigma = \begin{pmatrix}
1 & 2 & 3 & 4 & 5\\
5 & 1 & 4 & 3 & 2
\end{pmatrix}, 
\tau = \begin{pmatrix}
1 & 2 & 3 & 4 & 5\\
4 & 3 & 1 & 5 & 2
\end{pmatrix}$$

试计算:
\begin{enumerate}
	\item $\sigma, \tau$分别的不相交轮换分解和一个对换分解;
	\item $\sigma^{-1}\tau^{-1}$, 写成上面描述的形式.
\end{enumerate}
\subsection{solution}
\begin{enumerate}
	\item \begin{align*}
	\sigma = (1\ 5\ 2)(3\ 4) = (1\ 5)(5\ 2)(3\ 4)\\
	\tau = (1\ 4\ 5\ 2\ 3) = (1\ 4)(4\ 5)(5\ 2)(2\ 3)
	\end{align*}
	\item 不难得到$\tau\sigma = \begin{pmatrix}
	1 & 2 & 3 & 4 & 5\\
	2 & 4 & 5 & 1 & 3
	\end{pmatrix}$, 故$\sigma^{-1}\tau^{-1} = (\tau\sigma)^{-1} = \begin{pmatrix}
	1 & 2 & 3 & 4 & 5\\
	4 & 1 & 5 & 2 & 3
	\end{pmatrix}$, 可以写出分解
	$$\sigma^{-1}\tau^{-1} = (1\ 4\ 2)(3\ 5) = (1\ 4)(4\ 2)(3\ 5)$$
\end{enumerate}
\section{}
\subsection{description}
设$H_1, H_2$是群$G$的子群, 证明$H_1H_2$是$G$的子群当且仅当$H_1H_2 = H_2H_1$.
\subsection{solution}
$\Rightarrow$: 需要证明$H_1H_2 \subseteq H_2H_1$以及$H_2H_1 \subseteq H_1H_2$.

任取$a \in H_1H_2$, 由于$H_1H_2$是子群, 故$a^{-1} \in H_1H_2$, 设$a^{-1} = a_1a_2$其中$a_1 \in H_1, a_2 \in H_2$, 则$a = (a_1a_2)^{-1} = a_2^{-1}a_1^{-1} \in H_2H_1$, 于是$H_1H_2 \subseteq H_2H_1$.

任取$b \in H_2H_1$, 设$b = b_2b_1$其中$b_1 \in H_1, b_2 \in H_2$, 于是$b^{-1} = b_1^{-1}b_2^{-1} \in H_1H_2$, 由于是子群, 故$b = (b^{-1})^{-1} \in H_1H_2$, 于是$H_2H_1 \subseteq H_1H_2$.

因此$H_1H_2 = H_2H_1$, 必要性得证.

$\Leftarrow$: 只需要证明$\forall a, b \in H_1H_2$都有$ab^{-1} \in H_1H_2$.

记$a = a_1a_2, b = b_1b_2$其中$a_1, b_1 \in H_1, a_2, b_2 \in H_2$, 此时$a_2b_2^{-1} \in H_2$故$a_2b_2^{-1}b_1^{-1} \in H_2H_1 = H_1H_2$, 于是$a_1a_2b_2^{-1}b_1^{-1} \in H_1H_2$即$ab^{-1} \in H_1H_2$。充分性得证.
\section{}
\subsection{description}
证明正实数乘法群$(\mathbb R_{>0}, \times)$是非零实数乘法群$(\mathbb R^*, \times)$的子群, 并求出这个子群的指数.
\subsection{solution}
不难验证正实数乘法群$H$是非零实数乘法群$G$的一个子群, 而$\forall a \in G = \mathbb R^*$, 都有$aH = Ha = \begin{cases}
\mathbb R_{>0}, & a > 0\\
\mathbb R_{<0}, & a < 0\\
\end{cases}$, 故由定义, $H$是$G$的正规子群.

注意到$G = H \cup -H$且$H \cap -H = \varnothing$, 故$H$的指数$[G:H] = 2$.

\section{}
\subsection{description}
设$A, B$是群$G$的有限子群, 证明:
\begin{enumerate}
	\item $$|AB| = \frac{|A||B|}{|A \cap B|}$$
	\item 若$\gcd(|A|, |B|) = 1$, 则$|AB| = |A||B|$.
\end{enumerate}
\subsection{solution}
\begin{enumerate}
	\item 设$C = A \cap B$, 则$C$分别是$A, B$的子群. 根据陪集分解, 不妨设$A = \bigcup\limits_{i=1}^nx_iC, B = \bigcup\limits_{j=1}^mCy_j$, 其中$n = [A:C], m = [B:C]$, 就有
	\begin{align*}
		AB = \bigcup\limits_{i=1}^nx_iC\bigcup\limits_{j=1}^mCy_j = \bigcup_{(i, j) \in [n] \times [m]}x_iCy_j
	\end{align*}
	
	设$x_{i_1}Cy_{j_1} \cap x_{i_2}Cy_{j_2} \neq \varnothing$, 则存在$c_1, c_2 \in C$使得$x_{i_1}c_1y_{j_1} = x_{i_2}c_2y_{j_2}$, 于是$x_{i_2}^{-1}x_{i_1}c_1 = c_2y_{j_2}y_{j_1}^{-1} \in A \cap B = C$, 因此$x_{i_2}^{-1}x_{i_1}, y_{j_2}y_{j_1}^{-1} \in C$, 故$x_{i_1} = x_{i_2}, y_{j_1} = y_{j_2}$. 故上述分解是不交并的形式, 于是就有$$|AB| = n \times m \times |C| = \frac{|A|}{|C|}\times\frac{|B|}{|C|}\times |C| = \frac{|A||B|}{|C|}$$
	\item 由于$|C| \mid |A|$且$|C| \mid |B|$, 若$\gcd(|A|, |B|) = 1$, 则$|C| = 1$, 由上一小问结论立得$|AB| = |A||B|$.
\end{enumerate}

\section{}
\subsection{description}
\begin{enumerate}
	\item 设 $a$ 和 $b$ 是群 $G$ 的元, 阶数分别为 $m$ 和 $n$, 满足$\gcd(m, n) = 1$ 且 $ab = ba$. 求 $|\langle ab\rangle|$.
	\item 设群 $G$ 中元 $g$ 的阶 $t$ 与正整数 $n$ 互素, 在 $\langle g \rangle$ 中求解方程 $x^n = g$.
\end{enumerate}
\subsection{solution}
\begin{enumerate}
	\item 考虑证明$\langle ab \rangle = \langle a \rangle \langle b \rangle$. 首先$\langle ab \rangle \subseteq \langle a \rangle \langle b \rangle$显然, 考虑另一个方向: 任取$a^xb^y \in \langle a \rangle \langle b \rangle$, 注意到$\gcd(m, n) = 1$, 根据B\'ezout定理, 存在整数$u, v$使得$um - vn = 1$. 此时考虑$a^xb^y = a^xb^y(a^{um}b^{vn})^{y-x} = a^{x + (y - x)um}b^{y + (y - x)vn}$, 可验证$x + (y - x)um = y + (y - x)vn$, 故$a^xb^y = a^{x + (y - x)um}b^{y + (y - x)vn} \in \langle ab \rangle$, 于是$\langle a \rangle \langle b \rangle \subseteq \langle ab \rangle$, 从而有$\langle ab \rangle = \langle a \rangle \langle b \rangle$.
	
	直接代入上一题结论, 令$A = \langle a \rangle, B = \langle b \rangle$, 由于$\gcd(|A|, |B|) = \gcd(o(a), o(b)) = 1$, 故$|\langle ab \rangle| = |\langle a \rangle \langle b \rangle| = |\langle a \rangle| |\langle b \rangle| = nm$.
	\item $\gcd(t, n) = 1$说明存在唯一的$b \in [0, t)$使$at + bn = 1$, 此时$(g^b)^n = g^{bn} = g^{1 - at} = g$故$g^b$是$x^n = g$的解.
\end{enumerate}

\section{}
\subsection{description}
\begin{enumerate}
	\item 证明: 若$G$是$n$阶群, 则$G$中每个元素的阶都是$n$的因子.
	\item 证明: 素数阶群是循环群.
\end{enumerate}
\subsection{solution}
\begin{enumerate}
	\item 任意$g \in G$, 都有$\langle g \rangle$是$G$的子群. 由Lagrange定理知$|\langle g \rangle| \mid |G|$, 即$o(g) \mid n$.
	\item 任取$1 \neq g \in G$, 则$\langle g \rangle$是$G$的子群, 其阶又不是$1$, 故只能$\langle g \rangle = G$, 于是素数阶群一定是循环群.
\end{enumerate}

\section{}
\subsection{description}
设$N$是有限群$G$的正规子群, 满足$\gcd(|N|, |G / N|) = 1$. 证明: 若$a$的阶整除$|N|$, 则$a \in N$.
\subsection{solution}
考虑映射$\pi: G \to G / N$满足$\pi(a) = aN$. 由于$N$是正规子群, 故$\forall a, b \in G, \pi(a)\pi(b) = aNbN = abN^2 = abN = \pi(ab)$, 说明$\pi$是同态; 于是有$o(\pi(a)) \mid o(a)$, 又因为$o(a) \mid |N|$, 于是$o(\pi(a)) \mid |N|$. 

由于$\pi(a) \in G / N$, 根据Lagrange定理知群中元素的阶一定是群的阶的因子, 故$o(\pi(a)) \mid |G / N|$.

因为$\gcd(|N|, |G / N|) = 1$, 可以得到$o(\pi(a)) = 1$, 即$aN = \pi(a) = 1_{G / N} = N$, 这说明$a \in N$.

\section{}
\subsection{description}
证明: 除零同态外, 不存在从$(\mathbb Q, +)$到$(\mathbb Z, +)$的同态映射.
\subsection{solution}

假设存在同态映射$\sigma: \mathbb Q \to \mathbb Z$使得存在$r \in \mathbb Q, \sigma(r) = k \neq 0$. 由于同态映射保运算即$\forall a, b, \in \mathbb Q$有$\sigma(a + b) = \sigma(a) + \sigma(b)$, 故$\sigma(\frac r2) = \frac{k}{2}, \sigma(\frac r4) = \frac{k}{4}, ...$, 以此类推. 记$v_2(k)$表示$k$中包含的$2$的因子个数, 那么$\sigma(\frac{r}{2^{v_2(k) + 1}}) = \frac{k}{2^{v_2(k) + 1}} \notin \mathbb Z$矛盾. 故这样的同态映射不存在.

\section{}
\subsection{description}
设$S_n$是$n$元置换群, $G \le S_n$. 考虑映射$\text{sgn}: G \to \{0, 1\}$
$$\text{sgn}(\sigma) = \begin{cases}
0, & \sigma\mbox{是偶置换}\\
1, & \sigma\mbox{是奇置换}
\end{cases}$$
\begin{enumerate}
	\item 证明$\text{sgn}$是群同态;
	\item 求同态核;
	\item 利用这一同态证明: $G$中或者全为偶置换, 或者偶置换和奇置换的个数相等.
\end{enumerate}
\subsection{solution}
\begin{enumerate}
	\item 只需要验证$\text{sgn}$保运算即可. 两个奇置换或者两个偶置换的乘积是偶置换, 偶置换与奇置换或者奇置换与偶置换的乘积是奇置换, 因此$\forall \sigma_1, \sigma_2 \in G \le S_n$, 都有$\text{sgn}(\sigma_1\sigma_2) = \text{sgn}(\sigma_1) + \text{sgn}(\sigma_2)$.
	\item 同态核是$G$中所有偶置换构成的子群, 以下记为$K$.
	\item 若$G$中存在奇置换$\sigma_0$, 则$G = K \cup \sigma_0K$, 于是存在从$K$到$\sigma_0K$的双射, $|K| = |\sigma_0K|$, 从而$G$中奇置换与偶置换数量一样多.
\end{enumerate}

\end{document}
