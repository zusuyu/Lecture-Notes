\documentclass[UTF-8]{ctexart}

\usepackage{graphicx}
\usepackage{subfigure}
\usepackage{amsmath}
\usepackage{tabularx}
\usepackage{color}
\usepackage{hyperref}
\usepackage{ulem}
\usepackage{multirow}
\usepackage[cache=false]{minted}
\hypersetup{
	colorlinks=true,
	linkcolor=black
}

\usepackage{geometry}
\geometry{a4paper,centering,scale=0.8}
\geometry{left=2.0cm, right=2.0cm, top=2.5cm, bottom=2.5cm}
\usepackage[format=hang,font=small,textfont=it]{caption}
\usepackage[nottoc]{tocbibind}

\usepackage{algorithm}
\usepackage{algorithmicx}
\usepackage{algpseudocode}
\usepackage{amssymb}


\usepackage{tikz}  
\usetikzlibrary{arrows.meta}%画箭头用的包

\makeatletter
\def\@maketitle{%
	\newpage
	\begin{center}%
		\let \footnote \thanks
		{\LARGE \@title \par}%
		\vskip 1.5em%
		{\large
			\lineskip .5em%
			\begin{tabular}[t]{c}%
				\@author
			\end{tabular}\par}%
		\vskip 1em%
		{\large \@date}%
	\end{center}%
	\par
	\vskip 1.5em}
\makeatother


\title{\heiti\zihao{1} 逻辑习题}
\author{\kaishu\zihao{-3} 周书予\\2000013060@stu.pku.edu.cn}

\date{\today}

\begin{document}
\maketitle

\section{}
\subsection{}
\begin{enumerate}
	\item $(\psi \to \chi) \to (\varphi \to (\psi \to \chi))$ \hfill i
	\item $\psi \to \chi$
	\item $\varphi \to (\psi \to \chi)$ \hfill 1, 2, MP
	\item $(\varphi \to (\psi \to \chi)) \to ((\varphi \to \psi) \to (\varphi \to \chi))$\hfill ii
	\item $(\varphi \to \psi) \to (\varphi \to \chi)$ \hfill 3, 4, MP
	\item $\varphi \to \psi$
	\item $\varphi \to \chi$\hfill 5, 6, MP
\end{enumerate}
\subsection{}
\begin{enumerate}
	\item $(\psi \to \varphi) \to (\lnot\varphi \to \lnot\psi)$\hfill iv
	\item $((\psi \to \varphi) \to (\lnot\varphi \to \lnot\psi)) \to (\varphi \to ((\psi \to \varphi) \to (\lnot\varphi \to \lnot\psi))) $\hfill i
	\item $\varphi \to ((\psi \to \varphi) \to (\lnot\varphi \to \lnot\psi))$\hfill 1, 2, MP
	\item $(\varphi \to ((\psi \to \varphi) \to (\lnot\varphi \to \lnot\psi)))
	\to ((\varphi \to (\psi \to \varphi)) \to (\varphi \to (\lnot\varphi \to \lnot\psi)))$\hfill ii
	\item $(\varphi \to (\psi \to \varphi)) \to (\varphi \to (\lnot\varphi \to \lnot\psi))$\hfill 3, 4, MP
	\item $\varphi \to (\psi \to \varphi)$\hfill i
	\item $\varphi \to (\lnot\varphi \to \lnot\psi)$\hfill 5, 6, MP
\end{enumerate}
\subsection{}
\begin{enumerate}
	\item $\varphi \to (\lnot\varphi \to \lnot\psi)$\hfill 上一问结论
	\item $(\varphi \to (\lnot\varphi \to \lnot\psi)) \to (\lnot\varphi \to (\varphi \to \lnot \psi))$\hfill iii
	\item $\lnot\varphi \to (\varphi \to \lnot \psi)$\hfill 1, 2, MP
\end{enumerate}
\subsection{}
\begin{enumerate}
	\item $\lnot\varphi \to (\lnot\psi \to \lnot\varphi)$\hfill i
	\item $\lnot\varphi$
	\item $\lnot\psi \to \lnot\varphi$\hfill 1, 2, MP
	\item $(\lnot\psi \to \lnot\varphi) \to (\lnot\lnot\varphi \to \lnot\lnot\psi)$\hfill iv
	\item $\lnot\lnot\varphi \to \lnot\lnot\psi$\hfill 3, 4, MP
	\item $\varphi \to \lnot\lnot\varphi$\hfill v
	\item $\varphi$
	\item $\lnot\lnot\varphi$\hfill 6, 7, MP
	\item $\lnot\lnot\psi$\hfill 5, 8, MP
	\item $\lnot\lnot\psi \to \psi$\hfill vi
	\item $\psi$\hfill 9, 10, MP
	
\end{enumerate}

\section{}
若$\alpha_1, \alpha_2, \cdots, \alpha_n$是一般的逻辑表达式, 由于并不是命题变元, 这导致$\alpha_i$可能本身就是重言式, 例如$\alpha_1 = p \to p, \alpha_2 = q \wedge \lnot q$, 此时$\gamma = \alpha_1 \wedge \lnot \alpha_2$为重言, 但将$\alpha_1, \alpha_2$分别替换为$\beta_1 = p, \beta_2 = q$后得到的$\delta = \beta_1 \wedge \lnot \beta_2$不是重言式.

若$\alpha_1, \alpha_2, \cdots, \alpha_n$均为命题变元, 则等价于对于$2^n$种真值指派, $\gamma$均为真, 此时替换$\beta_1, \beta_2, \cdots, \beta_n$相当于任给$\alpha_1, \alpha_2, \cdots, \alpha_n$作真值指派, 故得到的$\delta$是重言式.

\section{}
由$\lnot (A * A)$永真可知$\mathbf T * \mathbf T = \mathbf F * \mathbf F = \mathbf F$. 由$(\mathbf T * \mathbf T) * \mathbf T = \mathbf T$可知$\mathbf F * \mathbf T = \mathbf T$, 而$(\mathbf T * \mathbf F) * \mathbf F = \mathbf F$说明了$\mathbf T * \mathbf F = \mathbf F$(否则左式的结果就变成了$\mathbf T$). 综上, $*$的真值表为

\begin{center}
\begin{tabular}{c|c|c}
	$p_1$ & $p_2$ & $p_1 * p_2$\\
	\hline
	$\mathbf T$ & $\mathbf T$ & $\mathbf F$\\
	\hline
	$\mathbf T$ & $\mathbf F$ & $\mathbf F$\\
\hline
	$\mathbf F$ & $\mathbf T$ & $\mathbf T$\\
\hline
	$\mathbf F$ & $\mathbf F$ & $\mathbf F$\\

\end{tabular}
\end{center}

\section{}

\subsection{}
$$[x = 0] := [\forall y, (x \le y)]$$
\subsection{}
$$[x = y + 1] := [\lnot (x \le y) \wedge (\forall z, (y \le z) \vee (z \le x))]$$
\subsection{}
$$[x = 3] := [\exists y, z, w, (w = 0) \wedge (z = w + 1) \wedge (y = z + 1) \wedge (x = y + 1)]$$
\end{document}
