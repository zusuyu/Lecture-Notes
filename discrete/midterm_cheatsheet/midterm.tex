\documentclass[UTF-8]{ctexart}

\usepackage{graphicx}
\usepackage{subfigure}
\usepackage{amsmath}
\usepackage{tabularx}
\usepackage{color}
\usepackage{hyperref}
\usepackage{ulem}
\usepackage{multirow}
\usepackage[cache=false]{minted}
\hypersetup{
	colorlinks=true,
	linkcolor=black
}

\usepackage{geometry}
\geometry{a4paper,centering,scale=0.8}
\geometry{left=0.5cm, right=1.0cm, top=1.0cm, bottom=0.5cm}
\usepackage[format=hang,font=small,textfont=it]{caption}
\usepackage[nottoc]{tocbibind}

\usepackage{algorithm}
\usepackage{algorithmicx}
\usepackage{algpseudocode}
\usepackage{amssymb}


\usepackage{tikz}  
\usetikzlibrary{arrows.meta}%画箭头用的包

\makeatletter
\def\@maketitle{%
	\newpage
	\begin{center}%
		\let \footnote \thanks
		{\LARGE \@title \par}%
		\vskip 0.5em%
		{\large
			\lineskip .5em%
			\begin{tabular}[t]{c}%
				\@author
			\end{tabular}\par}%
		\vskip 0.5em%
		{\large \@date}%
	\end{center}%
	\par
	\vskip 0.5em}
\makeatother


\title{\heiti\zihao{1} }
\author{\kaishu\zihao{-3} 周书予\\2000013060@stu.pku.edu.cn}

\date{\today}

\begin{document}

\textbf{ZFC公理体系十条}
\begin{enumerate}
	\item 外延公理。\textit{两个集合所有元素相同,则相等。}$\forall X \forall Y (\forall x(x \in X \Leftrightarrow x \in Y) \Rightarrow X = Y)$
	\item 分离公理。\textit{可以通过性质在已有集合中选子集。}$\forall Y \exists X \forall x(x \in X \Leftrightarrow (x \in Y \wedge \phi(x)))$
	\item 空集公理。\textit{存在不包含任何元素的集合。}$\exists X(\forall x(x \notin X))$
	\item 配对公理。\textit{把两个集合外面套个括号。}$\forall X\forall Y\exists Z \forall u(u \in Z\Leftrightarrow(u = X \vee u = Y))$
	\item 并集公理。\textit{把集合族所有集合的元素拆出来放一起。}$\forall I\exists A \forall x(x \in X \Leftrightarrow \exists i \in I(x \in i))$
	\item 幂集公理。\textit{存在一切子集构成的集合。}$\forall X\exists Y\forall Z(Z \subseteq X \Leftrightarrow Z \in Y)$
	\item 正则公理。\textit{每个集合都存在一个与自己交为空的元素,可以用来证明集合不存在。}$\forall X(X \neq \varnothing \Leftrightarrow \exists y \in X, y \cap X = \varnothing)$
	\item 替换公理。$\forall x \forall y \forall z(\phi(x, y, p) \wedge \phi(x, z, p) \Rightarrow y = z)\Rightarrow \forall X \exists Y \forall y(y \in Y \Leftrightarrow (\exists x \in X)\phi(x, y, p))$
	\item 无穷公理。\textit{归纳集存在。}
	\item 选择公理。能从集合族每个集合里选出来一个元素。$\forall X(\varnothing \notin X \Rightarrow \exists f: X \to \bigcup X, \forall A \in X(f(A) \in A))$
\end{enumerate}

用无穷公理与分离公理可以推出空集公理。用空集公理、幂集公理与替换公理可以推出配对公理。构造多元素集合需要配对公理和并集公理交替运用。构造笛卡尔积需要用两次幂集。

\textbf{归纳集} 定义为$0 = \varnothing \in A, \forall n \in A, n^+ := n \cup \{n\} \in A$。所有归纳集的交良定且唯一,这个集合为自然数集$\omega$。

\textbf{Peano 公理}\ 
\begin{enumerate}
	\item $0 = \varnothing$是自然数
	\item 每个自然数都有唯一后继
	\item $0$不是任意自然数的后继
	\item 不同自然数有不同的后继
	\item $\omega$满足归纳原理,$\omega := \bigcap \{Y: Y \subseteq X\mbox{且为归纳集}\}$,其中$X$是由无穷公理确保的一个归纳集
\end{enumerate}


\textbf{传递集} 定义为$a \in A \Rightarrow a \subseteq A$或者等价地$a' \in a \in A \Rightarrow a' \in A$。每个自然数都是传递集。

若$A$为传递集,则$\bigcup A, \bigcap A$也是。$A$是传递集等价于$\bigcup A \subseteq A$,等价于$A \subseteq \mathcal P(A)$,后者等价于$\mathcal P(A)$是传递集。

传递集的集合的并也是传递集。\textit{$\forall x \in \bigcup S$,存在某个$X_0 \in S$使得$x \in X_0$,$X_0$传递知$x \subseteq X_0$,而$X_0 \subseteq \bigcup S$,故$x \subseteq \bigcup S$,$\bigcup S$传递}

\textbf{Knaster-Tarski不动点定理}

设$X$是非空集合,映射$f: \mathcal P(X) \to \mathcal P(X)$满足对于任意的$A \subseteq B \subseteq X$都有$f(A) \subseteq f(B)$。证明存在“不动点”$T \subseteq X$使得$f(T) = T$。

\textit{证明}\ 取$T = \bigcup\{A: A\mbox{满足}A \subseteq f(A)\}$,一方面$T \subseteq \bigcup\{f(A): A\mbox{满足}A \subseteq f(A)\} \subseteq f(\bigcup\{A: A\mbox{满足}A \subseteq f(A)\}) = f(T)$(第一个$\subseteq$是因为$A \subset f(A)$,第二个是因为$f$的单调性,每一个$f(A)$都包含于$f(T)$于是$\bigcup f(A)$也包含于$f(T)$);另一方面由于$T \subseteq f(T) \Rightarrow f(T) \subseteq f(f(T))$,说明$f(T)$也是之前$\bigcup$中的一项,导致$f(T) \subseteq T$,从而$f(T) = T$。\hfill$\blacksquare$


\textbf{良序}\ 良序是指任意集合都有最小元。

已知$(X, <)$是一个良序集。

$f: X \to X$增函数,则$\forall x \in X, f(x) \ge x$ $\Rightarrow$ $f: X \to X$保序同构映射只能是恒等映射。$\Rightarrow$ $f: X \to Y$的保序同构映射唯一。

可以通过保序同构映射来证明良序集的三歧性。

\newpage

\textbf{序数}\ 关于序数有几条性质

$X$是在以$\in$定义的序下的良序集,则$X$是序数 iff $\forall x \in X, s_x = x$。\textit{(充分性证明传递,必要性证两个方向的$\subseteq$)}

$\alpha, \beta$是不同的序数,则$\alpha \in \beta$ iff $\alpha \subseteq \beta$。\textit{(充分性证明$\beta \setminus \alpha$中最小元$= \alpha$,需要证两个$\subseteq$。必要性是定义)}

序数中的任意元素都是序数。特别地,$0$属于任何序数。\textit{(证明任意元素的良序和传递)}

序数$A, A^+$之间没有新的序数,或者等价的,$A < B \Rightarrow A^+ \le B$。

任意序数等于其绝对前段$x = s(x)$。\textit{(证明两边的$\subseteq$)}

由序数构成的集合在$\in$的序下是良序的。

\textit{等价于证明任意由序数构成的集合存在最小元。任取$\alpha \in A$,要么$\alpha$就是$A$中最小元,要么$A \cap \alpha$非空,由$\alpha$良序可知$A \cap \alpha$中存在最小元$\gamma$,断言这就是$A$的最小元:若不然,存在$\gamma' \in \gamma \in \alpha$,传递性导出$\gamma' \in \alpha$,这与$\gamma$的最小性矛盾。}

不存在以所有序数为元素的集合。\textit{先证明$A$是序数$\Rightarrow \bigcup A$是序数},然后指出$A \subseteq \bigcup A$,从而例如$(\bigcup A)^+$就不可能是$A$的元素。

\textbf{Cantor-Schroder-Bernstein定理}

若存在集合$S, T$之间双向的单射$f: S \to T$和$g: T \to S$,则$S \sim T$。

\textit{证明}\ 考虑定义映射$F: \mathcal P(S) \to \mathcal P(S)$满足$F(A) = S - g(T - f(A))$。首先验证$F$满足单调性($A \subseteq B \Rightarrow f(A) \subseteq f(B) \Rightarrow T - f(B) \subseteq T - f(A) \Rightarrow g(T - f(B))\subseteq g(T - f(A)) \Rightarrow S - g(T - f(A)) \subseteq S - g(T - f(B))$),然后根据Knaster-Tarski不动点定理可知存在$A^*$使得$F(A^*)= A^*$,于是构造双射$h(x) = \begin{cases}
f(x) & x \in A^*\\g^{-1}(x) & x \in g(T - f(A^*))
\end{cases}$,需要验证一下两种case都是双射。\hfill $\blacksquare$

\textbf{基数}的定义:序数$\alpha$被称为基数,如果序数$\beta$与$\alpha$对等能推出$\alpha \le \beta$。\textit{可以理解成“最小的序数”。}

\textbf{对角线法}\ 用来说明两个集合不等势。证法通常是对于映射$f$,利用$x_i$和$f(x_i)$来构造出一个$\notin \text{Im} f$的元素,从而说明不可能存在满射。
\begin{enumerate}
	\item \textbf{Cantor 定理}$A \neq \varnothing$,则$A$和$\mathcal P(A)$不对等。对于$f: A \to \mathcal P(A)$,考虑对角线上每一位取反,即构造集合$B:= \{x \in A, x \notin f(x)\}$,则$B \notin \text{Im} f$,故满射不存在。
	\item $\kappa_i, \lambda_i$为两族基数,且$\forall i \in I, \kappa_i < \lambda_i$,于是有$\sum_{i \in I}\kappa_i < \prod_{i \in I}\lambda_i$。这里基数的$\sum$可以理解为不交并(就是加个第二维然后并起来),$\prod$可以理解为笛卡尔积。对于任意映射$f$,对于任意$i \in I$,总存在$\alpha_i \in \lambda_i$使得$\alpha_i$不等于任何$\sum_{i \in I}\kappa_i$在$f$下的像的第$i$位,因此构造$\alpha = (\alpha_1, \cdots) \notin \text{Im}f$说明不对等。
\end{enumerate}

\textbf{可数集}\ 与自然数对等的集合称为可数集。$\mathbb Q$是可数集。可数集的可数并与有限笛卡尔积是可数集。

要证明什么东西可数,可以尝试把它拆成可数个可数集的并。比如说证明从中任取实数构成正项级数序列总收敛的正实数集可数,只需要说明对于任意$n \in \mathbb N$,集合中$\ge \frac{1}{n}$的元素个数有限即可。

$\text{card}\mathcal P(\mathbb N) = c$,即$\mathcal P(\mathbb N)$与$\mathbb R$等势。考虑$f(x) = \{r \in \mathbb Q, r \le x\}$这是$\mathbb R$到$\mathcal P(\mathbb N)$的单射,而另一个方向的单射是很好构造的,所以根据Cantor-Schroder-Bernstein定理可知等势。

析取范式(Disjunctive Normal Form)是$\bigvee_{i=1}^m(\bigwedge_{j=1}^{n}q_{ij})$, 合取范式(Conjunctive Normal Form)是$\bigwedge_{i=1}^m(\bigvee_{j=1}^{n}q_{ij})$。构造析取范式,就是把所有真值为$\mathbf T$的情况列出来,构造合取范式,就是把所有真值为$\mathbf F$的情况列出来。

\textbf{命题逻辑的形式化推理}\ 假言三段论:如果有$\varphi \to \psi$和$\psi \to \chi$,先通过L1得到$\varphi \to (\psi \to \chi)$,再利用L2得到$(\varphi \to \psi) \to (\varphi \to \chi)$。

\textbf{语法和语义/可靠与完备}\ 语法是形式系统$\vdash_L \varphi$,语义是真值$\vDash_L \varphi$。可靠性是$\vdash_L\varphi \Rightarrow \vDash_L\varphi$,完备性是$\vDash_L\varphi \Rightarrow \vdash_L\varphi$。

可靠性是容易证明的,只需要对推导序列的公式个数作归纳,根据MP的重言性可知推出来的都是重言式。

证明完备性的过程略微繁琐。首先引入了扩充的概念,紧接着定义了一致扩充(存在某个公式不是该形式系统的定理)与完全扩充(任何一个公式$\varphi$,$\varphi, \lnot\varphi$中恰好有一个属于该形式系统),证明一致的形式系统一定存在使定理全为$\mathbf T$的赋值(先完全扩充到$J$,然后定义$v(\varphi) = \begin{cases}
\mathbf T & \vdash_J\varphi\\
\mathbf F & \vdash_J\lnot\varphi
\end{cases}$,反证$v(\varphi \to \psi) = \mathbf F$当且仅当$v(\varphi) = \mathbf T$且$v(\psi) = \mathbf F$)。最后说明$L$是完备的:假设存在一个不在$L$中的重言式,那么把$\lnot\varphi$加到$L$中得到的$L^*$是一致的,而一致的形式系统存在赋值使所有定理取值$\mathbf T$,这与$\varphi$重言矛盾。
\end{document}