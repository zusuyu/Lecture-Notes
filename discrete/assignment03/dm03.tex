\documentclass[UTF-8]{ctexart}

\usepackage{graphicx}
\usepackage{subfigure}
\usepackage{amsmath}
\usepackage{tabularx}
\usepackage{color}
\usepackage{hyperref}
\usepackage{ulem}
\usepackage{multirow}
\usepackage[cache=false]{minted}
\hypersetup{
	colorlinks=true,
	linkcolor=black
}

\usepackage{geometry}
\geometry{a4paper,centering,scale=0.8}
\geometry{left=2.0cm, right=2.0cm, top=2.5cm, bottom=2.5cm}
\usepackage[format=hang,font=small,textfont=it]{caption}
\usepackage[nottoc]{tocbibind}

\usepackage{algorithm}
\usepackage{algorithmicx}
\usepackage{algpseudocode}
\usepackage{amssymb}


\usepackage{tikz}  
\usetikzlibrary{arrows.meta}%画箭头用的包

\makeatletter
\def\@maketitle{%
	\newpage
	\begin{center}%
		\let \footnote \thanks
		{\LARGE \@title \par}%
		\vskip 1.5em%
		{\large
			\lineskip .5em%
			\begin{tabular}[t]{c}%
				\@author
			\end{tabular}\par}%
		\vskip 1em%
		{\large \@date}%
	\end{center}%
	\par
	\vskip 1.5em}
\makeatother

\def \card {\textrm{card}}

\title{\heiti\zihao{1}   基数习题}
\author{\kaishu\zihao{-3} 周书予\\2000013060@stu.pku.edu.cn}

\date{\today}

\begin{document}
\maketitle

{\color{blue}
\section{Prob 1}
\subsection{Statement}
设$A \subseteq B$且$\card A = \card A \cup C$, 证明$\card B = \card B \cup C$.
\subsection{Solution}
}

\section{Prob 6}
\subsection{Statement}
$A, B$是集合,判断下列两个命题的真假:
\begin{enumerate}
	\item 若$A_1 \subseteq A, B_1 \subseteq B$,且$A_1 \sim B_1, A \sim B$,则$A \setminus A_1 \sim B \setminus B_1$;
	\item 若$A \setminus B \sim B \setminus A$,则$A \sim B$。
\end{enumerate}
\subsection{Solution}
\begin{enumerate}
	\item 假。可举反例$A = [0, 1], A_1 = (0, 1), B = [0, 1), B_1 = (0, 1)$,则$A_1 \sim B_1, A \sim B$,而$A \setminus A_1 = \{0, 1\} \nsim B \setminus B_1 = \{0\}$。
	\item 真。$A \setminus B \sim B \setminus A$说明存在双射$f: A\setminus B \to B \setminus A$,考虑构造$g: A \to B$满足$g(x) = \begin{cases}
	x, & x \in A \cap B\\f(x), & x \in A \setminus B
	\end{cases}$,容易验证这是一个双射,从而$A \sim B$。
\end{enumerate}

\section{Prob 7}
\subsection{Statement}
把是整系数多项式根的实数称为代数数,否则称为超越数。计算全体超越数构成的集合的基数。
\subsection{Solution}
对于整系数多项式$f(x) = \sum_{i=0}^{n}a_ix^i$,定义其“等级”$\mathrm{level}(f) = \max\{n, \max_{i=0}^{n}\{|a_i|\}\}$。记$G_n = \{f \in \mathbb Z[x]: \mathrm{level}(f) = n\}$,容易证明$G_n$中仅含有有限个元素,而$\mathbb Z[x] = \bigcup_{n=0}^{\infty}G_n$是有限集合的可数并,故$\mathbb Z[x] \sim \mathbb N$。又由于每个整系数多项式都仅有有限个代数根,故代数数集$\sim \mathbb N \times \mathbb N \sim \mathbb N$。

由代数数集$\cup$超越数集 $ = \mathbb R$知card(代数数集) + card(超越数集) = card($\mathbb R$) = $c$,结合代数数集$\sim \mathbb N$可知card(超越数集) = $c$。


\section{Prob 9}
\subsection{Statement}
求证:若$\bigcup_{i=1}^{\infty}A_i$与$\mathbb R$对等,则存在某个$A_n$与$\mathbb R$对等。
\subsection{Solution}

假设任何一个$A_n$都不与$\mathbb R$对等,这说明对于每个$A_n$,任取映射$f_n: A_n \to \mathbb R^{\mathbb N}$,总会存在$\alpha_n \in \mathbb R$,使得$\alpha_n$不等于$A_n$中任何一个元素在$f_n$下的像(是一个实数序列)的第$n$位。

考虑实数序列$\alpha = \{\alpha_1, \alpha_2, \cdots, \alpha_n, \cdots\}$,显然$\forall n \in \mathbb N, \alpha \notin \textrm{Im} f_n$,注意到$f_n$的任意性,这使得$F: \bigcup_{n=1}^{\infty}A_n \to \mathbb R^{\mathbb N}$的双射无法构造,这与题目条件“$\bigcup_{n=1}^{\infty}A_n \sim \mathbb R$”矛盾。故一定存在某个$A_n$与$\mathbb R$对等。

\section{Prob 10}
\subsection{Statement}
设$\{A_i: i \in I\}$是一列集合,基数$\text{card}A_i = \alpha_i \le \beta_i(\forall i \in I)$,求证$$\text{card}(\bigcup_{i\in I}A_i) \le \sum_{i \in I}\alpha_i \le \sum_{i \in I}\beta_i$$

并导出可数集的可数并是可数集作为特例。
\subsection{Solution}

$\sum_{i \in I}\alpha_i$可以理解为$\text{card}\{(i, a) | i \in I, a \in A_i\}$,欲证明$\text{card}(\bigcup_{i\in I}A_i) \le \text{card}\{(i, a) | i \in I, a \in A_i\}$,只需要证明存在从$\bigcup_{i\in I}A_i$到$\{(i, a) | i \in I, a \in A_i\}$的单射即可。

$\forall x \in \bigcup_{i\in I}A_i$,记$Index_x = \{i \in I | x \in A_i\}$,根据选择公理,存在映射$F: \bigcup_{i\in I}A_i \to I$使得$\forall x \in \bigcup_{i\in I}A_i, F(x) \in Index_x$即$x \in A_{F(x)}$,故可以构造单射
\begin{align*}
	\sigma_1: \bigcup_{i\in I}A_i &\to \{(i, a) | i \in I, a \in A_i\}\\
	x &\to (F(x), x)
\end{align*}

考虑$\alpha_i \le \beta_i$,等价于存在从$\alpha_i$到$\beta_i$的单射$f_i$,于是可以自然地构造从$\sum_{i\in I}\alpha_i$到$\sum_{i\in I}\beta_i$的单射
\begin{align*}
\sigma_2: \sum_{i\in I}\alpha_i &\to \sum_{i\in I}\beta_i\\
(i, x) &\to (i, f_i(x))
\end{align*}

令每个$A_i \sim \mathbb N$,则根据结论有

$$\text{card}(\bigcup_{i\in I}A_i) \le \sum_{i \in I}\text{card}(\mathbb N) = \text{card}(\mathbb N)$$

由$\text{card}(\bigcup_{i\in I}A_i) \ge \text{card}(A_i) = \text{card}(\mathbb N)$得到$\text{card}(\bigcup_{i\in I}A_i) = \text{card}(\mathbb N)$即可数集的可数并是可数集。

\end{document}
