\documentclass[8pt]{article}
\usepackage[UTF8]{ctex}
\usepackage[a4paper]{geometry}

\usepackage{amsthm,amsmath,amssymb}
\usepackage{graphicx}
\usepackage{subfigure}
\usepackage{amsmath}
\usepackage{tabularx}
\usepackage{color}
\usepackage{hyperref}
\usepackage{ulem}
\usepackage{multirow}
\usepackage[cache=false]{minted}
\hypersetup{
	colorlinks=true,
	linkcolor=blue
}

\usepackage{appendix}
\geometry{a4paper,centering,scale=0.8}
\geometry{left=2.0cm, right=2.0cm, top=2.5cm, bottom=2.5cm}
\usepackage[format=hang,font=small,textfont=it]{caption}
\usepackage[nottoc]{tocbibind}

\usepackage{algorithm}
\usepackage{algorithmicx}
\usepackage{algpseudocode}
\usepackage{amssymb}
\usepackage{extarrows}
\usepackage{qcircuit}
\usepackage{fancyhdr}
\usepackage{cleveref}
\usepackage{bbm}

\usepackage{tikz}  
\usetikzlibrary{arrows.meta}%画箭头用的包

\makeatletter
\def\@maketitle{%
	\newpage
	\begin{center}%
		\let \footnote \thanks
		{\LARGE \@title \par}%
		\vskip 1.5em%
		{\large
			\lineskip .5em%
			\begin{tabular}[t]{c}%
				\@author
			\end{tabular}\par}%
		\vskip 1em%
		{\large \@date}%
	\end{center}%
	\par
	\vskip 1.5em}
\makeatother

\newtheoremstyle{compact}%
{3pt}{3pt}%
{}{}%
{\bfseries}{\textcolor{red}{.}}%  % Note that final punctuation is omitted.
{.5em}{\mbox{\textcolor{red}{\thmname{#1}\thmnumber{ #2}}\thmnote{ (\textcolor{blue}{#3})}}}
\theoremstyle{compact}
\newtheorem{innercustomgeneric}{\customgenericname}
\providecommand{\customgenericname}{}
\newcommand{\newcustomtheorem}[2]{%
	\newenvironment{#1}[1]
	{%
		\renewcommand\customgenericname{#2}%
		\renewcommand\theinnercustomgeneric{##1}%
		\innercustomgeneric
	}
	{\endinnercustomgeneric}
}

\DeclareMathOperator{\card}{card}

\newtheorem{theorem}{定理}
\newtheorem{lemma}{Lemma}
\newtheorem{definition}{定义}
\newtheorem{proposition}{命题}
\newtheorem{corollary}{推论}
\newtheorem{example}{例}
\newtheorem{claim}{声明}
\newtheorem{remark}{注}
\newtheorem{thesis}{论点}
\newtheorem{Proof}{证明}

\def\obj#1{\textbf{\uline{#1}}}
\def\num#1{\textnormal{\textbf{\mbox{\textcolor{blue}{(#1)}}}}}
\def\le{\leqslant}
\def\ge{\geqslant}
\def\im{\text{im }}
\def\P#1{\mathbb{P}\left({#1}\right)}
\def\e{\mathrm{e}}
\def\E#1{\mathbb{E}\left[{#1}\right]}
\def\Var#1{\text{Var}\left[{#1}\right]}


\title{\heiti\zihao{2} Machine Learning Homework: Week 9 \& 10}
\author{\kaishu\zihao{-3} 周书予\\2000013060@stu.pku.edu.cn}

\CTEXoptions[today=old]
\date{\today}

\usepackage{totpages}
\begin{document}
\fancypagestyle{plain}{
	\fancyhf{}
	\rhead{Homework: Week 9 \& 10}
	\chead{2022 Spring}
	\lhead{Machine Learning}
	\cfoot{Page \thepage \ of \pageref{TotPages}}
}
\pagestyle{plain}
\crefname{theorem}{定理}{定理}
\crefname{lemma}{Lemma}{Lemma}
\crefname{figure}{图}{图}
\crefname{table}{表}{表}	
\maketitle

\begin{algorithm}
	\caption{Randomized Weighted Updating}
	\begin{algorithmic}[1]
		\State Initialize $w_{1, i} \gets 1, \forall i \in [n]$
		\State Choose parameter $\beta \in [\frac12, 1)$
		\For{$t = 1 \to T$}
			\State Chooes $\tilde{y_t} = \tilde{y_{t, i}}$ with probability proportional to $w_{t, i}$
			\State $w_{t + 1, i} \gets \begin{cases}
				\beta \cdot w_{t, i}, & \tilde{y_{t, i}} \neq y_t\\
				w_{t, i}, & \tilde{y_{t, i}} = y_t
			\end{cases}, \forall i \in [n]$
		\EndFor
	\end{algorithmic}
\end{algorithm}
\begin{theorem}
	在Randomized Weighted Updating 算法下, 有 $$\E{L_T} \le (2 - \beta)m_T^* + \frac{\ln n}{1 - \beta}$$
	\label{random-weighted-update}
\end{theorem}
\begin{proof}
	注意到权值的更新无关于每轮有没有答错, 因此 $\mathbbm 1 [\tilde{y_t} \neq y_t]$ 是独立随机变量. 
	
	第 $i$ 轮结束后, 总权值的变化一定是 $W \to W(1 - (1 - \beta)\P{\tilde{y_t} \neq y_t})$, 由于 $\E{L_T} = \sum\limits_{t=1}^T\P{\tilde{y_t} \neq y_t}$, 因此

	$$\beta^{m_T^*} \le n \prod_{t=1}^{T}(1 - (1 - \beta)\P{\tilde{y_t} \neq y_t}) \le n\prod_{t=1}^{T}\e^{-(1-\beta)\P{\tilde{y_t} \neq y_t}} = n\e^{-(1 - \beta)\E{L_T}}$$ 从而得到了 $$\E{L_T} \le \frac{\ln(1/\beta)m_T^* + \ln n}{1 - \beta}$$

	只需要进一步证明 $\frac{\ln(1 / \beta)}{1 - \beta} \le 2 - \beta$. 考虑函数 $f(\beta) = \ln\beta + (1 - \beta)(2 - \beta), f'(\beta) = \frac{(1-  \beta)(1 - 2\beta)}{\beta}$, 当 $\beta \in [\frac12, 1)$ 时恒有 $f'(\beta) \le 0$, 从而 $f(\beta) \ge f(1) = 0$, 说明了 $\ln(1 / \beta) \le (1 - \beta)(2 - \beta), \frac{\ln(1 / \beta)}{1 - \beta} \le 2 - \beta$.
\end{proof}

\end{document}


