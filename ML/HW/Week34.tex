\documentclass[8pt]{article}
\usepackage[UTF8]{ctex}
\usepackage[a4paper]{geometry}

\usepackage{amsthm,amsmath,amssymb}
\usepackage{graphicx}
\usepackage{subfigure}
\usepackage{amsmath}
\usepackage{tabularx}
\usepackage{color}
\usepackage{hyperref}
\usepackage{ulem}
\usepackage{multirow}
\usepackage[cache=false]{minted}
\hypersetup{
	colorlinks=true,
	linkcolor=blue
}

\usepackage{appendix}
\geometry{a4paper,centering,scale=0.8}
\geometry{left=2.0cm, right=2.0cm, top=2.5cm, bottom=2.5cm}
\usepackage[format=hang,font=small,textfont=it]{caption}
\usepackage[nottoc]{tocbibind}

\usepackage{algorithm}
\usepackage{algorithmicx}
\usepackage{algpseudocode}
\usepackage{amssymb}
\usepackage{extarrows}
\usepackage{qcircuit}
\usepackage{fancyhdr}
\usepackage{cleveref}

\usepackage{tikz}  
\usetikzlibrary{arrows.meta}%画箭头用的包

\makeatletter
\def\@maketitle{%
	\newpage
	\begin{center}%
		\let \footnote \thanks
		{\LARGE \@title \par}%
		\vskip 1.5em%
		{\large
			\lineskip .5em%
			\begin{tabular}[t]{c}%
				\@author
			\end{tabular}\par}%
		\vskip 1em%
		{\large \@date}%
	\end{center}%
	\par
	\vskip 1.5em}
\makeatother

\newtheoremstyle{compact}%
{3pt}{3pt}%
{}{}%
{\bfseries}{\textcolor{red}{.}}%  % Note that final punctuation is omitted.
{.5em}{\mbox{\textcolor{red}{\thmname{#1}\thmnumber{ #2}}\thmnote{ (\textcolor{blue}{#3})}}}
\theoremstyle{compact}
\newtheorem{innercustomgeneric}{\customgenericname}
\providecommand{\customgenericname}{}
\newcommand{\newcustomtheorem}[2]{%
	\newenvironment{#1}[1]
	{%
		\renewcommand\customgenericname{#2}%
		\renewcommand\theinnercustomgeneric{##1}%
		\innercustomgeneric
	}
	{\endinnercustomgeneric}
}

\DeclareMathOperator{\card}{card}

\newtheorem{theorem}{定理}
\newtheorem{lemma}{引理}
\newtheorem{definition}{定义}
\newtheorem{proposition}{命题}
\newtheorem{corollary}{推论}
\newtheorem{example}{例}
\newtheorem{claim}{声明}
\newtheorem{remark}{注}
\newtheorem{thesis}{论点}
\newtheorem{Proof}{证明}

\def\obj#1{\textbf{\uline{#1}}}
\def\num#1{\textnormal{\textbf{\mbox{\textcolor{blue}{(#1)}}}}}
\def\le{\leqslant}
\def\ge{\geqslant}
\def\im{\text{im }}
\def\P#1{\mathbb{P}\left({#1}\right)}
\def\e{\mathrm{e}}
\def\E#1{\mathbb{E}\left[{#1}\right]}
\def\Var#1{\text{Var}\left[{#1}\right]}


\title{\heiti\zihao{2} Machine Learning Homework: Week 3 \& 4}
\author{\kaishu\zihao{-3} 周书予\\2000013060@stu.pku.edu.cn}

\CTEXoptions[today=old]
\date{\today}

\usepackage{totpages}
\begin{document}


\fancypagestyle{plain}{
	\fancyhf{}
	\lhead{机器学习}
	\chead{2022 Spring}
	\rhead{Machine Learning}
	\cfoot{第 \thepage 页, 共 \pageref{TotPages} 页}
}
\pagestyle{plain}

\crefname{theorem}{定理}{定理}
\crefname{lemma}{引理}{引理}
\crefname{figure}{图}{图}
\crefname{table}{表}{表}	
\maketitle

\section{}
\subsection*{Statement}
Prove that
\begin{equation}
	\begin{split}
		\frac12 \P{\sup_{\phi \in \Phi}\left|\mathbb E_z{\phi(z)} - \frac1n\sum_{i=1}^{n}\phi(z_i)\right| \ge 2\varepsilon} 
		&\le \P{\sup_{\phi \in \Phi}\left|\frac1n\sum_{i=1}^{n}\phi(z_i) - \frac1n\sum_{i=n+1}^{2n}\phi(z_i)\right| \ge \varepsilon} \\
		&\le 2\P{\sup_{\phi \in \Phi}\left|\mathbb E_z{\phi(z)} - \frac1n\sum_{i=1}^{n}\phi(z_i)\right| \ge \frac{\varepsilon}{2}}
	\end{split}
\end{equation}
\subsection*{Solution}
\begin{equation}
	\begin{split}
		&\P{\sup_{\phi \in \Phi}\left|\frac1n\sum_{i=1}^{n}\phi(z_i) - \frac1n\sum_{i=n+1}^{2n}\phi(z_i)\right| \ge \varepsilon} \\
		=& \P{\forall \delta > 0, \exists \phi \in \Phi \text{ s.t. } \left|\frac1n\sum_{i=1}^{n}\phi(z_i) - \frac1n\sum_{i=n+1}^{2n}\phi(z_i)\right| > \varepsilon - \delta} \\
		\le& \P{\forall \delta > 0, \exists \phi \in \Phi \text{ s.t. } \left|\frac1n\sum_{i=1}^{n}\phi(z_i) - \mathbb E_z\phi(z)\right| > \frac{\varepsilon - \delta}{2} \text{ or } \left|\frac1n\sum_{i=n+1}^{2n}\phi(z_i) - \mathbb E_z\phi(z)\right| > \frac{\varepsilon - \delta}{2}} \\
		\le& \P{\forall \delta > 0, \exists \phi \in \Phi \text{ s.t. } \left|\frac1n\sum_{i=1}^{n}\phi(z_i) - \mathbb E_z\phi(z)\right| > \frac{\varepsilon - \delta}{2} } + \P{ \forall \delta > 0, \exists \phi \in \Phi \text{ s.t. } \left|\frac1n\sum_{i=n+1}^{2n}\phi(z_i) - \mathbb E_z\phi(z)\right| > \frac{\varepsilon - \delta}{2}} \\
		=& \P{\sup_{\phi \in \Phi}\left|\frac1n\sum_{i=1}^{n}\phi(z_i)-\mathbb E_z{\phi(z)}\right| \ge \frac{\varepsilon}{2}} + \P{\sup_{\phi \in \Phi}\left|\frac1n\sum_{i=n+1}^{2n}\phi(z_i)-\mathbb E_z{\phi(z)}\right| \ge \frac{\varepsilon}{2}}\\
		=& 2\P{\sup_{\phi \in \Phi}\left|\mathbb E_z{\phi(z)} - \frac1n\sum_{i=1}^{n}\phi(z_i)\right| \ge \frac{\varepsilon}{2}}
	\end{split}
\end{equation}

which provides the proof for the inequality on the right side. 

\begin{equation}
	\begin{split}
		&\P{\sup_{\phi \in \Phi}\left|\frac1n\sum_{i=1}^{n}\phi(z_i) - \frac1n\sum_{i=n+1}^{2n}\phi(z_i)\right| \ge \varepsilon} \\
		=& \P{\forall \delta > 0, \exists \phi \in \Phi \text{ s.t. } \left|\frac1n\sum_{i=1}^{n}\phi(z_i) - \frac1n\sum_{i=n+1}^{2n}\phi(z_i)\right| > \varepsilon - \delta} \\
		\ge& \P{\forall \delta > 0, \exists \phi \in \Phi \text{ s.t. } \left|\frac1n\sum_{i=1}^{n}\phi(z_i) - \mathbb E_z\phi(z)\right| > 2\varepsilon - \delta \text{ and } \left|\frac1n\sum_{i=n+1}^{2n}\phi(z_i) - \mathbb E_z\phi(z)\right| < \varepsilon} \\
		=& \P{\forall \delta > 0, \exists \phi_0 \in \Phi \text{ s.t. } \left|\frac1n\sum_{i=1}^{n}\phi_0(z_i) - \mathbb E_z\phi_0(z)\right| > 2\varepsilon - \delta }\P{\left|\frac1n\sum_{i=n+1}^{2n}\phi_0(z_i) - \mathbb E_z\phi_0(z)\right| < \varepsilon} \\
	\end{split}
\end{equation}

By Chernoff bound we obtain that $\P{\left|\frac1n\sum\limits_{i=n+1}^{2n}\phi(z_i) - \mathbb E_z\phi(z)\right| < \varepsilon} \ge \frac12$ when $n \ge \frac{\ln 2}{\varepsilon^2}$, which gives the proof for the inequality on the left side

\begin{equation}
	\begin{split}
		\cdots &\ge \frac12 \P{\forall \delta > 0, \exists \phi \in \Phi \text{ s.t. } \left|\frac1n\sum_{i=1}^{n}\phi(z_i) - \mathbb E_z\phi(z)\right| > 2\varepsilon - \delta }\\
		&= \frac12 \P{\sup_{\phi \in \Phi}\left|\mathbb E_z{\phi(z)} - \frac1n\sum_{i=1}^{n}\phi(z_i)\right| \ge 2\varepsilon} 
	\end{split}
\end{equation}


\section{}
\subsection*{Statement}
Prove that $$\sum_{k=0}^d\binom{n}{k} < \left(\frac{\e n}{d}\right)^d$$ for any $n \in \mathbb N^+$ and $1 \le d \le n$.
\subsection*{Solution}
By Taylor expansion we have $$\e^d = \sum_{k = 0}^{\infty}\frac{d^k}{k!} > \sum_{k=0}^d\frac{d^k}{k!}$$ where immediately follows the proof 
\begin{equation}
	\begin{split}
		\left(\frac{\e n}{d}\right)^d &> n^d\sum_{k=0}^{d}\frac{d^{k-d}}{k!}\\
		&\ge n^d\sum_{k=0}^{d}\frac{n^{k-d}}{k!}\\
		&= \sum_{k=0}^{d}\frac{n^k}{k!}\\
		&\ge \sum_{k=0}^{d}\frac{n^{\underline k}}{k!}\\
		&= \sum_{k=0}^d\binom nk
	\end{split}
\end{equation}

\section{}
\subsection*{Statement}
Prove that the following two programming problems are equivalent. For simplicity, we assume that the first problem has non-trivial solution, i.e. there exists $t > 0$ satisfying $y_i(\mathbf w^{\text T}\mathbf x_i + b) \ge t \ (\forall i \in [n])$.\\

\begin{center}
\begin{tabular}{cc}
	$\max\limits_{\mathbf w, b, t}$ & $t$\\
	s.t. & $y_i(\mathbf w^{\text T}\mathbf x_i + b) \ge t \ (\forall i \in [n])$\\
	& $\|\mathbf w\| = 1$
\end{tabular} \qquad
\begin{tabular}{cc}
	$\min\limits_{\mathbf w, b, t}$ & $\frac12 \|\mathbf w\|$\\
	s.t. & $y_i(\mathbf w^{\text T}\mathbf x_i + b) \ge 1 \ (\forall i \in [n])$
\end{tabular}
\end{center}

\subsection*{Solution}

Let the solution to the two problems above be $t_0$ and $\mathbf w_0$ respectively. By the assumption, we have $t_0 > 0$.

By dividing the inequality in the first problem by $t_0$ we obtain $y_i\left(\frac{\mathbf w^{\text T}}{t_0}\mathbf x_i + \frac{b}{t_0}\right) \ge 1$, so $\mathbf w' = \frac{\mathbf w}{t_0}$ is a candidate for the second problem, and thus $\|\mathbf w_0\| \le \|\mathbf w'\| = \frac{1}{t_0}$.

By dividing the inequality in the second problem by $\|w_0\|$ we obtain $y_i \left(\frac{\mathbf w_0}{\|\mathbf w_0\|}\mathbf x_i + \frac{b}{\|\mathbf w_0\|}\right) \ge \frac{1}{\|\mathbf w_0\|}$, thus $t \ge \frac{1}{\|\mathbf w_0\|}$ since $\left\|\frac{\mathbf w_0}{\|\mathbf w_0\|}\right\| = 1$.

So we know that $\|\mathbf w_0\|t_0 = 1$, which shows the equivalence relationship between these two programming problems.
\end{document}


