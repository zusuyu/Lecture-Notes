\documentclass[8pt]{article}
\usepackage[UTF8]{ctex}
\usepackage[a4paper]{geometry}

\usepackage{amsthm,amsmath,amssymb}
\usepackage{graphicx}
\usepackage{subfigure}
\usepackage{amsmath}
\usepackage{tabularx}
\usepackage{color}
\usepackage{hyperref}
\usepackage{ulem}
\usepackage{multirow}
\usepackage[cache=false]{minted}
\hypersetup{
	colorlinks=true,
	linkcolor=blue
}

\usepackage{appendix}
\geometry{a4paper,centering,scale=0.8}
\geometry{left=2.0cm, right=2.0cm, top=2.5cm, bottom=2.5cm}
\usepackage[format=hang,font=small,textfont=it]{caption}
\usepackage[nottoc]{tocbibind}

\usepackage{algorithm}
\usepackage{algorithmicx}
\usepackage{algpseudocode}
\usepackage{amssymb}
\usepackage{extarrows}
\usepackage{qcircuit}
\usepackage{fancyhdr}
\usepackage{cleveref}

\usepackage{tikz}  
\usetikzlibrary{arrows.meta}%画箭头用的包

\makeatletter
\def\@maketitle{%
	\newpage
	\begin{center}%
		\let \footnote \thanks
		{\LARGE \@title \par}%
		\vskip 1.5em%
		{\large
			\lineskip .5em%
			\begin{tabular}[t]{c}%
				\@author
			\end{tabular}\par}%
		\vskip 1em%
		{\large \@date}%
	\end{center}%
	\par
	\vskip 1.5em}
\makeatother

\newtheoremstyle{compact}%
{3pt}{3pt}%
{}{}%
{\bfseries}{\textcolor{red}{.}}%  % Note that final punctuation is omitted.
{.5em}{\mbox{\textcolor{red}{\thmname{#1}\thmnumber{ #2}}\thmnote{ (\textcolor{blue}{#3})}}}
\theoremstyle{compact}
\newtheorem{innercustomgeneric}{\customgenericname}
\providecommand{\customgenericname}{}
\newcommand{\newcustomtheorem}[2]{%
	\newenvironment{#1}[1]
	{%
		\renewcommand\customgenericname{#2}%
		\renewcommand\theinnercustomgeneric{##1}%
		\innercustomgeneric
	}
	{\endinnercustomgeneric}
}

\DeclareMathOperator{\card}{card}

\newtheorem{theorem}{定理}
\newtheorem{lemma}{Lemma}
\newtheorem{definition}{定义}
\newtheorem{proposition}{命题}
\newtheorem{corollary}{推论}
\newtheorem{example}{例}
\newtheorem{claim}{声明}
\newtheorem{remark}{注}
\newtheorem{thesis}{论点}
\newtheorem{Proof}{证明}

\def\obj#1{\textbf{\uline{#1}}}
\def\num#1{\textnormal{\textbf{\mbox{\textcolor{blue}{(#1)}}}}}
\def\le{\leqslant}
\def\ge{\geqslant}
\def\im{\text{im }}
\def\P#1{\mathbb{P}\left({#1}\right)}
\def\e{\mathrm{e}}
\def\E#1{\mathbb{E}\left[{#1}\right]}
\def\Var#1{\text{Var}\left[{#1}\right]}


\title{\heiti\zihao{2} Machine Learning Homework: Week 7 \& 8}
\author{\kaishu\zihao{-3} 周书予\\2000013060@stu.pku.edu.cn}

\CTEXoptions[today=old]
\date{\today}

\usepackage{totpages}
\begin{document}
\fancypagestyle{plain}{
	\fancyhf{}
	\rhead{Homework: Week 7 \& 8}
	\chead{2022 Spring}
	\lhead{Machine Learning}
	\cfoot{Page \thepage \ of \pageref{TotPages}}
}
\pagestyle{plain}
\crefname{theorem}{定理}{定理}
\crefname{lemma}{Lemma}{Lemma}
\crefname{figure}{图}{图}
\crefname{table}{表}{表}	
\maketitle

\section{}
\subsection*{Statement}
Prove the following lemma.
\begin{lemma}
	For any $\delta > 0$,$$\mathbb P_{S \sim D^n} \left(\mathbb E_{h \sim \mathcal P} [ \e^{n(err_D(h) - err_S(h))^2} ] \ge 3/\delta\right) \le \delta$$
	\label{PAC-Bayesian-lem-2}
\end{lemma}
\subsection*{Solution}
	First we prove that for some fixed $h \sim \mathcal P$, $$\mathbb E_{S \sim D^n}[\e^{n(err_D(h) - err_S(h))^2}] \le 3$$

	Denote $|err_D(h) - err_S(h)|$ by $\Delta(h)$, by applying Chernoff bound, $$\mathbb P_{S \sim D^n}(\Delta(h) \ge \varepsilon) \le 2\exp(-2n\varepsilon^2)$$

	Thus\begin{equation*}
		\begin{split}			
			\mathbb E_{S \sim D^n} \left[\e^{n\Delta(h)^2}\right] &= \int_0^{+\infty}\mathbb P_{S \sim D^n}\left(\e^{n\Delta(h)^2} \ge t\right) \text dt\\
			&= \int_1^{+\infty}\mathbb P_{S \sim D^n}\left(\Delta(h) \ge \sqrt{\frac{\ln t}{n}}\right) \text dt + 1\\
			&\le \int_1^{+\infty}2\e^{-2\ln t}\text dt + 1\\
			&= 3
		\end{split}
	\end{equation*}

	Then by applying Markov Inequality we get\begin{equation*}
		\begin{split}
			\mathbb P_{S \sim D^n} \left(\mathbb E_{h \sim \mathcal P} [ \e^{n\Delta(h)^2} ] \ge 3/\delta\right) &\le \frac{\mathbb E_{S \sim D^n} \left(\mathbb E_{h \sim \mathcal P} [ \e^{n\Delta(h)^2} ]\right)}{3/\delta} = \frac{ \mathbb E_{h \sim \mathcal P}\left(\mathbb E_{S \sim D^n} [ \e^{n\Delta(h)^2} ]\right)}{3/\delta} \le \frac{ \mathbb E_{h \sim \mathcal P}\left(3\right)}{3/\delta} = \delta
		\end{split}
	\end{equation*}
\section{}
\subsection*{Statement}
Determine $D_{KL}(\mathcal Q \| \mathcal P)$ for $\mathcal P = \mathcal N(\mathbf 0, I_d)$ and $\mathcal Q = \mathcal N(\mu\mathbf w, I_d)$, where $\mathbf w \in \mathbb R^d$ satisfies $\|\mathbf w\|^2 = 1$, $\mu$ stands for the scale factor.
\subsection*{Solution}
\begin{equation*}
	\begin{split}
		D_{KL}(\mathcal Q \| \mathcal P)
		&= \int_{\mathbb R^d}\frac{1}{(2\pi)^{d/2}}\exp\left[-\frac12\|\mathbf x - \mu\mathbf w\|^2\right]\frac12\left(\|\mathbf x\|^2 - \|\mathbf x - \mu\mathbf w\|^2\right)\text d\mathbf x\\
		&= \int_{\lambda}\int_{\mathbf y \in \mathbb R^{d-1}, \mathbf y \perp \mathbf w}\frac{1}{(2\pi)^{d/2}}\exp\left[-\frac12\|\lambda\mathbf w + \mathbf y - \mu\mathbf w\|^2\right]\frac12\left(\|\lambda\mathbf w + \mathbf y\|^2 - \|\lambda\mathbf w + \mathbf y - \mu\mathbf w\|^2\right)\text d\lambda\text d\mathbf y\\
		&= \int_{\lambda}\int_{\mathbf y \in \mathbb R^{d-1}, \mathbf y \perp \mathbf w}\frac{1}{(2\pi)^{d/2}}\exp\left[-\frac12(\lambda - \mu)^2 - \frac12 \|\mathbf y\|^2\right]\frac12\left(\lambda^2 + \|\mathbf y\|^2 - (\lambda - \mu)^2 - \|\mathbf y\|^2\right)\text d\lambda\text d\mathbf y\\
		&= \int_{-\infty}^{+\infty}\frac{1}{\sqrt{2\pi}}\exp\left[-\frac12(\lambda - \mu)^2\right]\frac12(2\lambda\mu - \mu^2)\text d\lambda\left[\int_{\mathbf y \in \mathbb R^{d-1}, \mathbf y \perp \mathbf w}\frac{1}{(2\pi)^{(d-1)/2}}\exp\left(-\frac12 \|\mathbf y\|^2\right)\text d\mathbf y\right]\\
		&= \int_{-\infty}^{+\infty}\frac{1}{\sqrt{2\pi}}\exp\left[-\frac12(\lambda - \mu)^2\right](\lambda\mu - \mu^2)\text d\lambda + \frac{\mu^2}{2}\\
		&=\int_{-\infty}^{+\infty}\frac{1}{\sqrt{2\pi}}\exp\left[-\frac12(\lambda - \mu)^2\right]\mu\text d\frac{(\lambda - \mu)^2}{2} + \frac{\mu^2}{2}\\
		&= \frac{\mu^2}{2}
	\end{split}
\end{equation*}

\end{document}


