\documentclass[8pt]{article}
\usepackage[UTF8]{ctex}
\usepackage[a4paper]{geometry}

\usepackage{amsthm,amsmath,amssymb}
\usepackage{graphicx}
\usepackage{subfigure}
\usepackage{amsmath}
\usepackage{tabularx}
\usepackage{color}
\usepackage{hyperref}
\usepackage{ulem}
\usepackage{multirow}
\usepackage[cache=false]{minted}
\hypersetup{
	colorlinks=true,
	linkcolor=blue
}

\usepackage{appendix}
\geometry{a4paper,centering,scale=0.8}
\geometry{left=2.0cm, right=2.0cm, top=2.5cm, bottom=2.5cm}
\usepackage[format=hang,font=small,textfont=it]{caption}
\usepackage[nottoc]{tocbibind}

\usepackage{algorithm}
\usepackage{algorithmicx}
\usepackage{algpseudocode}
\usepackage{amssymb}
\usepackage{extarrows}
\usepackage{qcircuit}
\usepackage{fancyhdr}
\usepackage{cleveref}


\usepackage{tikz}  
\usetikzlibrary{arrows.meta}%画箭头用的包

\makeatletter
\def\@maketitle{%
	\newpage
	\begin{center}%
		\let \footnote \thanks
		{\LARGE \@title \par}%
		\vskip 1.5em%
		{\large
			\lineskip .5em%
			\begin{tabular}[t]{c}%
				\@author
			\end{tabular}\par}%
		\vskip 1em%
		{\large \@date}%
	\end{center}%
	\par
	\vskip 1.5em}
\makeatother

\newtheoremstyle{compact}%
{3pt}{3pt}%
{}{}%
{\bfseries}{\textcolor{red}{.}}%  % Note that final punctuation is omitted.
{.5em}{\mbox{\textcolor{red}{\thmname{#1}\thmnumber{ #2}}\thmnote{ (\textcolor{blue}{#3})}}}
\theoremstyle{compact}
\newtheorem{innercustomgeneric}{\customgenericname}
\providecommand{\customgenericname}{}
\newcommand{\newcustomtheorem}[2]{%
	\newenvironment{#1}[1]
	{%
		\renewcommand\customgenericname{#2}%
		\renewcommand\theinnercustomgeneric{##1}%
		\innercustomgeneric
	}
	{\endinnercustomgeneric}
}

\DeclareMathOperator{\card}{card}

\newtheorem{theorem}{定理}
\newtheorem{lemma}{引理}
\newtheorem{definition}{定义}
\newtheorem{proposition}{命题}
\newtheorem{corollary}{推论}
\newtheorem{example}{例}
\newtheorem{claim}{声明}
\newtheorem{remark}{注}
\newtheorem{thesis}{论点}
\newtheorem{Proof}{证明}

\def\obj#1{\textbf{\uline{#1}}}
\def\num#1{\textnormal{\textbf{\mbox{\textcolor{blue}{(#1)}}}}}
\def\le{\leqslant}
\def\ge{\geqslant}
\def\im{\text{im }}
\def\P#1{\mathbb{P}\left[{#1}\right]}
\def\e{\mathrm{e}}
\def\E#1{\mathbb{E}\left[{#1}\right]}
\def\Var#1{\text{Var}\left[{#1}\right]}


\title{\heiti\zihao{1} Machine Learning Homework: Week 1}
\author{\kaishu\zihao{-3} 周书予\\2000013060@stu.pku.edu.cn}

\CTEXoptions[today=old]

\date{\today}

\begin{document}
\pagestyle{fancy}
\lhead{Machine Learning}
\chead{2022 Spring}
\rhead{机器学习}


\crefname{theorem}{定理}{定理}
\crefname{lemma}{引理}{引理}
\crefname{figure}{图}{图}
\crefname{table}{表}{表}	
\maketitle

\section{}
\subsection*{Statement}

	Random variable $X \sim \mathcal N(0, 1)$, let $f(x) = \P{X \ge x}$, find an elementary function $g(x)$ satisfying $\lim\limits_{x \to \infty} \frac{f(x)}{g(x)} = 1$. Furthermore, find such elementary functions $g_{u}(x)$ and $g_{l}(x)$ satisfying $g_l(x) \le f(x) \le g_u(x)$ for $\forall t \in \mathbb R^+$.

\subsection*{Solution}

	$$g_l(x) = \left( \frac{1}{x} - \frac{1}{x^3} \right)\frac{\e^{-x^2/2}}{\sqrt{2\pi}}, \qquad g_u(x) = \frac{\e^{-x^2/2}}{\sqrt{2\pi}x}.$$

	We first prove that $g_l(x)$ and $g_u(x)$ are the lowerbound and upperbound of $f(x)$ respectively. Notice that $$f(x) = \int_{x}^{+\infty}\frac{\e^{-t^2/2}}{\sqrt{2\pi}}\text dt$$ by the technique of \obj{integration by parts}, we have

	\begin{align}
		\sqrt{2\pi}f(x) = \int_{x}^{+\infty} \frac{1}{t} \text d\left( -\e^{-t^2/2} \right) = -\frac{\e^{-t^2/2}}{t}\bigg|_{x}^{+\infty} - \int_{x}^{+\infty}\frac{\e^{-t^2/2}}{t^2}\text dt \le \frac{\e^{-x^2/2}}{x} = \sqrt{2\pi} g_u(x)
	\end{align}

	Furthermore, applying such technique to the remaining part above gives

	\begin{align}
		\int_{x}^{+\infty}\frac{\e^{-t^2/2}}{t^2}\text dt = \int_x^{+\infty}\frac{1}{t^3}\text d \left( -\e^{-t^2/2} \right) = -\frac{\e^{-t^2/2}}{t^3}\bigg|_{x}^{+\infty} - \int_{x}^{+\infty}\frac{3\e^{-t^2/2}}{t^4}\text dt \le \frac{\e^{-x^2/2}}{x^3}
	\end{align}
	and thus
	\begin{align}
		\sqrt{2\pi}f(x) \ge \left( \frac{1}{x} - \frac{1}{x^3} \right)\e^{-x^2/2} = \sqrt{2\pi} g_l(x)
	\end{align}
	
	$\lim\limits_{x \to +\infty}\frac{f(x)}{g_u(x)} = 1$ and $\lim\limits_{x \to +\infty}\frac{f(x)}{g_l(x)} = 1$ can be easily proved using \obj{L'Hôpital's rule}:

	\begin{align}
		\lim_{x \to +\infty}\frac{f(x)}{g_u(x)} = \lim_{x \to +\infty}\frac{f'(x)}{g_u'(x)} = \lim_{x \to +\infty} \frac{-\e^{-x^2/2}}{-\left(1 + \frac{1}{x^2}\right)\e^{-x^2/2}} = 1\\
		\lim_{x \to +\infty}\frac{f(x)}{g_l(x)} = \lim_{x \to +\infty}\frac{f'(x)}{g_l'(x)} = \lim_{x \to +\infty} \frac{-\e^{-x^2/2}}{-\left(1 + \frac{3}{x^4}\right)\e^{-x^2/2}} = 1
	\end{align}
\end{document}
