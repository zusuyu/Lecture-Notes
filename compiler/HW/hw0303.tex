\documentclass[8pt]{article}
\usepackage[UTF8]{ctex}
\usepackage[a4paper]{geometry}

\usepackage{amsthm,amsmath,amssymb}
\usepackage{graphicx}
\usepackage{subfigure}
\usepackage{amsmath}
\usepackage{tabularx}
\usepackage{color}
\usepackage{hyperref}
\usepackage{ulem}
\usepackage{multirow}
\usepackage[cache=false]{minted}
\hypersetup{
	colorlinks=true,
	linkcolor=blue
}

\usepackage{appendix}
\geometry{a4paper,centering,scale=0.8}
\geometry{left=2.0cm, right=2.0cm, top=2.5cm, bottom=2.5cm}
\usepackage[format=hang,font=small,textfont=it]{caption}
\usepackage[nottoc]{tocbibind}

\usepackage{algorithm}
\usepackage{algorithmicx}
\usepackage{algpseudocode}
\usepackage{amssymb}
\usepackage{extarrows}
\usepackage{qcircuit}
\usepackage{fancyhdr}
\usepackage{cleveref}
\usepackage{totpages}
\usepackage{pgf}
\usepackage{tikz}
\usepackage{bbm}
\usetikzlibrary{arrows,automata}
\usetikzlibrary{arrows.meta}%画箭头用的包

\makeatletter
\def\@maketitle{%
	\newpage
	\begin{center}%
		\let \footnote \thanks
		{\LARGE \@title \par}%
		\vskip 1.5em%
		{\large
			\lineskip .5em%
			\begin{tabular}[t]{c}%
				\@author
			\end{tabular}\par}%
		\vskip 1em%
		{\large \@date}%
	\end{center}%
	\par
	\vskip 1.5em}
\makeatother

\newtheoremstyle{compact}%
{3pt}{3pt}%
{}{}%
{\bfseries}{\textcolor{red}{.}}%  % Note that final punctuation is omitted.
{.5em}{\mbox{\textcolor{red}{\thmname{#1}\thmnumber{ #2}}\thmnote{ (\textcolor{blue}{#3})}}}
\theoremstyle{compact}
\newtheorem{innercustomgeneric}{\customgenericname}
\providecommand{\customgenericname}{}
\newcommand{\newcustomtheorem}[2]{%
	\newenvironment{#1}[1]
	{%
		\renewcommand\customgenericname{#2}%
		\renewcommand\theinnercustomgeneric{##1}%
		\innercustomgeneric
	}
	{\endinnercustomgeneric}
}

\DeclareMathOperator{\card}{card}

\newtheorem{theorem}{定理}
\newtheorem{lemma}{引理}
\newtheorem{definition}{定义}
\newtheorem{proposition}{命题}
\newtheorem{corollary}{推论}
\newtheorem{remark}{注}
\newtheorem{Proof}{证明}

\def\obj#1{\textbf{\uline{#1}}}
\def\num#1{\textnormal{\textbf{\mbox{\textcolor{blue}{(#1)}}}}}
\def\le{\leqslant}
\def\ge{\geqslant}
\def\im{\text{im }}
\def\Pr#1{\text{Pr}\left[{#1}\right]}
\def\E#1{\mathbb{E}\left[{#1}\right]}
\def\Var#1{\text{Var}\left[{#1}\right]}
\def\Enc{\textsf{Enc}}
\def\Dec{\textsf{Dec}}
\def\Gen{\textsf{Gen}}
\def\rep#1{\llcorner{#1}\lrcorner}

\title{\heiti\zihao{1} 编译原理 \ 第一次作业}
\author{\kaishu\zihao{-3} 周书予\\2000013060@stu.pku.edu.cn}

\CTEXoptions[today=old]
\date{\today}

\begin{document}\large
\fancypagestyle{plain}{
	\fancyhf{}
	\lhead{编译原理}
	\chead{2023 Spring}
	\rhead{第一次作业}
	\cfoot{第 \thepage 页, 共 \pageref{TotPages} 页}
}
\pagestyle{plain}



\crefname{theorem}{定理}{定理}
\crefname{lemma}{引理}{引理}
\crefname{figure}{图}{图}
\crefname{table}{表}{表}	
\maketitle

\section*{Ex. 3.3.2}
正则表达式定义的语言分别由满足如下条件的字符串构成:
\begin{enumerate}
	\item \textbf{a(a|b)*a}: 仅包含字符 $a, b$, 长度至少为 $2$ 且以字符 $a$ 开头结尾的串.
	\item \textbf{(($\varepsilon$|a)b*)*}: 仅包含字符 $a, b$ 且不存在相邻的字符 $a$ 的串.
	\item \textbf{(a|b)*a(a|b)(a|b)}: 仅包含字符 $a, b$, 长度至少为 $3$ 且倒数第 $3$ 个字符为 $a$ 的串.
	\item \textbf{a*ba*ba*ba*}: 仅包含字符 $a, b$ 且有恰好 $3$ 个字符 $b$ 的串.
	\item \textbf{(aa|bb)*((ab|ba)(aa|bb)*(ab|ba)(aa|bb)*)*}: 仅包含字符 $a, b$ 且字符 $a, b$ 出现次数均为偶数的串.
\end{enumerate}

\section*{Ex. 3.3.3}
在一个长度为 $n$ 的字符串中, 
\begin{itemize}
	\item 前缀有 $n + 1$ 个.
	\item 后缀有 $n + 1$ 个.
	\item 真前缀有 $n - 1$ 个.
	\item 子串有至少 $n + 1$, 至多 $\frac{n(n+1)}{2} + 1$ 个.
	\item 子序列有至少 $n + 1$, 至多 $2^n$ 个.
\end{itemize}

\section*{Ex. 3.3.5}
\begin{enumerate}
	\item \textbf{[\^{}aeiou]*a[\^{}eiou]*e[\^{}aiou]*i[\^{}aeou]*o[\^{}aeiu]*u[\^{}aeio]*}
	\item \textbf{a*b*c*d*e*f*g*h*i*j*k*l*m*n*o*p*q*r*s*t*u*v*w*x*y*z*}
	\item \textbf{($\backslash$/$\backslash$*) ([\^{}*"] | $\backslash$".*$\backslash$" | $\backslash$*+[\^{}/])* ($\backslash$*$\backslash$/)}
	\item \textbf{a(aa|bb)*(ab|ba)A|bA}, where \textbf{A$\to$(aa|bb)*((ab|ba)(aa|bb)*(ab|ba)(aa|bb)*)*}
	\item \textbf{b*a*b?a*}
\end{enumerate}

\end{document}
