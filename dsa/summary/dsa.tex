\documentclass[8pt]{article}
\usepackage[UTF8]{ctex}
\usepackage[a4paper]{geometry}

\usepackage{graphicx}
\usepackage{subfigure}
\usepackage{amsmath}
\usepackage{tabularx}
\usepackage{color}
\usepackage{hyperref}
\usepackage{ulem}
\usepackage{multirow}
\usepackage[cache=false]{minted}
\hypersetup{
	colorlinks=true,
	linkcolor=blue
}

\usepackage{appendix}
\geometry{a4paper,centering,scale=0.8}
\geometry{left=2.0cm, right=2.0cm, top=2.5cm, bottom=2.5cm}
\usepackage[format=hang,font=small,textfont=it]{caption}
\usepackage[nottoc]{tocbibind}

\usepackage{algorithm}
\usepackage{algorithmicx}
\usepackage{algpseudocode}
\usepackage{amssymb}
\usepackage{qcircuit}
\usepackage{fancyhdr}
\usepackage{cleveref}


\usepackage{tikz}  
\usetikzlibrary{arrows.meta}%画箭头用的包

\makeatletter
\def\@maketitle{%
	\newpage
	\begin{center}%
		\let \footnote \thanks
		{\LARGE \@title \par}%
		\vskip 1.5em%
		{\large
			\lineskip .5em%
			\begin{tabular}[t]{c}%
				\@author
			\end{tabular}\par}%
		\vskip 1em%
		{\large \@date}%
	\end{center}%
	\par
	\vskip 1.5em}
\makeatother

\newtheorem{theorem}{定理}
\newtheorem{lemma}{引理}
\newtheorem{definition}{定义}
\newtheorem{proposition}{命题}
\newtheorem{corollary}{推论}


\title{\heiti\zihao{1} 数据结构与算法A}
\author{\kaishu\zihao{-3} 周书予\\2000013060@stu.pku.edu.cn}

\date{\today}

\begin{document}\small
\pagestyle{fancy}
\lhead{阿拉丁}
\chead{}
\rhead{阿拉丁}


\crefname{theorem}{定理}{定理}
\crefname{lemma}{引理}{引理}
\crefname{figure}{图}{图}
\crefname{table}{表}{表}	
\maketitle

\section{图论}

好像没有什么阿拉丁玩意儿。

拓扑排序、最短路、最小生成树总还是会的吧

\section{内排序}

\subsection{直接插入排序}
就是维护一个已排好序的前缀,每轮把这段前缀的后面一个数加进去。注意一开始默认第一个数是有序的,因此$k$轮后已排序前缀的长度是$k+1$。

\subsection{Shell排序}

取$D = \{d_1, d_2, \cdots, d_m\}$,第$k$轮以$d_k$的间隔把整个序列分成$d_k$个子序列,每个长度(差不多)为$n / d_k$的子序列内部做插入排序排成有序。

\subsection{直接选择排序}
仍然是维护一个已排好序的前缀,每轮在未排序的部分找到一个最小的数放在这段前缀的后面。

直接选择排序的交换次数不超过$n-1$。同时比较次数不依赖于序列的初始状态。


\subsection{堆排序}

注意建堆的时间复杂度是$\Theta(n)$。

\subsection{冒泡排序}

每轮冒泡排序结束后,一个数后面比自己小的数一定会恰好减少一个。所以冒泡排序需要的轮数就是这个东西的最大值$+1$(最后一轮发现不需要再排了)。

冒泡排序也维护了已排好的前缀,而且也是每次长度加一。

\subsection{快速排序}

先把轴值(pivot)交换到最后边,然后取出来(相当于把最右边当成空位),$l, r$指针每次把左边大于轴值的和右边小于轴值的元素放到空位上(并产生新的空位)。

\subsection{归并排序}

\subsection{桶排序与基数排序}
\subsection{索引排序}
索引$1$下标$\mathrm{IndexArray1}$就是$\mathrm{Rank}$,索引$2$下标$\mathrm{IndexArray2}$就是$\mathrm{SA}$。两者互为逆置换。

\subsection{排序算法的稳定性}
直接插入排序、冒泡排序、归并排序、桶排序和基数排序是稳定的。

直接选择排序、Shell排序、快速排序和堆排序是不稳定的。

(上述)只有直接插入排序和冒泡排序在最好情况下有更低量级的复杂度,其他算法的复杂度即使在最好情况下也不变。

Shell排序和堆排序不适合链表,因为访问不连续。其他排序算法都是可以访问连续的。

\section{文件管理与外排序}
\subsection{置换选择排序}
维护一个内存大小为$M$的堆,需要持续从输入buffer中读数据,并往输出buffer中写一个递增的序列。如果某次读到一个数比已经输出过的数要小,那么这个数就寄了(不可能再被输出了),也就相当于堆大小减一。

平均意义下可以输出长度为$2M$的递增序列,称为\textbf{顺串}。

\subsection{多路归并}
如果要归并$k$个顺串,那就建一棵大小为$k$的线段树,在每个非叶子节点上写子树里最优的结果来自于那个顺串$(\in [1, k])$。这棵树叫做赢者树。

把线段树节点上记录的最优改成记录“哪个串在这个节点发生的比较上寄了”,这样每个$[1, k]$都会在树上出现恰好一次——冠军的话,会在线段树的根上面再长出一个节点来记录。

\subsection{读盘和写盘次数}
就是每次参与归并的总长度。

\section{检索}

\subsection{平均检索长度}

需要注意检索失败时检索长度被认为是$n+1$。以$p$的概率成功顺序检索的平均检索长度是$p(\frac{n+1}{2}) + (1-p)(n+1) = \frac{n+1}{2}(1 - \frac p2)$。

\subsection{二分查找某个节点的比较次数}

就是二叉搜索树上节点的深度。

\subsection{散列冲突解决方法}

有开散列方法和闭散列方法。

开散列方法就是Hash挂链。

闭散列就是冲突/碰撞时找别的位置。具体的,对于每个关键码$K$,构造一列散列地址序列$d_0, d_1, d_2, \cdots$,其中$d_0 = h(K), d_i = (d_0 + p(K, i)) \bmod M$。
\begin{itemize}
	\item 线性探测:$p(K, i) = i$,可改进为$p(K, i) = ci$
	\item 二次探测:$p(K, 2i-i) = i^2, p(K, 2i) = -i^2$
	\item 随机探测:$p(K, i) = \sigma_i$其中$\sigma$是一个$[1, M-1]$的排列
	\item 双散列探测:$p(K, i) = ih_2(K)$,其中$h_2$是另一个散列函数
\end{itemize}
\subsection{散列表的删除}
注意,散列表删除是不能直接删掉元素的。这是因为在闭散列表中,直接删除一个元素会导致其他元素的探测路径上出现空位,导致查询时出错。正确的做法是留下一个墓碑表示被删除,但仍然占据这个地址。

插入的时候可以插在墓碑上,但需要扫描整个探测序列直到发现空地(为了避免重复插入)
\section{索引技术}
\subsection{倒排表}
就是把某一项特征写在第一列。
\subsection{静态索引和动态索引}
就是支不支持动态维护。接下来的B树和B+树都属于动态索引。
\subsection{B树}
$m$阶B树满足
\begin{itemize}
	\item 每个节点不超过$m$个子节点
	\item 非根非叶子节点至少有$\lceil m/2 \rceil$个子节点
	\item 根至少有两个子节点,除非平凡的情况
	\item 所有叶子节点在同一层,有$\lceil m/2 \rceil - 1$到$m - 1$个关键码
\end{itemize}

一个节点有$k$个子节点,说明有$k-1$个关键码

一棵$m$阶深度为$k$(第$0$层到第$k-1$层)的B树, 键值(key)数$N$至少有?

考虑键值数会恰好等于最后一层连出去的空指针数量减一,也就是“理论上的第$k$层”的节点数,是$2\lceil \frac m2 \rceil^{k-1}$,所以$$N + 1 \ge 2\lceil \frac m2 \rceil^{k-1}, k \le 1 + \log_{\lceil m / 2 \rceil}\frac{N+1}{2}$$
\subsubsection{插入}
插入时一定会找到一个对应的叶子节点,插入后该叶子节点关键码数量小于$m$则直接结束;

否则产生上溢出,需要分裂成两个分别有$\lceil (m-1)/2 \rceil$和$\lfloor (m-1)/2 \rfloor$个关键码的叶子,多出来一个关键码丢给父亲。

父亲可能会继续上溢出,持续这个过程,如果分裂根,则将导致树的深度加一。

访外次数:向下找的读盘次数(=深度) + 1 + 2 $\times$ 分裂次数

\subsubsection{删除}

如果待删除的节点不在叶子层,就与后继交换。后继一定在叶子层。

直接删除后叶子节点关键码数量可能少于$\lceil m/2 \rceil - 1$,产生下溢出。先找兄弟借,如果借不了说明邻居都恰好是$\lceil m/2 \rceil - 1$个关键码,此时考虑和兄弟以及父亲处的分界关键码合并。

合并后相当于吃掉了父亲节点的一个关键码,于是父亲节点重复这个过程。如果根仅有的两个子节点合并了,那么树的深度减一。

最多访外次数:向下找的读盘次数(=深度) + 访问左右兄弟并在失败后合并 $\times$ (深度 - 1) + 根节点重写 = $4h - 2$

\subsection{B+树}

相比于B树中父节点维护的是子节点分界的关键码,B+树中维护子节点的最大值,也即B+树叶子节点层包含了维护的所有关键码。

不过注意这时候一个有$k$个关键码的节点对应就有$k$个子节点,因此每个节点的关键码数量为$\lceil m/2 \rceil$到$m$。

插入和删除是差不多的

\subsection{红黑树}

\begin{itemize}
	\item 红黑树每个节点要么是红的,要么是黑的。根节点一定是黑的。
	\item 外部节点被认为是黑的。
	\item 根到所有外部节点路径上的黑点数量相同。
	\item 不存在相邻的红色节点。
\end{itemize}

一个节点的rank是从这个节点出发到子树内一个外部节点路径上的黑色节点数量,不算自身。整棵树的rank是根节点的rank。

阶为$n$的红黑树,树高最小是$n+1$,最大是$2n+1$;内部节点最少时是一棵完全满二叉树,内部节点数为$2^n-1$

含有$m$个内部节点的红黑树树高最大是$2\log_2(m+1)+1$

\subsubsection{插入}
按照正常BST插入,插入后默认红色,如果父亲是黑色则直接结束。如果父亲是红色,那么祖父一定是黑色

此时讨论叔叔的颜色:如果叔叔是黑色,那么就转一下,就能结束了;

如果叔叔的红色,那么就把父亲、叔叔和祖父都反色,相当于把祖父的黑色下放。此时祖父可能产生新的红红冲突,需要继续向上调整。特别地,如果把根的颜色改成了红色,那就再改成黑色,此时树的rank加一

\subsubsection{删除}

如果两个子节点都是内部节点,就和自己的后继交换。此时待删除节点一定有至少一个外部子节点

如果有恰好一个外部子节点,那么待删除节点一定是黑色,子节点一定是红色,此时只需要把子节点提上来改成黑色即可。

如果两个都是外部子节点,那么红色就直接删,黑色的话就留一个“双黑”标记。

接下来考虑怎么解决“双黑”。讨论“双黑”的兄弟节点及其子节点

\begin{itemize}
	\item 兄弟黑,且有红色子节点——把父亲、兄弟、兄弟的红儿子三个点摊平,染上恰当的颜色后可以直接结束
	\item 兄弟黑,且儿子全黑——把“双黑”和兄弟的一个黑色上传给父亲,此时父亲可能变成新的“双黑”,递归处理
	\item 兄弟红——此时父亲一定黑,对兄弟做一次单旋,这样“双黑”的父亲就变成红色,回到前两种情况
\end{itemize}

\section{高级数据结构}

\subsection{广义表}

表就是直观意义上的表,类似于文件系统

\begin{center}
	线性表 $\subseteq$ 纯表(树) $\subseteq$ 可重入表(DAG) $\subseteq$ 循环表(图)
\end{center}

$C = (c, (d, A), B, c)$是一张广义表,则$head(C)$表示第一个元素,$tail(C)$表示去掉第一个元素后得到的子表。

\subsection{Trie \& Patricia}

Patricia就是压缩后的二进制Trie

\subsection{最佳二叉搜索树}

内部节点频率$\{p_1, p_2, \cdots, p_n\}$,外部节点频率$\{q_0, q_1, \cdots, q_n\}$。

动态规划。$f(i, j)$表示$\{q_i, p_{i+1}, q_{i+1}, \cdot, p_j, q_j\}$这些节点安排好的最小代价,转移$$f_{i,j} = W(i, j) + \min_{k=i+1^{j}}(f(i, k - 1) + f(k, j))$$

ASL = Average Search Length 平均检索长度,就是$\sum p_i c_i$,$p$是概率,$c$个某个元素的检索长度。

\subsection{AVL树}

任意一个节点,其左右子树高度差不超过$1$

插入可能导致不平衡,从下往上处理:LL不平衡或者RR不平衡就做一次单旋,LR,RL需要做两次(或者说需要把三个点摊平)

\subsection{splay}

\end{document}
