
\documentclass{beamer}
\usepackage{ctex, hyperref}
\setCJKmainfont[ItalicFont=Noto Sans CJK SC Bold, BoldFont=Noto Sans CJK SC Black]{Noto Sans CJK SC}
\setCJKsansfont[ItalicFont=Noto Serif CJK SC Bold, BoldFont=Noto Serif CJK SC Black]{Noto Serif CJK SC}
\usepackage[T1]{fontenc}

% other packages
\usepackage{latexsym,amsmath,xcolor,multicol,booktabs,calligra}
\usepackage{graphicx,pstricks,listings,stackengine}
\usepackage{color, cleveref}
\usepackage{graphicx}
\usepackage{subfigure}
\usepackage{amsmath}
\usepackage{tabularx}
\usepackage{color}
\usepackage{hyperref}
\usepackage{ulem}
\usepackage{multirow}
\usepackage{algorithm}
\usepackage{algorithmicx}
\usepackage[noend]{algpseudocode}
\usepackage{amssymb}
\usepackage{qcircuit}
\usepackage{fancyhdr}

\usepackage{bibentry}
\hypersetup{
	colorlinks=false,
	linkcolor=none
}
\useoutertheme{infolines}


\author{陈威宇 \ 黄兆鋆 \ 周书予}
\title{《\textsf{PRIMES} is in \textbf{P}》 论文精讲}
\institute{信息科学技术学院}
\CTEXoptions[today=old]
\date{\today}
\usepackage{PekingU}

% defs
\def\P#1{\mathbb{P}\left({#1}\right)}

\def\cmd#1{\texttt{\color{red}\footnotesize $\backslash$#1}}
\def\env#1{\texttt{\color{blue}\footnotesize #1}}
\definecolor{deepblue}{rgb}{0,0,0.5}
\definecolor{deepred}{rgb}{0.6,0,0}
\definecolor{deepgreen}{rgb}{0,0.5,0}
\definecolor{halfgray}{gray}{0.55}

\definecolor{uibred}  {HTML}{db3f3d}
\definecolor{uibblue} {HTML}{4ea0b7}
\definecolor{uibgreen}{HTML}{789a5b}
\definecolor{uibgray} {HTML}{d0cac2}
\definecolor{uiblink} {HTML}{00769E}
\setbeamercolor{block body} {bg = uibgray}
\setbeamercolor{block title}{fg = white,bg = uibred}

\def\le{\leqslant}
\def\ge{\geqslant}
\def\ord{\textrm{ord}}


\lstset{
    basicstyle=\ttfamily\small,
    keywordstyle=\bfseries\color{deepblue},
    emphstyle=\ttfamily\color{deepred},    % Custom highlighting style
    stringstyle=\color{deepgreen},
    numbers=left,
    numberstyle=\small\color{halfgray},
    rulesepcolor=\color{red!20!green!20!blue!20},
    frame=shadowbox,
}

\usefonttheme[onlymath]{serif}

\begin{document}

%\kaishu
\begin{frame}
    \titlepage
    \begin{figure}[htpb]
        \begin{center}
            \includegraphics[width=1.5cm]{cwy.jpg} \qquad
            \includegraphics[width=1.5cm]{hzj.jpg} \qquad
            \includegraphics[width=1.5cm]{zsy.jpg}
        \end{center}
    \end{figure}
\end{frame}

\begin{frame}
    \tableofcontents[sectionstyle=show,subsectionstyle=show/shaded/hide,subsubsectionstyle=show/shaded/hide]
\end{frame}

\section{Intro: PRIMES in Complexity Theory}
\begin{frame}{Introduction}
	$$\mathsf{PRIMES} = \{w | w \text{ is binary representation of a prime number}\}$$

	~\\

	\textsf{PRIMES} 属于以下复杂度类:
	\begin{itemize}
		\item \textbf{EXP}, 比如众所周知的有 \textit{Sieve of Eratosthenes}.
		\item \textbf{coNP}, 因为 $\overline{\textsf{PRIMES}} = \textsf{COMPOSITES} \in \textbf{NP}$.
		\item \textbf{NP}, \cite{Pratt75}.
		\item \textbf{BPP}, \cite{AB99}.
		\item \textbf{P}, \cite{AKS}.
	\end{itemize}

\end{frame}

\subsection{PRIMES is in NP}
\begin{frame}{\cite{Pratt75}: Every Prime Has a Succinct Certificate}
	\begin{block}{Lucas Primality Test}
		正整数 $n \ge 3$ 是素数当且仅当存在整数 $1 < a < n$满足 $a^{n-1} \equiv 1 \mod 1$ 且对于 $n-1$ 的每个素因子 $p$ 都有 $a^{(n-1)/p} \not\equiv 1 \bmod n$.
	\end{block}
	\begin{proof}
		$\Leftarrow$: 考虑$\ord_{n}(a)$. $\Rightarrow$: 取 $n$ 的原根作为 $a$.
	\end{proof}
	
	想验证 $p$ 是素数, 只需要给出满足条件的 $a$ 以及 $p - 1$ 的素因数分解 $p - 1 = \prod\limits_{i=1}^{k}p_i$, \textbf{同时}进一步验证 $p_i$ 都是素数.  

	可以归纳证明验证 $p$ 时所需的 certificate 长度为 $O(\textrm{poly}(\log p))$ , 从而说明了 $\textsf{PRIMES} \in \textbf{NP}$.	
	
\end{frame}

\subsection{PRIMES is in BPP}
\begin{frame}{Generalization of Fermat's Little Theorem}
	\begin{block}{Theorem}
		正整数 $n \ge 2$ 是素数当且仅当存在与 $n$ 互素的整数 $a$, 满足 $$(x + a)^n \equiv x^n + a \mod n$$
	\end{block}

	上述定理中的等式常被称为 freshman's dream.

	\begin{proof}
		$n$ 是素数 $\Rightarrow$ $n | \binom{n}{k} \ (\forall 1 \le k < n)$, $(x + a)^n - x^n - a^n$ 系数全为 $0$.

		$n$ 不是素数 $\Rightarrow$ 若 $p^k \| n$, 则 $p^k \nmid \binom{n}{p} = \frac{n \times (n-1) \times \cdots \times (n-p+1)}{1 \times 2 \times \cdots \times p}$, $p$次项系数非零.
	\end{proof}
	
\end{frame}

\begin{frame}{\cite{AB99}: \textsf{PRIMES} is in \textbf{BPP}}
	前述结论用于素数测试的一大障碍在于需要计算的多项式次数太大了.\\~\\
	
	所以可以考虑随机选取一个模数多项式 $r(x) \in \mathbb Z[x]$, 然后检验$(x + a)^n \equiv x^n + a \bmod (r(x), n)$ 是否成立.\\~\\

	\begin{block}{Theorem}
		设$n$ 是一个含有素因子 $p \ge 17$ 的非素数, 且不是某个素数的整数次幂.
		
		在$\mathbb Z[x]$中随机选取一个 $l = \lceil \log n \rceil$ 次首一多项式 $r(x)$, 则有至少 $60\%$ 的概率, $(x + 1)^n \not\equiv x^n + 1 \bmod (r(x), n)$.
	\end{block}
	
\end{frame}
\begin{frame}{\cite{AB99}: \textsf{PRIMES} is in \textbf{BPP}}
	令 $P_n(x) = (x + 1)^n - 1 - x^n$. 当 $n$ 不是素数时, $P_n(x) \not\equiv 0 \bmod p$, 这是因为$[x^{p^k}] P_n(x) = \binom{n}{p^k} = \frac{n \times (n-1) \times \cdots \times (n-p^k+1)}{(p^k)!} \not\equiv 0 \bmod p$, 其中 $p^k \| n$, 这里要求了 $n \neq p^k$.\\~\\

	于是我们在 $\mathbb Z_p[x]$ 下考虑问题. 注意到 $\mathbb Z_p[x]$ 下存在唯一分解, 不可约因式的概念, 记 $I(d)$ 表示其中 $d$ 次不可约首一多项式的数量, 可以估计其下界\footnote{\tiny 这个结论不太初等, 就不在此过多阐述了, 可详细参考 \cite{lidl_niederreiter_1994}.}

	$$ I(d) = \frac{1}{d}\sum_{e | d}\mu(e)p^{d/e} \ge \frac{1}{d}\left( p^d -  \sum_{e < d, e | d}p^{d/e}\right) \ge \frac{p^d}{d} - p^{d/2} $$
\end{frame}
\begin{frame}{\cite{AB99}: \textsf{PRIMES} is in \textbf{BPP}}
	记 $\mathcal C$ 表示 $\mathbb Z_p[x]$ 中含有 $> \frac{l}{2}$ 次不可约因式的$l$次首一多项式集合. 
	
	同样对 $|\mathcal C|$ 估计下界

	$$|\mathcal C| = \sum_{k = \lfloor \frac l2 \rfloor + 1}^{l}I(k)p^{l-k} \ge \sum_{k = \lfloor \frac l2 \rfloor + 1}^{l}p^l\left(\frac 1k - \frac{1}{p^{k/2}}\right) \ge \left(\ln 2 - \frac{1}{48}\right)p^l$$

	(其中利用了 $p \ge 17$.)

	注意一个 $l$ 次多项式中不可能包含超过一个$> \frac{l}{2}$ 次不可约因式.
\end{frame}
\begin{frame}{\cite{AB99}: \textsf{PRIMES} is in \textbf{BPP}}
	因为 $\deg P_n(x) < n$,  $P_n(x)$ 只有不超过 $\frac{2n}{l}$ 个 $>\frac{l}{2}$次不可约因式, 从而 $\mathcal C$ 中只有不超过 $\frac{2n}{l} \cdot p^{\frac l2 - 1}$ 个多项式可能是 $P_n(x)$ 的因式.

	\begin{align*}
		\P{r \textrm{ doesn't divide } P_n} &\ge \P{r \in \mathcal C \wedge r \textrm{ doesn't divide } P_n}\\
		&= \P{r \in \mathcal C} - \P{r \in \mathcal C \wedge r \textrm{ divides } P_n}\\
		&\ge \left(\ln 2 - \frac{1}{48}\right) - \frac{2n}{l} \cdot p^{\frac l2 - 1} / p^l\\
		& \ge \ln 2 - \frac{1}{48} - \frac{2n}{l4^{\frac l2 + 1}} > \frac 35
	\end{align*}

	(注意到 $n > p \ge 17, l = \lceil \log n \rceil \ge 5$.)
\end{frame}

\section{Notations \& Preliminaries}
\begin{frame}{Notations \& Preliminaries}
	在这一部分我们引入一些记号.\\~\\

	当 $\gcd(a, r) = 1$ 时, 满足 $a^k \equiv 1 \bmod r$ 的正整数 $k$ 是一定存在的, 称其中最小的为 $a$ 模 $r$ 的阶, 记作$\ord_{r}(a)$. 用欧拉函数 $\varphi(r)$ 表示比 $r$ 小的与 $r$ 互素的数的数量, 容易验证 $\ord_r(a) | \varphi(r)$.\\~\\

	当 $p$ 是素数, $h(x)$是$\mathbb Z_p[x]$中不可约多项式时, $\mathbb Z_p[x] / h(x)$ 是有限域. 之后我们会用记号 $f(x) \equiv g(x) \bmod (h(x), p)$ 来表示 $f(x)$ 和 $g(x)$ 在 $\mathbb F \triangleq \mathbb Z_p[x] / h(x)$ 中相等.\\~\\

	$\tilde O(f(n)) = O(f(n) \textrm{ poly}(\log f(n))) = O(f^{1 + \varepsilon}(n)) $ 表示忽略对数因子.

\end{frame}
\section{AKS Primality Test}
\subsection{Algorithm}
\begin{frame}{Description}
	算法分为如下五个步骤: 
	\begin{enumerate}
		\item 第一步检查 $n$ 是否是“整数幂”形式的非素数.
		\item 第二步求出最小的\footnote{\tiny 我们会在下一个 section 中证明这样的 $r$ 的存在性并给出上界.}满足 $\gcd(n, r) = 1$ 且 $\ord_r(n) > \log^2n$ 的正整数 $r$.
		\item 第三步检查 $n$ 是否有不超过 $r$ 的非平凡因子.
		\item 第四步是平凡的: 如果 $n \le r$ 没有不超过 $r$ 的非平凡因子, 此时 $n$ 一定是素数.
		\item 第五步对于 $a = 1, \cdots, \lfloor\sqrt{\varphi(r)}\log n\rfloor$, 检查$(x + a)^n \equiv x^n + a \bmod (x^r - 1, n)$是否成立, 若存在不满足则 $n$ 不是素数, \textit{否则 $n$ 是素数}.
	\end{enumerate}

	只有最后一句话的正确性是非平凡的.
\end{frame}
\begin{frame}{Pseodocode}
	\begin{algorithmic}[1]
		\Function{AKS\_PRIMALITY\_TESTING}{$n$}
			\If{$n = a^b$ for $a \in \mathbb N$ and $b > 1$}
				\State \Return \texttt{COMPOSITE}
			\EndIf
			\State $r \gets \min\{m \in \mathbb N | \gcd(n, m) = 1 \wedge \ord_m(n) > \log^2n\}$
			\If{$1 < \gcd(a, n) < n$ for some $a \le r$}
				\State \Return \texttt{COMPOSITE}
			\EndIf
			\If{$n \le r$}
				\State \Return \texttt{PRIME}
			\EndIf
			\For{$a = 1 \to \lfloor \sqrt{\varphi(r) \log n}\rfloor$}
				\If{$(x + a)^n \not\equiv x^n + a \bmod (x^r - 1, n)$}
					\State \Return \texttt{COMPOSITE}
				\EndIf
			\EndFor
			\State \Return \texttt{PRIME}
		\EndFunction
	\end{algorithmic}

\end{frame}
\subsection{Correctness}
\begin{frame}
	正确性证明中, 非平凡的部分只有“证明通过第五步测试的 $n$ 是素数”. \\~\\

	任取这样的 $n$ 的一个满足 $\ord_r(p) > 1$ 的素因子 $p$, 注意到有 $p > r$.\\~\\

	根据有限域上分圆多项式那套理论, $x^r - 1$ 在 $\mathbb Z_p$ 下有一个 $\ord_r(p)$ 次不可约多项式因子 $h(x)$. 以下记 $\mathbb F = \mathbb Z_p[x] / h(x)$. 事实上我们没有显式地给出 $p$ 和 $h(x)$ 的具体值, 只是考虑在 $\bmod (x^r - 1, n)$ 下做运算, 是因为 $f(x) \equiv g(x) \bmod (x^r-1, n) \Rightarrow f(x) \equiv g(x) \bmod (h(x), p)$.\\~\\

	前述算法中我们只限制了 $\ord_r(n) \ (> \log^2n)$, 而没有限制 $\ord_r(p)$. 原作者在 2002 年的预印版论文\footnote{\tiny \href{https://www.cse.iitk.ac.in/users/manindra/algebra/primality_original.pdf}{\color{red}这里有下载链接}}中使用了一些解析数论手段来限制 $\ord_r(p)$, 但后来 Hendrik Lenstra Jr. 的工作简化了证明.

	\begin{center}
		{\zihao{-5}\kaishu 谢谢你, Hendrik Lenstra Jr. !}		
	\end{center}
\end{frame}
\begin{frame}{Introspectivity: Generalization of Freshman's Dream}
	对于 $a = 0, 1, \cdots, l \triangleq \lfloor \sqrt{\varphi(r)} \log n \rfloor$, $(x + a)^m \equiv x^m + a \bmod (x^r-1, p)$ 对于 $m = n$ (由第五步验证)和 $m = p$ (由扩展费马小)均成立.\\~\\
	
	我们指出它对于 $m = \frac np$ 也成立, 这是因为
	$$(x^p + a)^{\frac np} \equiv [(x + a)^p]^{\frac np} \equiv (x + a)^n \equiv x^n + a \equiv (x^p)^{\frac np} + a$$

	用 $x$ 替代\footnote{\tiny 为什么可以替代呢?} $x^p$ 即可. 我们可以从中抽象出这样的概念:

	\begin{definition}[内省]
		如果 $m \in \mathbb N$ 和多项式 $f \in \mathbb Z[x]$ 满足 $f(x^m) \equiv f^m(x) \bmod (x^r-1, p)$, 则称 $m$ 对 $f$ 内省(introspective).
	\end{definition}
	惊喜地发现内省性对数乘和多项式乘都是封闭的.
\end{frame}
\begin{frame}{Introspectivity: Generalization of Freshman's Dream}
	
	\begin{lemma}[数乘封闭性]
		如果 $m_1$ 和 $m_2$ 都对 $f$ 内省, 则 $m_1m_2$ 对 $f$ 内省.
	\end{lemma}
	\begin{proof}
		$f(x^{m_1m_2}) \equiv [f(x^{m_1})]^{m_2} \equiv [f^{m_1}(x)]^{m_2} \equiv f^{m_1m_2}(x)$, 其中第一个 $\equiv$ 在 $\bmod (x^{m_1r} - 1, p)$ 下成立, 但这比在 $\bmod (x^{r} - 1, p)$ 下要强.
	\end{proof}
	\pause
	\begin{lemma}[多项式乘封闭性]
		如果 $m$ 对 $f_1$ 和 $f_2$ 都内省, 则 $m$ 对 $f_1f_2$ 内省.
	\end{lemma}
	\begin{proof}
		$[f_1(x)f_2(x)]^m \equiv f_1^m(x)f_2^m(x) \equiv f_1(x^m)f_2(x^m) \equiv f_1f_2(x^m)$
	\end{proof}
\end{frame}
\begin{frame}{Introspectivity: Generalization of Freshman's Dream}
	由于 $p, \frac np$ 对 $x, x + 1, \cdots, x + l$ 分别内省, 两边分别作为生成元得到集合 $I = \left\{p^i \left(\frac np\right)^j \bigg| i, j \le 0\right\}$ 和 $P = \left\{\prod\limits_{a=0}^{l}(x + a)^{e_a} \bigg| e_a \ge 0\right\}$ 也两两内省. \\~\\

	因为模 $x^r - 1$ 下 $x^{kr + b}$ 和 $x^b$ 是一样的, 考虑对 $I$ 中元素模 $r$ 得到 $\mathbb Z_r^*$ 的子群 $G$. 记 $|G| = t$, 注意到 $\langle n \rangle \le G$, 故 $t \ge \ord_r(n) > \log^2n$.\\~\\

	类似的, 把 $P$ 中元素映到 $\mathbb F$ 上得到 $\mathbb F$ 的乘法子群 $\mathcal G$. \\~\\

	接下来的工作就是给出 $\mathcal G$ 大小的上下界, 并从中归谬得出 “$n$ 是素数”的结论.
\end{frame}
\begin{frame}{Lowerbound of $|\mathcal G|$}
	\begin{lemma}[Hendrik Lenstra Jr.]
		$|\mathcal G| \ge \binom{t + l}{t - 1}$.
	\end{lemma}
	考虑证明 $P$ 中任意两个不同的小于 $t$ 次多项式映到 $\mathbb F$ 上仍不同. 反证, 假设 $f, g \in P$ 映到 $\mathbb F$ 上相同, 那么根据内省性, $d \triangleq f - g$ 代入所有 $x^m$ 后映到 $\mathbb F$ 上都是 $\mathbf{0}$, 其中 $m \in G, x^m \in \mathbb F$. 这说明如果我们把 $d$ 看成 $\mathbb F[y]$ 的元素, $d$ 在 $\mathbb F$ 上就有至少 $t$ 个根\footnote{\tiny 这里有赖于 $x^m$ 在 $\mathbb F$ 中两两不同, 即所谓“$x$ 是 $\mathbb F$的 $r$ 次本原根”.}, 而 $\deg (d) < t$ 产生了矛盾. 

	$P$ 中小于 $t$ 次多项式个数是 $\binom{t + l}{t - 1}$, 这是不定方程 $\sum\limits_{a=0}^{l}e_a < t$ 的解数\footnote{\tiny 这里还需要说明 $x, \cdots, x + l$ 映到 $\mathbb F$ 中互不相同.}.\\~\\

	{\kaishu\zihao{-5} 在 Hendrik Lenstra Jr. 之前, 一个平凡的结论是 $|\mathcal G| \ge \binom{\ord_r(p) + l}{\ord_r(p) - 1}$.}

\end{frame}
\begin{frame}{Upperbound of $|\mathcal G|$}
	\begin{lemma}
		如果 $n$ 不是 $p$ 的整数次幂, 那么 $|\mathcal G| \le n^{\lfloor\sqrt{t}\rfloor}$.
	\end{lemma}
	考虑集合 $\hat{I} = \left\{p^i\left(\frac np\right)^j \bigg| 0 \le i, j \le \lfloor \sqrt {t}\rfloor \right\}$ , 如果 $n$ 不是 $p$ 的整数次幂, 那么$\hat{I}$ 中有 $(\lfloor\sqrt t\rfloor + 1)^2 > t$ 个互异元素, 且一定存在两个数之差是 $r$ 的倍数.
	
	不妨设为 $m_1, m_2 \in \hat{I}$, 满足 $x^{m_1} \equiv x^{m_2} \bmod (x^r-1, p)$. 任取 $f \in \mathcal G$, 由于 $f^{m_1}(x) \equiv f(x^{m_1}) \equiv f(x^{m_2}) \equiv f^{m_2}(x)$, 故 $Q(Y) = Y^{m_1} - Y^{m_2} \in \mathbb F[Y]$ 以 $\mathcal G$ 中任意元素为根, 从而 $|\mathcal G| \le \deg Q = m_1 \le (p \cdot \frac np)^{\lfloor \sqrt {t}\rfloor} = n^{\lfloor \sqrt {t}\rfloor}$.

\end{frame}
\begin{frame}{Sufficiency for AKS Prime Testing}
	\begin{align*}
		|\mathcal G| &\ge \binom{t + l}{t - 1} = \binom{t + l}{l + 1} & \\
		&\ge \binom{\lfloor \sqrt{t}\log n\rfloor + 1 + l}{l + 1} & \textrm{since} \ t > \log^2n \Rightarrow t \ge \lfloor \sqrt{t}\log n\rfloor\\
		&= \binom{\lfloor \sqrt{t}\log n\rfloor + 1 + l}{\lfloor \sqrt{t}\log n\rfloor} & \\
		&\ge \binom{2\lfloor \sqrt{t}\log n\rfloor + 1 }{\lfloor \sqrt{t}\log n\rfloor} & \textrm{since} \ G \le \mathbb Z_r^* \Rightarrow t \le \varphi(r) \\
		&>2^{\lfloor \sqrt{t}\log n\rfloor}+1 & \textrm{since} \ \binom{2k+1}{k} > 2^{k+1} \ \textrm{for} \ k > 1\\
		&\ge n^{\lfloor \sqrt t \rfloor}&
	\end{align*}
\end{frame}
\begin{frame}{Sufficiency for AKS Prime Testing}
	上下界的矛盾导致了 $n$ 只能是 $p$ 的整数次幂, 而由第一步, 这个幂次不能大于 $1$, 所以 $n = p$ 是素数.
\end{frame}
\subsection{Complexity Analysis}
\begin{frame}{Existence \& Upperbound of $r$}
	\begin{lemma}
		存在 $r \le \max\{3, \lceil \log^5n + \log n \rceil\}$ 满足 $\ord_r(n) > \log^2n$.
	\end{lemma}
	$n = 2$ 的情况是平凡的, 取 $r = 3$ 即可.

	对于 $n > 2$, 反证, 假设该范围内与 $n$ 互素的所有数都满足 $\ord_r(n) \le \log^2n$, 那么它们都将整除 $\prod\limits_{i=1}^{\lfloor\log^2n\rfloor}(n^i-1) < n^{\log^4n} = 2^{\log^5n}$

	加上那些与 $n$ 不互素的, 我们知道了不超过该范围的所有数都整除 $n\prod\limits_{i=1}^{\lfloor\log^2n\rfloor}(n^i-1) < 2^{\lceil\log^5n+\log n \rceil}$,  即它们的 $\textrm{LCM}$ 不会超过这个数 

	但是 $\lceil \log^5n + \log n \rceil \ge 7$ 而 $\textrm{LCM}(m) \ge 2^m (\forall m \ge 7)$, 所以矛盾了.
\end{frame}
\begin{frame}{$\textrm{LCM}(m) > 2^m ?$}
	\begin{lemma}
		对于正整数 $m \ge 7$, 有$$\textrm{LCM}(m) \triangleq \textrm{lcm}(1, 2, \cdots, m) > 2^m$$
	\end{lemma}

	证明...\sout{我不好说了.}

	\cite{farhi2009identity} 解析地给出了一个稍弱的结论, 可供参考.
\end{frame}
\begin{frame}{Complexity Analysis}
	两个 $m$ 比特数的四则运算复杂度为 $\tilde O(m)$.

	两个以 $m$ 比特数为系数的不超过 $d$ 次多项式乘法的复杂度为 $\tilde O(dm)$.

	\begin{theorem}
		算法的渐进时间复杂度为 $\tilde O(\log^{10.5}n)$.
	\end{theorem}
\end{frame}
\begin{frame}{Complexity Analysis}
	\begin{itemize}
		\item 第一步的复杂度是 $\tilde O(\log^3n)$ , 因为如果 $n = a^b$ , 那么 $b \le \log n$ , 可以先枚举 $b$ 再二分求 $a$, 注意到一共有 $O(\log^2n)$ 对有序对 $(a, b)$ 需要计算 $a^b$, 而单次计算的复杂度是 $\tilde O(\log n)$ (对 $b$ 快速幂) , 所以总复杂度是 $\tilde O(\log^3n)$. \pause
		\item 第二步需要从小到大枚举 $r$ 并检验, 每个 $r$ 的检验需要做 $O(\log^2n)$ 次乘法, 不过由于是 $\bmod r$ 所以乘法的复杂度是 $O(\log^{\varepsilon} n)$ 可以忽略, 总复杂度是 $\tilde O(r\log^2n) = \tilde O(\log^7n)$.
	\end{itemize}
\end{frame}
\begin{frame}{Complexity Analysis}
	\begin{itemize}
		\item 第三步需要算 $r$ 次 $\gcd$ , 每次都是 $n$ 和一个 $\le r$ 的数, 因为瓶颈只在于第一次用 $n$ 去试除 $r$, 那么一次的复杂度是 $\tilde O(\log n)$ , 总复杂度是 $\tilde O(r\log n) = \tilde O(\log^6n)$. \pause
		\item 第五步需要进行 $\lfloor\sqrt{\varphi(r)}\log n\rfloor$ 轮检验, 每轮检验需要做 $O(\log n)$ 次以 $O(\log n)$ 比特数为系数的$r$次多项式乘法, 复杂度是 $\tilde O(r\log^2n)$ , 检验轮数是 $O(r^{0.5}\log n)$ 所以总复杂度是 $\tilde O(r^{1.5}\log^{3} n) = \tilde O(\log^{10.5}n)$.
	\end{itemize}
\end{frame}
\begin{frame}{Improvements via Conjectures}
	\begin{block}{Sophie-Germain Prime Density Conjecture}
		若素数 $p$ 满足 $\frac{p-1}{2}$ 也是素数, 则称它为苏菲-热尔曼素数. 苏菲-热尔曼素数在渐进意义下有 $\frac{2C_2m}{\ln^2m}$ 个, 其中 $C_2$ 是孪生素数常数($\approx 0.66$).
	\end{block}

	可选取常数 $c_1, c_2$ 使得 $[c_1\log^2, c_2\log^2n(\log\log n)^2]$ 之间有 $O(\log^2n)$ 个苏菲-热尔曼素数. 因为 $\ord_p(n) | \varphi(p) = p - 1$, 这些苏菲-热尔曼素数要么 $\ord_{p}(n) \le 2$, 要么 $\ord_p(n) \ge \frac{p-1}{2}$. 满足前者的 $p$ 都是 $n^2 - 1$ 的约数, 不会有超过 $O(\log n)$ 个, 因此存在$r = \tilde O(\log^2n)$ 满足 $\ord_r(n) \ge \frac{r-1}{2} = O(\log^2n)$.
\end{frame}
\begin{frame}{Improvements via Conjectures (proved)}
	\begin{lemma}[Fouvry, \cite{Fouvry1985}]
		记 $P(n)$ 表示 $n$ 的最大素因子. 满足$P(q - 1) > q^{\frac23}$ 的素数 $p$ 在渐进意义下有 $c\frac{m}{\ln m}$ 个, 其中 $c > 0$ 是常数.
	\end{lemma}

	类似前面的分析方式, 我们可以给出 $r = O(\log^3n)$ 的上界, 从而把算法的复杂度优化至 $\tilde O(\log^{7.5}n)$.
\end{frame}
\subsection{Future Work}
\begin{frame}{}
	在算法的第五步中, 我们取了 $l = \lfloor \sqrt{\varphi(r)}\log n \rfloor$ 以保证 $|\mathcal G|$ 足够大. 能不能减小一点呢?

	如果以下猜想成立, 复杂度就可以被优化至 $\tilde O(\log^3n)$.

	\begin{block}{Conjecture}
		如果 $r$ 是与 $n$ 互素的素数, 且满足 $(x - 1)^n \equiv x^n - 1 \bmod (x^r-1, n)$, 则要么 $n$ 是素数, 要么 $n^2 \equiv 1 \bmod r$.
	\end{block}

	做法是找到一个不整除 $n^2 - 1$ 的素数 $r$, 并判定 $(x - 1)^n \equiv x^n - 1 \bmod (x^r-1, n)$ 是否成立, 若成立则根据猜想 $n$ 是素数, 否则 $n$ 一定不是素数.

	上述要求的 $r$ 是一定存在, 且不超过 $2\ln n$ 的, 这是因为根据\cite{APO}, 不超过 $x$ 的素数乘积至少是 $\mathrm{e}^x$, 故不超过 $2 \ln n$ 的素数乘积至少是 $\mathrm{e}^{2 \ln n} = n^2$, 它们不可能都是 $n^2 - 1$ 的因子.

	\cite{KS} 验证了该猜想在 $r \le 100, n \le 10^{10}$ 时是成立的.
\end{frame}
\section{Acknowledgments \& References}
\begin{frame}[fragile]{Acknowledgments}
	参考了如下文章.
	\begin{itemize}
		\item \href{https://guyutongxue.github.io/blogs/primes_is_in_p.html}{\color{red} 多项式复杂度内的素性检测算法
		PRIMES is in P - guyutongxue}
		\item \href{https://www.bananaspace.org/wiki/%E7%94%A8%E6%88%B7:Rushcheyo/%E3%80%8A%E7%AE%97%E6%B3%95%E5%88%86%E6%9E%90%E4%B8%8E%E8%AE%BE%E8%AE%A1%E3%80%8B%E8%AE%BA%E6%96%87%E9%98%85%E8%AF%BB/PRIMES_is_in_P}{\color{red}《算法分析与设计》论文阅读/PRIMES is in P - Rushcheyo}
	\end{itemize}
\end{frame}
\begin{frame}[allowframebreaks]{References}
	\bibliographystyle{amsalpha}
	\bibliography{ref}
\end{frame}
\begin{frame}
	Q. \& A.
\end{frame}	

\end{document}
