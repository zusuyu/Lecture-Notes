\documentclass[8pt]{article}
\usepackage[UTF8]{ctex}
\usepackage[a4paper]{geometry}

\usepackage{graphicx}
\usepackage{subfigure}
\usepackage{amsmath}
\usepackage{tabularx}
\usepackage{color}
\usepackage{hyperref}
\usepackage{ulem}
\usepackage{multirow}
\usepackage[cache=false]{minted}
\hypersetup{
	colorlinks=true,
	linkcolor=blue
}

\usepackage{appendix}
\geometry{a4paper,centering,scale=0.8}
\geometry{left=2.0cm, right=2.0cm, top=2.5cm, bottom=2.5cm}
\usepackage[format=hang,font=small,textfont=it]{caption}
\usepackage[nottoc]{tocbibind}

\usepackage{algorithm}
\usepackage{algorithmicx}
\usepackage[noend]{algpseudocode}
\usepackage{amssymb}
\usepackage{qcircuit}
\usepackage{fancyhdr}
\usepackage{cleveref}


\usepackage{tikz}  
\usetikzlibrary{arrows.meta}%画箭头用的包

\makeatletter
\def\@maketitle{%
	\newpage
	\begin{center}%
		\let \footnote \thanks
		{\LARGE \@title \par}%
		\vskip 1.5em%
		{\large
			\lineskip .5em%
			\begin{tabular}[t]{c}%
				\@author
			\end{tabular}\par}%
		\vskip 1em%
		{\large \@date}%
	\end{center}%
	\par
	\vskip 1.5em}
\makeatother

\newtheorem{theorem}{定理}
\newtheorem{lemma}{引理}
\newtheorem{definition}{定义}
\newtheorem{proposition}{命题}
\newtheorem{corollary}{推论}

\def\obj#1{\textbf{\uline{#1}}}
\def\num#1{\textnormal{\textbf{\mbox{\textcolor{blue}{(#1)}}}}}
\def\le{\leqslant}
\def\ge{\geqslant}

\title{\heiti\zihao{1} 算分第三次作业}
\author{\kaishu\zihao{-3} 周书予\\2000013060@stu.pku.edu.cn}

\CTEXoptions[today=old]
\date{\today}

\begin{document}
\pagestyle{fancy}
\lhead{算法设计与分析}
\chead{2022 Spring}
\rhead{Algorithm Design and Analysis}


\crefname{theorem}{定理}{定理}
\crefname{lemma}{引理}{引理}
\crefname{figure}{图}{图}
\crefname{table}{表}{表}	
\maketitle

\section*{3.12}

用 $P_{i, j} \ (i, j \in [1, n], j - i \ge 2)$ 表示由顶点 $i, i + 1, \cdots, j$ 围成的凸多边形, $F_{i, j}$ 表示对 $P_{i, j}$ 做题目中所述的划分得到的最小权值. 考虑边 $i, j$ 所在的三角形, 记剩下的一个顶点为 $k$, 则显然 $k \in [i + 1, j - 1]$. $\triangle ijk$ 将 $A_{i, j}$ 分割成了两部分, 每部分都是一个凸多边形(具体的, 分别是 $A_{i, k}$ 和 $A_{k, j}$), 从而问题可以递归为更小规模的子问题. 

$F_{i, j}$ 的递推关系为
\begin{equation}
	F_{i, j} = \begin{cases}
		\min\limits_{i < k < j} \{ d_{ij} + d_{ik} + d_{jk} + F_{i, k} + F_{k, j} \}, & j - i > 2\\
		d_{i, i+1} + d_{i, i + 2} + d_{i + 1, i + 2}, & j - i = 2
	\end{cases}
\end{equation}

原问题结果即为 $F_{1, n}$. 此算法的时间复杂度为 $O(n^3)$.

\section*{3.15}

(为了方便, 我们不妨认为该机器在第 $0$ 天与第 $n + 1$ 天均会进行检修.)

用 $F_{i}$ 表示在第 $i$ 天检修的条件下, 前 $i$ 天加工任务的最大数量. 考虑第 $i$ 天检修的前一次发生在第 $j$ 天($0 \le j < i$), 可以得到递推方程
\begin{equation}
	F_i = \begin{cases}
		\max\limits_{0 \le j < i} \left\{F_j + \sum\limits_{k=j+1}^{i-1}\min(x_k, s_{k - j})\right\}, & i \ge 1\\
		0, & i = 0
	\end{cases}
\end{equation}

原问题结果即为 $F_{n + 1}$. 朴素实现可以得到 $O(n^3)$ 的时间复杂度. 考虑记 $w_{i,j} = \sum\limits_{k=j+1}^{i-1}\min(x_k, s_{k - j})$, 显然有 $w_{j,i} = \begin{cases}
	w_{j, i - 1} + \min(x_{i-1}, s_{i-j-1}), & i - j > 1\\
	0, & i - j = 1
\end{cases}$, 因为可以通过 $O(n^2)$ 的预处理得到. 算法的时间复杂度可被优化为 $O(n^2)$.

\newpage
\section{}
\begin{center}
	\begin{algorithmic}[1]
		\Function{Knapsack}{$n, w, v, W$}
			\State $f \gets$ an array of length $W + 1$, initialized $0$
			\State $g \gets$ an array of length $W + 1$, initialized $0$
			\For{$i = 1 \to n$}
				\For{$j = w_i \to W$}
					\If{$f_j < f_{j - w_i} + v_i$}
						\State $f_j \gets f_{j - w_i} + v_i$
						\State $g_j \gets i$
					\EndIf
				\EndFor
			\EndFor
			\State $S \gets \varnothing$
			\State $i \gets W$
			\While{$g_i \neq 0$}
				\State append $i$ to $S$
				\State $i \gets g_{i - w_i}$
			\EndWhile
			\State \Return $S$
		\EndFunction
	\end{algorithmic}
\end{center}

\section{}

\subsection{description}
	给定一个DNA模式序列$S$, 对于$\forall i \in [1, |S|]$, 求出有多少长度为$m$的DNA序列$T$使$\textrm{LCS}(S, T) = i$, 其中 $\textrm{LCS}(s_1, s_2)$ 表示串 $s_1, s_2$ 的最长公共子序列. 
\subsection{solution}
	考虑 LCS 的 DP, $f_{i,j} = \max(f_{i-1,j}, f_{i, j-1}, f_{i-1, j-1} + [S_i = T_i])$. 
	
	我们用$g_{i, f_{i, 1}, f_{i, 2}, \cdots, f_{i, |S|}}$来表示DP状态(即, 确定了 $T$ 的前 $i$ 位, 已确定部分与 $S$ 的每一段前缀的 LCS 长度分别是多少), 会发现这种记录状态的方式是可行的. 
	
	注意到$f_{i, *}$的相邻位置相差至多为$1$, 所以把$f_{i, *}$差分后, 可以用一个长为$|S|$的二进制串来表示. 
	
	预处理转移, 可以做到$O((|S| + m)2^{|S|})$的复杂度. 
\subsection{为什么要选这道题}
	\sout{\kaishu 从做过的题里随便找了一道.}

	DP 的结果也可以作为另一个 DP 的状态表示?
\end{document}
