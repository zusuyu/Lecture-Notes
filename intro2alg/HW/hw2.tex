\documentclass[8pt]{article}
\usepackage[UTF8]{ctex}
\usepackage[a4paper]{geometry}

\usepackage{graphicx}
\usepackage{subfigure}
\usepackage{amsmath}
\usepackage{tabularx}
\usepackage{color}
\usepackage{hyperref}
\usepackage{ulem}
\usepackage{multirow}
\usepackage[cache=false]{minted}
\hypersetup{
	colorlinks=true,
	linkcolor=blue
}

\usepackage{appendix}
\geometry{a4paper,centering,scale=0.8}
\geometry{left=2.0cm, right=2.0cm, top=2.5cm, bottom=2.5cm}
\usepackage[format=hang,font=small,textfont=it]{caption}
\usepackage[nottoc]{tocbibind}

\usepackage{algorithm}
\usepackage{algorithmicx}
\usepackage{algpseudocode}
\usepackage{amssymb}
\usepackage{qcircuit}
\usepackage{fancyhdr}
\usepackage{cleveref}


\usepackage{tikz}  
\usetikzlibrary{arrows.meta}%画箭头用的包

\makeatletter
\def\@maketitle{%
	\newpage
	\begin{center}%
		\let \footnote \thanks
		{\LARGE \@title \par}%
		\vskip 1.5em%
		{\large
			\lineskip .5em%
			\begin{tabular}[t]{c}%
				\@author
			\end{tabular}\par}%
		\vskip 1em%
		{\large \@date}%
	\end{center}%
	\par
	\vskip 1.5em}
\makeatother

\newtheorem{theorem}{定理}
\newtheorem{lemma}{引理}
\newtheorem{definition}{定义}
\newtheorem{proposition}{命题}
\newtheorem{corollary}{推论}

\def\obj#1{\textbf{\uline{#1}}}
\def\num#1{\textnormal{\textbf{\mbox{\textcolor{blue}{(#1)}}}}}
\def\le{\leqslant}
\def\ge{\geqslant}

\title{\heiti\zihao{1} 算分第二次作业}
\author{\kaishu\zihao{-3} 周书予\\2000013060@stu.pku.edu.cn}

\CTEXoptions[today=old]
\date{\today}

\begin{document}
\pagestyle{fancy}
\lhead{算法设计与分析}
\chead{2022 Spring}
\rhead{Algorithm Design and Analysis}


\crefname{theorem}{定理}{定理}
\crefname{lemma}{引理}{引理}
\crefname{figure}{图}{图}
\crefname{table}{表}{表}	
\maketitle

\section{}

定义一个规模为 $k$ 的问题为需要覆盖大小为 $2^k \times 2^k$, 有一个特殊方格的棋盘. 当 $k = 1$ 时, 解法是平凡的. 当 $k > 1$ 时:

\begin{enumerate}
	\item 将棋盘划分成四部分, 每部分是一个大小为 $2^{k-1} \times 2^{k-1}$ 的子棋盘. 在这四部分中, 恰有一部分包含特殊方格.
	\item 在棋盘的正中央放置一个 L 型骨牌, 使得其覆盖\textbf{不包含特殊方格}的三个子棋盘各一个方格.
	\item 把被骨牌覆盖的方格也视作特殊方格, 得到了 $4$ 个规模为 $k-1$ 的子问题(需要覆盖大小为 $2^{k-1} \times 2^{k-1}$, 有一个特殊方格的棋盘), 递归处理即可.
\end{enumerate}

该算法的时间复杂度为 $T(k) = 4T(k-1) + O(1)$, 展开易知 $T(k) = O(4^k).$

\section{}
\subsection{}

假设矩阵中点上写的数是 $y$.

\begin{itemize}
	\item 若 $x = y$, 则查询到了 $x$ 所在的位置, 算法结束.
	\item 若 $x > y$, 则任意在矩阵中点左上方的数 $z$ 都满足 $z < y < x$, 因此 $x$ 不可能出现在矩阵的左上方.
	\item 若 $x < y$, 则任意在矩阵中点右下方的数 $z$ 都满足 $z > y > x$, 因此 $x$ 不可能出现在矩阵的右下方.
\end{itemize}

因此该算法不会错误地排除正确答案, 因而正确性得证.

复杂度为 $T(n) = O(1) + 3T(n/2)$, 由主定理知 $T(n) = O(n^{\log_23})$.

\subsection{}

定义 $l_i = \max\{1 \le j \le n | A_{i,j} \le x\} \ (1 \le i \le n)$, 特别的, 定义 $\max \varnothing = 0$.

\begin{itemize}
	\item 由于$A_{i, j} < A_{i,j+1}, A_{i,j} < A_{i+1, j}$, 可知 $l_i \ge l_{i + 1}$, 因此求出所有 $\{l_i\}$ 的复杂度是 $O(n)$.
	\item 如果矩阵中的第 $i$ 行存在 $x$, 则必然有 $A_{i, l_i} = x$, 故求出 $\{l_i\}$ 后便可以找到 $x$ 的出现位置.
\end{itemize}

因此可以实现复杂度为 $O(n)$ 的查找特定元素 $x$ 的算法.

\section{}

Toom-Cook 乘法的大致思路是将给定的大整数 $m$ 和 $n$ 拆分成 $k$ 部分, 对这些部分实现递归乘法计算, 再向上合并结果. 具体的, 算法分为五个步骤:

\begin{enumerate}
	\item 拆分: 选取基数 $B = 10^i$, 把两个整数 $m, n$ 拆分成 $k$ 段大小小于 $B$ 的数字, 其中 $i$ 可以取 $$\max\left\{\left\lfloor\frac{\log_{10}m}{k}\right\rfloor, \left\lfloor\frac{\log_{10}n}{k}\right\rfloor\right\} + 1$$
	假设拆分得到 $m = \sum\limits_{i=0}^{k-1}m_iB^i, n = \sum\limits_{i=0}^{k-1}n_iB^i$, 其中 $0 \le m_i, n_i < B$, 记多项式 $p(x) = \sum\limits_{i=0}^{k-1}m_ix^i, q(x) = \sum\limits_{i=0}^{k-1}n_ix_i$, 则显然 $p(B) = m, q(B) = n$.

	欲求 $m \times n$, 只需要求出 $r(x) = p(x)q(x)$ 后代入 $x = B$ 即可.

	\item 求值: 由于 $r(x)$ 是一个不超过 $2k-2$ 次多项式, 只需要 $2k-1$ 个点的点值即可唯一确定. 因此考虑选取一列点 $x_1, x_2, \cdots, x_{2k-1}$, 分别求出 $p(x_i)$ 与 $q(x_i)$. $x_i$ 通常取较小的整数, 可以简化运算.

	\item 点乘: 计算 $r(x_i) = p(x_i)q(x_i)$, 在这一步中, 需要进一步递归计算整数乘法.
	\item 插值: 利用 $r$ 的 $2k-1$ 个点值, 插值还原出 $r$ 的各项系数. 通过特定的 $x_i$ 取值, 可以使插值系数变得简单.
	\item 重代入: 在已得到的 $r(x)$ 中代入 $x = B$ 即可得到 $r(B) = mn$.
\end{enumerate}

该算法的复杂度为 $T(n) = O(c(k)n) + (2k-1)T(n/k)$, 由主定理知 $T(n) = O(c(k)n^{\log(2k-1) / \log(k)})$, 其中 $c(k)$ 是小常数加法与乘法所需要的常数时间.

例如说, 当 $k = 2$ 时, 取插值点 $x = \{0, 1, \infty\}$ (其中 $\infty$ 指的是 $\lim\limits_{x \to \infty} p(x) / x^{\deg p}$ 即 $p(x)$ 的首项系数), 于是

$$\begin{pmatrix}
	p(0) \\ p(1) \\ p(\infty)
\end{pmatrix} = 
\begin{pmatrix}
	1 & 0\\
	1 & 1\\
	0 & 1
\end{pmatrix}
\begin{pmatrix}
	m_0 \\ m_1
\end{pmatrix}$$

插值时, 需要利用系数矩阵的逆矩阵

$$\begin{pmatrix}
	r(0) \\ r(1) \\ r(\infty)
\end{pmatrix} = 
\begin{pmatrix}
	1 & 0 & 0\\
	1 & 1 & 1\\
	0 & 0 & 1
\end{pmatrix}
\begin{pmatrix}
	r_0 \\ r_1 \\ r_2
\end{pmatrix}
\Rightarrow
\begin{pmatrix}
	r_0 \\ r_1 \\ r_2
\end{pmatrix} = 
\begin{pmatrix}
	1 & 0 & 0\\
	1 & 1 & 1\\
	0 & 0 & 1
\end{pmatrix}^{-1}
\begin{pmatrix}
	r(0) \\ r(1) \\ r(\infty)
\end{pmatrix} = 
\begin{pmatrix}
	r(0) \\ r(1) - r(0) - r(\infty) \\ r(\infty)
\end{pmatrix}
$$

\end{document}
