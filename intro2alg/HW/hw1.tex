\documentclass[8pt]{article}
\usepackage[UTF8]{ctex}
\usepackage[a4paper]{geometry}

\usepackage{graphicx}
\usepackage{subfigure}
\usepackage{amsmath}
\usepackage{tabularx}
\usepackage{color}
\usepackage{hyperref}
\usepackage{ulem}
\usepackage{multirow}
\usepackage[cache=false]{minted}
\hypersetup{
	colorlinks=true,
	linkcolor=blue
}

\usepackage{appendix}
\geometry{a4paper,centering,scale=0.8}
\geometry{left=2.0cm, right=2.0cm, top=2.5cm, bottom=2.5cm}
\usepackage[format=hang,font=small,textfont=it]{caption}
\usepackage[nottoc]{tocbibind}

\usepackage{algorithm}
\usepackage{algorithmicx}
\usepackage{algpseudocode}
\usepackage{amssymb}
\usepackage{qcircuit}
\usepackage{fancyhdr}
\usepackage{cleveref}


\usepackage{tikz}  
\usetikzlibrary{arrows.meta}%画箭头用的包

\makeatletter
\def\@maketitle{%
	\newpage
	\begin{center}%
		\let \footnote \thanks
		{\LARGE \@title \par}%
		\vskip 1.5em%
		{\large
			\lineskip .5em%
			\begin{tabular}[t]{c}%
				\@author
			\end{tabular}\par}%
		\vskip 1em%
		{\large \@date}%
	\end{center}%
	\par
	\vskip 1.5em}
\makeatother

\newtheorem{theorem}{定理}
\newtheorem{lemma}{引理}
\newtheorem{definition}{定义}
\newtheorem{proposition}{命题}
\newtheorem{corollary}{推论}

\def\obj#1{\textbf{\uline{#1}}}
\def\num#1{\textnormal{\textbf{\mbox{\textcolor{blue}{(#1)}}}}}
\def\le{\leqslant}
\def\ge{\geqslant}

\title{\heiti\zihao{1} 算分第一次作业}
\author{\kaishu\zihao{-3} 周书予\\2000013060@stu.pku.edu.cn}

\CTEXoptions[today=old]
\date{\today}

\begin{document}
\pagestyle{fancy}
\lhead{算法设计与分析}
\chead{2022 Spring}
\rhead{Algorithm Design and Analysis}


\crefname{theorem}{定理}{定理}
\crefname{lemma}{引理}{引理}
\crefname{figure}{图}{图}
\crefname{table}{表}{表}	
\maketitle
\section{}
\iffalse
\subsection*{1.2}
\begin{enumerate}
	\item 第$i$轮\texttt{for}循环中需要进行$n - i$次比较, 因此最坏情况下比较次数是$\sum_{i=1}^{n-1}(n-i) = \frac{n(n-1)}{2}$.
	\item 当输入的$n$个数以严格降序排列时, 交换次数达到最大, 也为$\frac{n(n-1)}{2}$.
\end{enumerate}

\subsection*{1.6}
该算法实现了对于多项式$P(t) = \sum_{i=0}^{n}P[i]t^i$求$x$处的点值.

该算法需要$2n$次乘法与$n$次加法.
\fi

\subsection*{1.7}
\begin{enumerate}
	\item 最坏需要进行$1 + 2(n-2) = 2n-3$次比较.
	\item 考虑$S$中第$i\ (3 \le i \le n)$个数$S[i]$, 如果它是前$i$个数$S[1 \cdots i]$中最小的或者次小的, 则会贡献两次比较; 否则会贡献一次比较. 因此, 平均比较次数为$$1 + \sum_{i=3}^{n}\left(2\cdot \frac{2}{i} + 1 \cdot \frac{i-2}{i}\right) \approx n-4 + 2\ln n$$
\end{enumerate}

\iffalse
\subsection*{1.15}
\begin{enumerate}
	\item $f(n) = (n^2-n)/2, g(n) = 6n \Rightarrow g(n) = O(f(n)).$
	\item $f(n) = n + 2\sqrt n, g(n) = n^2 \Rightarrow f(n) = O(g(n)).$
	\item $f(n) = n + n\log n, g(n) = n\sqrt n \Rightarrow f(n) = O(g(n)).$
	\item $f(n) = 2\log^2n, g(n) = \log n + 1 \Rightarrow g(n) = O(f(n)).$
	\item $f(n) = \log(n!), g(n) = n^{1.05} \Rightarrow f(n) = O(g(n)).$
\end{enumerate}

\subsection*{1.18}
\begin{align*}
	n! &>_a 2^{2n} >_a n2^n >_a (\log n)^{\log n} = n^{\log \log n} >_a n^3 >_a n\log n = \Theta(\log(n!)) \\
	&>_a n = \Theta(\log 10^n) >_a 2^{\log\sqrt n} >_a 2^{\sqrt{2\log n}} >_a \log n = \Theta\left(\sum_{k=1}^{n}\frac 1k\right) >_a \log\log n
\end{align*}

其中$f(n) >_a g(n)$表示$f(n) = \omega(g(n))$.
\fi

\subsection*{1.19}
\subsubsection*{(1)}
\begin{align*}
	T(n) = T(n-1) + n^2 = T(n-2) + (n-1)^2 + n^2 = \cdots = \sum_{i=1}^{n}i^2 = \frac{n(n+1)(2n+1)}{6}
\end{align*}
\subsubsection*{(3)}
根据递归树
\begin{align*}
	T(n) = cn + \frac{3}{4}cn + \left(\frac34\right)^2cn + \cdots = \left[1 + \frac34 + \left(\frac34\right)^2 + \cdots \right]cn = \Theta(n)
\end{align*}
\subsubsection*{(5)}
使用主定理, 其中$a = 5, b = 2, f(n) = n^2\log^2n = O(n^{\log_25 - \varepsilon})$, 从而$T(n) = \Theta(n^{\log_25})$.
\subsubsection*{(7)}
\begin{align*}
	T(n) = T(n-1) + \frac1n = T(n-2) + \frac{1}{n-1} + \frac1n = \cdots = \sum_{i=1}^{n}\frac1i = \Theta(\log n)
\end{align*}

\iffalse
\subsection*{1.21}
\begin{itemize}
	\item 算法A: $T_A(n) = 5T_A(n/2) + O(n) \Rightarrow T_A(n) = \Theta(n^{\log_25})$.
	\item 算法B: $T_B(n) = 2T_B(n-1) + O(1) \Rightarrow T_B(n) = \Theta(2^n)$.
	\item 算法C: $T_C(n) = 9T_C(n/3) + O(n^3) \Rightarrow T_C(n) = \Theta(n^3)$.
\end{itemize}

$T_B(n) >_a T_C(n) >_a T_A(n)$, 算法A是时间复杂度最低的.
\fi

\section{}
\begin{itemize}
	\item $(\lg n)! = \Omega(k^{\log n}) = \Omega(n^{\log k})\ (\forall k > 0)$.
	\item $(\sqrt 2)^{\lg n} = n^{\lg \sqrt 2}$.
	\item $\lg (2^{\sqrt{2\lg n}}) = \Theta(\lg^{0.5}n), \lg(\lg^2n) = \Theta(\lg\lg n) \Rightarrow 2^{\sqrt{2\lg n}} =  \Omega(\lg^2n) $.
\end{itemize}

因此
\begin{align*}
	(\lg n)! \ge_a n^2 \ge_a (\sqrt 2)^{\lg n} \ge_a 2^{\sqrt{2\lg n}} \ge_a \lg^2n 
\end{align*}

其中$f(n) \ge_a g(n)$表示$f(n) = \Omega(g(n))$.

\section{}

设生成函数为$A(x) = \sum\limits_{n \ge 0}a_nx^n$. 

\begin{align*}
	& A(x) + 2x - 1 = 5x(A(x) - 1) - 6x^2A(x) \\ \Rightarrow & A(x) = \frac{1-7x}{6x^2-5x+1} = \frac{5}{1-2x} - \frac{4}{1-3x} = \sum_{n \ge 0}\left(5(2x)^n - 4(3x)^n\right)
\end{align*}

从而$a_n = 5 \cdot 2^n - 4 \cdot 3^n$.

\end{document}
