\documentclass[8pt]{article}
\usepackage[UTF8]{ctex}
\usepackage[a4paper]{geometry}
%\setCJKmainfont[ItalicFont=Noto Serif CJK SC Bold, BoldFont=Noto Serif CJK SC Black]{Noto Serif CJK SC}

\usepackage{amsthm,amsmath,amssymb}
\usepackage{graphicx}
\usepackage{subfigure}
\usepackage{amsmath}
\usepackage{tabularx}
\usepackage{color}
\usepackage{hyperref}
\usepackage{ulem}
\usepackage{multirow}
\usepackage[cache=false]{minted}
\hypersetup{
	colorlinks=true,
	linkcolor=blue
}

\usepackage{appendix}
\geometry{a4paper,centering,scale=0.8}
\geometry{left=2.0cm, right=2.0cm, top=2.5cm, bottom=2.5cm}
\usepackage[format=hang,font=small,textfont=it]{caption}
\usepackage[nottoc]{tocbibind}


\usepackage{algorithm}
\usepackage{algorithmicx}
\usepackage{algpseudocode}
\usepackage{amssymb}
\usepackage{extarrows}
\usepackage{qcircuit}
\usepackage{fancyhdr}
\usepackage{cleveref}
\usepackage{totpages}



\usepackage{pgf}
\usepackage{tikz}
\usetikzlibrary{arrows,automata}
\usetikzlibrary{arrows.meta}%画箭头用的包

\makeatletter
\def\@maketitle{%
	\newpage
	\begin{center}%
		\let \footnote \thanks
		{\LARGE \@title \par}%
		\vskip 1.5em%
		{\large
			\lineskip .5em%
			\begin{tabular}[t]{c}%
				\@author
			\end{tabular}\par}%
		\vskip 1em%
		{\large \@date}%
	\end{center}%
	\par
	\vskip 1.5em}
\makeatother

\newtheoremstyle{compact}%
{3pt}{3pt}%
{}{}%
{\bfseries}{\textcolor{red}{.}}%  % Note that final punctuation is omitted.
{.5em}{\mbox{\textcolor{red}{\thmname{#1}\thmnumber{ #2}}\thmnote{ (\textcolor{blue}{#3})}}}
\theoremstyle{compact}
\newtheorem{innercustomgeneric}{\customgenericname}
\providecommand{\customgenericname}{}
\newcommand{\newcustomtheorem}[2]{%
	\newenvironment{#1}[1]
	{%
		\renewcommand\customgenericname{#2}%
		\renewcommand\theinnercustomgeneric{##1}%
		\innercustomgeneric
	}
	{\endinnercustomgeneric}
}

\DeclareMathOperator{\card}{card}

\newtheorem{theorem}{定理}
\newtheorem{lemma}{引理}
\newtheorem{definition}{定义}
\newtheorem{proposition}{命题}
\newtheorem{corollary}{推论}
\newtheorem{remark}{注}
\newtheorem{Proof}{证明}

\def\obj#1{\textbf{\uline{#1}}}
\def\num#1{\textnormal{\textbf{\mbox{\textcolor{blue}{(#1)}}}}}
\def\le{\leqslant}
\def\ge{\geqslant}
\def\im{\text{im }}
\def\Pr#1{\text{Pr}\left[{#1}\right]}
\def\E#1{\mathbb{E}\left[{#1}\right]}
\def\Var#1{\text{Var}\left[{#1}\right]}
\def\rep#1{\llcorner{#1}\lrcorner}

\title{\heiti\zihao{1} 计算理论导论 \ 第三次作业}
\author{\kaishu\zihao{-3} 周书予\\2000013060@stu.pku.edu.cn}

\CTEXoptions[today=old]
\date{\today}

\begin{document}
\fancypagestyle{plain}{
	\fancyhf{}
	\lhead{计算理论导论}
	\chead{2022 Spring}
	\rhead{第三次作业}
	\cfoot{第 \thepage 页, 共 \pageref{TotPages} 页}
}
\pagestyle{plain}



\crefname{theorem}{定理}{定理}
\crefname{lemma}{引理}{引理}
\crefname{figure}{图}{图}
\crefname{table}{表}{表}	
\maketitle

\section{}
考虑构造一台多项式时间图灵机 $M$ 识别语言 \textsf{2COL}.

对于输入 $G = (V, E)$, $M$ 的工作流程是:

\begin{enumerate}
	\item 初始化队列 $Q$ 为空, 同时将 $V$ 每个点的染色标记为 $0$.
	\item 如果 $Q$ 为空, 则跳转到第 $6$ 步. 否则从 $Q$ 中取出 $(v, c)$.
	\item 如果此时 $v$ 的染色为 $3 - c$, 则拒绝输入 $G$.
	\item 如果 $v$ 的染色为 $0$, 则将 $v$ 的染色标记为 $c$, 并遍历 $v$ 在 $G$ 中的所有相邻点 $w$, 将 $(w, 3 - c)$ 加入 $Q$.
	\item 返回第 $2$ 步.
	\item 找到 $v \in V$ 满足 $v$ 的染色为 $0$, 将 $(v, 1)$ 加入 $Q$. 如果这样的 $v$ 不存在, 则接受输入 $G$.
\end{enumerate}

对于任意的输入 $G$, $M$ 都只会执行上述流程 $O(|G|)$ 步并停机, 流程中的每一步也只需要图灵机 $M$ 的 $O(|G|)$ 步计算, 因此可以说明 $L(M) \in \mathbf P$. 接下来证明 $L(M) = \textsf{2COL}$.

\begin{itemize}
	\item $\forall G \in L(M)$, $M$ 在停机前其已经构造出了一种合法的染色方案, 因此说明 $G \in \textsf{2COL}$.
	\item $\forall G \notin L(M)$, 根据 $M$ 的工作流程可知在对 $G$ 进行染色的过程中出现了颜色冲突, 由归纳可知此时 $G$ 中存在奇环, 而存在奇环的图是不存在2染色的, 因此 $G \notin \textsf{2COL}$.
\end{itemize}

综上, $\textsf{2COL} \in \mathbf P$.

\section{}
由 $L_1, L_2 \in \textbf{NP}$ 知, 存在 DTM $M_1, M_2$ 以及多项式函数 $p_1, p_2: \mathbb N \to \mathbb N$, 满足对于任意 $x \in \{0, 1\}^*$, 都有 $x \in L_1 \Leftrightarrow \exists u_1 \in \{0, 1\}^{p_1(|x|)}, M_1(x, u_1) = 1, x \in L_2 \Leftrightarrow \exists u_2 \in \{0, 1\}^{p_2(|x|)}, M_2(x, u_2) = 1$.

构造 DTM $M_3$ 满足 $M_3(x, u_1 \circ u_2) = 1$ 当且仅当 $M_1(x, u_1) = 1$ \obj{或} $M_2(x, u_2) = 1$, 从而
$$x \in L_1 \cup L_2 \Leftrightarrow \begin{matrix}
	\exists u_1 \in \{0, 1\}^{p_1(|x|)}, M_1(x, u_1) = 1\\
	\text{or}\\
	\exists u_2 \in \{0, 1\}^{p_2(|x|)}, M_2(x, u_2) = 1
\end{matrix} \Leftrightarrow \exists u_1 \circ u_2 \in \{0, 1\}^{p_1(|x|) + p_2(|x|)}, M_3(x, u_1 \circ u_2) = 1$$
故构造证明了 $L_1 \cup L_2 \in \mathbf{NP}$.

同理, 构造 DTM $M_4$ 满足 $M_4(x, u_1 \circ u_2) = 1$ 当且仅当 $M_1(x, u_1) = 1$ \obj{且} $M_2(x, u_2) = 1$, 从而
$$x \in L_1 \cap L_2 \Leftrightarrow \begin{matrix}
	\exists u_1 \in \{0, 1\}^{p_1(|x|)}, M_1(x, u_1) = 1\\
	\text{and}\\
	\exists u_2 \in \{0, 1\}^{p_2(|x|)}, M_2(x, u_2) = 1
\end{matrix} \Leftrightarrow \exists u_1 \circ u_2 \in \{0, 1\}^{p_1(|x|) + p_2(|x|)}, M_4(x, u_1 \circ u_2) = 1$$
证明了 $L_1 \cap L_2 \in \mathbf{NP}$.

\section{}
称 $G = (V_1, E_1)$ 与 $H = (V_2, E_2)$ 同构, 如果 $G$ 可以被“重标号”使得与 $H$ 相同, 亦即存在一个 $\sigma: V_1 \to V_2$ 的双射, 满足 $(u, v) \in E_1 \Leftrightarrow (\sigma(u), \sigma(v)) \in E_2$. 而这样的双射 $\sigma$ 很容易用长度为 $|V_1| \log |V_1|$ 的 $01$ 串来编码.

构造一台图灵机 $M$, 其工作流程是对于输入的 $G = (V_1, E_1), H = (V_2, E_2)$ 以及 $\sigma: V_1 \to V_2$, 检查 \num{1} $\sigma$ 是不是双射, \num{2} $(u, v) \in E_1 \Leftrightarrow (\sigma(u), \sigma(v)) \in E_2$ 是否成立, 若两项均成立则接受, 否则拒绝. 显然 $M$ 是多项式时间图灵机, 且根据定义, $\langle G, H, \sigma \rangle \in L(M)$ 当且仅当 $G$ 与 $H$ 同构, 即 $\langle G, H \rangle \in \textsf{ISO}$.

取 $p(n) = n^2$. 此时对于任意 $x \in \{0, 1\}^*, x \in \textsf{ISO}$ 当且仅当存在 $\sigma \in \{0, 1\}^{p(|x|)}, M(x, \sigma) = 1$, 故根据定义, $\textsf{ISO} \in \mathbf{NP}$.

\section{}

对于任意 $L \in \mathbf{NP}$, 考虑构造多项式时间规约 $f$ 满足 $x \in L \Leftrightarrow f(x) \in \textsf{HALT}$.

假设图灵机 $M$ 与多项式 $p$ 满足 $\forall x \in \{0, 1\}^*, x \in L \Leftrightarrow \exists u \in \{0, 1\}^{p(|x|)}, M(x, u) = 1$, 考虑构造 $f(x) = \langle M', x \rangle$, 其中 $M'$ 的工作流程是: 循环枚举所有长度为 $p(|x|)$ 的 01 串 $u$ 并运行 $M(x, u)$, 一旦 $M(x, u)$ 返回 $1$ 则停机, 否则永不停机.
\begin{itemize}
	\item 虽然 $M'$ 的运行时间可能达到 $|x|$ 的指数级别甚至更大, 但构造 $M'$ 的描述是只需要多项式时间的, 因此 $f$ 是多项式时间可计算的.
	\item 同时也不难验证 $\langle M', x \rangle \in \textsf{HALT}$ 当且仅当存在 $u \in \{0, 1\}^{p(|x|)}, M(x, u) = 1$, 即 $x \in L$. 
\end{itemize}

因此我们证明了 $L \le_p \textsf{HALT}$. 由 $L$ 的任意性知 $\textsf{HALT} \in \textbf{NP}$-hard.

$\textsf{HALT} \notin \textbf{NP}$-complete, 因为否则会导致 $\textsf{HALT} \in \textbf{NP} \subseteq \textbf{EXP}$, 即存在图灵机 $M$ 可以在 $O(2^{n^c})$ 的时间内判定停机问题, 而我们知道停机问题是不可判定的, 因此矛盾.

\section{}

对于任意 $B \in \textbf P \setminus \{\varnothing, \Sigma^*\}$, 都存在 $x, y \in \Sigma^*$ 满足 $x \in B$ 而 $y \notin B$.

显然 $B \in \textbf{NP}$. 欲证明 $B \in \textbf{NP}$-complete, 还需要考虑将任意 $\textbf{NP}$ 语言 $A$ 多项式时间规约到 $B$. 考虑映射 $f$ 满足 $f(w) = \begin{cases}
	x, & w \in A \\ y, & w \notin A
\end{cases} \ (\forall w \in \Sigma^*)$, 由 $\textbf{P} = \textbf{NP}$ 假设可知 $w \in A$ 可在多项式时间内判定, 因此 $f$ 是多项式时间可计算的, 而 $w \in A \Leftrightarrow f(w) \in B \ (\forall w \in \Sigma^*)$, 说明 $A \le_p B$, 因此 $B \in \textbf{NP}$-complete.

\textit{语言 $B$ 对应的 $x, y$ 可能并不是多项式时间可计算的, 但只需要指出其“存在性”即可, 因为在“多项式时间规约”的定义中, 只要求了映射 $f$ “存在”, 而不是某种意义上的“多项式时间可构造”. 满足条件的 $x, y$ 完全可以视作映射的 $f$ 的“硬编码”.}

\section{}
\begin{enumerate}
	\item 考虑 $\phi = \bigwedge_i\left(\bigvee_j v_{ij}\right)$ 的一个 $\neq$-assignment $x$, 对于 $\phi$ 的每个从句 $c_i = \bigvee_j v_{ij}$, 都存在 $k, k'$ 使得 $v_{ik}(x) = \text{True}, v_{ik'}(x) = \text{False}$, 这说明 $v_{ik}(\lnot x) = \text{False}, v_{ik'}(\lnot x) = \text{True}$, 从而 $\lnot x$ 也使得 $\phi$ 的每个从句中包含两种不同的真值, 这说明 $\neq$-assignment 的否定仍是 $\neq$-assignment.
	\item 考虑建立 $\textsf{3SAT}$ 到 $\neq\textsf{SAT}$ 的多项式时间规约: 考虑映射 $f$ 满足 $$f\left(\bigwedge_i v_{i1} \vee v_{i2} \vee v_{i3}\right) = \bigwedge_i(v_{i1} \vee v_{i2} \vee{z_i}) \wedge (\overline{z_i} \vee v_{i3} \vee b)$$
	\textit{我们不妨假设所有 3CNF 都形如上式左边括号里的形式, 因为如若不然, 则可以通过重复一些文字来得到一个如上格式的等价的 CNF.}

	考虑证明 $w \in \textsf{3SAT} \Leftrightarrow f(w) \in \neq\textsf{SAT}$:
	\begin{enumerate}
		\item 如果 $w = \left(\bigwedge_i v_{i1} \vee v_{i2} \vee v_{i3} \right) \in \textsf{3SAT}$, 说明存在 assignment $x$ 使得 $w(x) = \text{True}$. 在 $x$ 的基础上, 将每个 $z_i$ 赋值为 $\lnot \left(v_{i_1} \vee v_{i_2}\right)$, 将 $b$ 赋值为 False, 可以得到对 $f(w) = \left(\bigwedge_i(v_{i1} \vee v_{i2} \vee{z_i}) \wedge (\overline{z_i} \vee v_{i3} \vee b)\right)$ 的赋值 $x'$. 
		
		只要验证 $x'$ 是 $f(w)$ 的一个 $\neq$-assignment, 就能说明 $f(w) \in \neq\textsf{SAT}$: 首先由 $z_i$ 的赋值知从句 $v_{i1} \vee v_{i2} \vee{z_i}$ 中一定包含两种真值, 而要使 $\overline{z_i} \vee v_{i3} \vee b$ 只包含一种真值, 则必须有 $\overline{z_i}(x') = v_{i3}(x') = \text{False}$, 这导致了 $v_{i1}(x) = v_{i2}(x) = v_{i3}(x) = \text{False}$, 与 $w(x) = \text{True}$ 矛盾.

		\item 如果 $f(w) \in \neq\textsf{SAT}$, 说明存在一个 $\neq$-assignment $y$. 讨论 $b(y)$ 的真值:
		\begin{itemize}
			\item 如果 $b(y) = \text{False}$, 则截取 $y$ 中关于 $v_{ij}$ 的部分得到的赋值 $y'$ 可以满足 $w(y') = \text{True}$, 这是因为对于任意一组 $v_{i1}(y'), v_{i2}(y'), v_{i3}(y')$, 三者不可能同时为 $\text{False}$, 否则在 $y$ 赋值下 $v_{i1} \vee v_{i2} \vee{z_i}$, $\overline{z_i} \vee v_{i3} \vee b$ 这两个从句中会有恰好一个从句包含三个 False, 与 $\neq$-assignment 矛盾.
			\item 如果 $b(y) = \text{True}$, 则选取前述 $y'$ 的否定 $\lnot y'$ 可以满足 $w(\lnot y') = \text{True}$, 这是因为对于任意一组 $v_{i1}(\lnot y'), v_{i2}(\lnot y'), v_{i3}(\lnot y')$, 三者不可能同时为 $\text{False}$, 否则在 $y$ 赋值下 $v_{i1} \vee v_{i2} \vee{z_i}$, $\overline{z_i} \vee v_{i3} \vee b$ 这两个从句中会有恰好一个从句包含三个 True, 与 $\neq$-assignment 矛盾.
		\end{itemize}
		故 $w \in \textsf{3SAT}$.
		
	\end{enumerate}

	$f$ 显然是多项式时间可计算的, 因此说明了 $\textsf{3SAT} \le_p \neq\textsf{SAT}$.

	\item 由 Cook-Levin Theorem 知 $\textsf{3SAT} \in \textbf{NP}$-complete, 即 $\forall L \in \textbf{NP}, L \le_p \textsf{3SAT}$. 而前面我们证明了 $\textsf{3SAT} \le_p \neq\textsf{SAT}$, 因此由 $\le_p$ 的传递性可知 $\forall L \in \textbf{NP}, L \le_p \neq\textsf{SAT}$, 即 $\neq\textsf{SAT} \in \textbf{NP}$-complete.
\end{enumerate}


\end{document}
