\documentclass[8pt]{article}
\usepackage[UTF8]{ctex}
\usepackage[a4paper]{geometry}

\usepackage{amsthm,amsmath,amssymb}
\usepackage{graphicx}
\usepackage{subfigure}
\usepackage{amsmath}
\usepackage{tabularx}
\usepackage{color}
\usepackage{hyperref}
\usepackage{ulem}
\usepackage{multirow}
\usepackage[cache=false]{minted}
\hypersetup{
	colorlinks=true,
	linkcolor=blue
}

\usepackage{appendix}
\geometry{a4paper,centering,scale=0.8}
\geometry{left=2.0cm, right=2.0cm, top=2.5cm, bottom=2.5cm}
\usepackage[format=hang,font=small,textfont=it]{caption}
\usepackage[nottoc]{tocbibind}

\usepackage{algorithm}
\usepackage{algorithmicx}
\usepackage{algpseudocode}
\usepackage{amssymb}
\usepackage{extarrows}
\usepackage{qcircuit}
\usepackage{fancyhdr}
\usepackage{cleveref}
\usepackage{totpages}
\usepackage{pgf}
\usepackage{tikz}
\usetikzlibrary{arrows,automata}
\usetikzlibrary{arrows.meta}%画箭头用的包

\makeatletter
\def\@maketitle{%
	\newpage
	\begin{center}%
		\let \footnote \thanks
		{\LARGE \@title \par}%
		\vskip 1.5em%
		{\large
			\lineskip .5em%
			\begin{tabular}[t]{c}%
				\@author
			\end{tabular}\par}%
		\vskip 1em%
		{\large \@date}%
	\end{center}%
	\par
	\vskip 1.5em}
\makeatother

\newtheoremstyle{compact}%
{3pt}{3pt}%
{}{}%
{\bfseries}{\textcolor{red}{.}}%  % Note that final punctuation is omitted.
{.5em}{\mbox{\textcolor{red}{\thmname{#1}\thmnumber{ #2}}\thmnote{ (\textcolor{blue}{#3})}}}
\theoremstyle{compact}
\newtheorem{innercustomgeneric}{\customgenericname}
\providecommand{\customgenericname}{}
\newcommand{\newcustomtheorem}[2]{%
	\newenvironment{#1}[1]
	{%
		\renewcommand\customgenericname{#2}%
		\renewcommand\theinnercustomgeneric{##1}%
		\innercustomgeneric
	}
	{\endinnercustomgeneric}
}

\DeclareMathOperator{\card}{card}

\newtheorem{theorem}{定理}
\newtheorem{lemma}{引理}
\newtheorem{definition}{定义}
\newtheorem{proposition}{命题}
\newtheorem{corollary}{推论}
\newtheorem{remark}{注}
\newtheorem{Proof}{证明}

\def\obj#1{\textbf{\uline{#1}}}
\def\num#1{\textnormal{\textbf{\mbox{\textcolor{blue}{(#1)}}}}}
\def\le{\leqslant}
\def\ge{\geqslant}
\def\im{\text{im }}
\def\Pr#1{\text{Pr}\left[{#1}\right]}
\def\E#1{\mathbb{E}\left[{#1}\right]}
\def\Var#1{\text{Var}\left[{#1}\right]}
\def\rep#1{\llcorner{#1}\lrcorner}

\title{\heiti\zihao{1} 计算理论导论 \ 第六次作业}
\author{\kaishu\zihao{-3} 周书予\\2000013060@stu.pku.edu.cn}

\CTEXoptions[today=old]
\date{\today}

\begin{document}
\fancypagestyle{plain}{
	\fancyhf{}
	\lhead{计算理论导论}
	\chead{2022 Spring}
	\rhead{第六次作业}
	\cfoot{第 \thepage 页, 共 \pageref{TotPages} 页}
}
\pagestyle{plain}



\crefname{theorem}{定理}{定理}
\crefname{lemma}{引理}{引理}
\crefname{figure}{图}{图}
\crefname{table}{表}{表}	
\maketitle
\def\L{\textbf{L}}
\def\P{\textbf{P}}
\def\NC{\text{uni-}\textbf{NC}}
\section{}

$L \notin \NC \Rightarrow \P \neq \NC$ 是显然的, 因为此时 $L \in \P \setminus \NC$.

当 $L \in \NC$ 时, 考虑识别 $L$ 的, 规模为 $O(n^c)$ 深度为 $O(\log^d n)$ 的 logspace-uniform 线路 $\{C_n\}$. 任取 $L' \in \P$, 我们希望构造满足同样条件的线路 $\{D_n\}$ 识别语言 $L'$.

$L \in \P$-complete 说明存在 implicitly logspace reduction $f$ 满足 $x \in L' \Leftrightarrow f(x) \in L$. 根据定义存在图灵机 $M_{f, i}$ 可以在 $O(\log |x|)$ 空间内计算 $f(x)_i$, 考虑构造其 configuration graph $G_{M_{f, i}, x}$, 容易发现 $M_{f, i}(x) = 1 \Leftrightarrow \langle G_{M_{f, i}, x}, C_{\text{start}}, C_{\text{accept}} \rangle \in \textsf{PATH}$. 如果证明了 $\textsf{PATH} \in \NC$, 那么只需要如下地构造线路 $\{D_n\}$: 对于输入 $x$, 首先计算出 $|f(x)|$, 然后\obj{并行地}根据 $M_{f, i}$ 计算出 $f(x)$ 的每一位, 得到 $f(x)$, 再接上线路 $C_{|f(x)|}$, 得到 $$D_{|x|}(x) = C_{|f(x)|}(f(x)) = [f(x) \in L] = [x \in L']$$

由于 $\{D_n\}$ 的两部分(实现 \textsf{PATH}, 以及接上 $C_{|f(x)|}$)都是 \NC 的, 故其自身也是 \NC 的.

\begin{lemma}
	$\textsf{PATH} \in \NC$.
\end{lemma}
\begin{proof}
	考虑图 $G$ 的邻接矩阵 $A$. 记 $B^k_{i, j}$ 表示图 $G$ 中是否存在长度不超过 $2^k$ 的从 $i$ 到 $j$ 的路径, 有初值 $B^0 = A$, 以及递推关系 $$B^k_{i, j} = \bigvee_{l}B^{k-1}_{i,l} \wedge B^{k-1}_{l,j}$$

	因此若 $B^{k-1}$ 能被多项式规模, 深度为 $d$ 的线路计算, 则 $B^k$ 也能被多项式规模, 深度为 $d+2$ 的线路计算.

	取 $K = \lceil \log n \rceil$ 其中 $n$ 是图 $G$ 的点数, 则 $\langle G, s, t \rangle \in \textsf{PATH}$ 当且仅当 $B^K_{s, t} = 1$. 计算 $B^K$ 的线路是 poly-size, log-depth 以及 uniform 的, 故 $\textsf{PATH} \in \NC$.
\end{proof}
\section{}
如果 $\textsf{MAJ} \in \textbf{AC}^0$, 考虑构造计算 $\textsf{PARITY}$ 的 poly-size const-depth 线路.

假设任意 $\textsf{MAJ}_n$ 都可以在 $O(n^c)$ 规模, $d$ 深度内计算, 考虑记 $A_i = \{x | x \text{ has at least } i \text{ 1s }\}$, 根据输入规模 $n$ 适当补充输入的 0 或 1 的数量便可以由 \textsf{MAJ} 的线路构造出 $A_i$ 的线路.

注意到 $$\textsf{PARITY}_n = (A_0 \wedge \overline{A_1}) \vee (A_2 \wedge \overline{A_3}) \vee \cdots \vee (A_{2k} \wedge \overline{A_{2k+1}})$$
其中 $2k + 1 \ge n$. 故一旦构造出了 $A_i$ 的 poly-size const-depth 线路, 便也可以构造出 \textsf{PARITY} 的 poly-size const-depth 线路, 即 $\textsf{PARITY} \in \textbf{AC}^0$.

由 Switching Lemma 知 $\textsf{PARITY} \notin \textbf{AC}^0$, 产生矛盾. 故 $\textsf{MAJ} \notin \textbf{AC}^0$.
\section{}
\begin{enumerate}
	\item 取 $k = \lceil \log_2 N \rceil$, 随机算法 $\mathcal A$ 消耗 $k$ 个随机比特, 即以 $k$ 位 01 串作为输入, 且满足 $$\mathcal A(r) = \begin{cases}
		\langle r \rangle + 1, & \langle r \rangle < N \\
		? , & \langle r \rangle \ge N
	\end{cases}$$
	其中 $\langle r \rangle$ 表示将 $r$ 看成一个 $k$ 位二进制数得到的非负整数.

	对于任意 $k = 1, 2, \cdots, N$, 有 $$\mathbb P(\mathcal A(r) = k | \mathcal A(r) \neq ?) = \frac{\mathbb P(\mathcal A(r) = k)}{\mathbb P(\mathcal A(r) \neq ?)} = \frac{2^{-k}}{N \cdot 2^{-k}} = \frac 1N$$
	故 $\mathcal A$ 的输出是 $[N]$ 的均匀分布. 显然 $\mathcal A$ 的运行时间是 $O(\log N)$.

	\item 仍然取 $k = \lceil \log_2 N \rceil$, 随机算法 $\mathcal B$ 消耗 $k\cdot \lceil \log_2(1 / \delta)\rceil$ 个随机比特, 满足 $$\mathcal B(r_1, \cdots, r_k) = \begin{cases}
		\langle r_i \rangle + 1, & \langle r_1 \rangle, \cdots, \langle r_{i-1} \rangle \ge N, \langle r_i \rangle < N \\
		?, & \langle r_1 \rangle, \cdots, \langle r_{\lceil\log_2(1 / \delta)\rceil} \rangle \ge N
	\end{cases}$$
	容易验证 $\mathcal B$ 的输出也是 $[N]$ 的均匀分布, 且 $\mathcal B$ 的运行时间是 $O(\log N \log(1/\delta))$. 分析 $\mathcal B$ 输出 $?$ 的概率: $$\mathbb P(\mathcal B(r_1, \cdots, r_k) = ?) = \left(\frac{2^k - N}{2^k}\right)^{\lceil \log_2(1 / \delta)\rceil} \le \left(\frac12\right)^{\lceil \log_2(1 / \delta)\rceil} \le \left(\frac12\right)^{\log_2(1 / \delta)} = \delta$$
\end{enumerate}
\section{}
只需要证明 $L \le_r \textsf{3SAT} \Rightarrow $ 存在 poly-size nondeterministic circuit $\{C_n\}$ 可以识别 $L$.

$L \le_r \textsf{3SAT}$ 说明存在多项式时间图灵机 $M$ 满足 $\mathbb P_{r \in \{0, 1\}^{m(n)}}[L(x) = \textsf{3SAT}(M(x, r))] \ge 2/3$, 其中 $n = |x|, m(n) = \text{poly}(n)$ 是 $M$ 消耗的随机比特数量. 根据 error reduction 的结论, 我们知道也存在多项式时间图灵机 $M'$ 满足 $\mathbb P_{r \in \{0, 1\}^{m'(n)}}[L(x) \neq \textsf{3SAT}(M'(x, r))] \le 2^{-n-1}$, 其中 $m'(n) = \text{poly}(n)$.

根据 Union Bound, \begin{align*}
	\begin{split}
		&\mathbb P_{r \in \{0, 1\}^{m'(n)}} \left(\exists x \in \{0, 1\}^n, L(x) \neq \textsf{3SAT}(M'(x, r))\right) \\
		\le & \sum_{x \in \{0, 1\}^n}\mathbb P_{r \in \{0, 1\}^{m'(n)}}[L(x) \neq \textsf{3SAT}(M'(x, r))] \\
		\le & \ 2^n \cdot 2^{-n-1} = 1/2 < 1
	\end{split}
\end{align*}

故存在 $r_n \in \{0, 1\}^{m'(n)}$ 使得 $L(x) = \textsf{3SAT}(M'(x, r_n))$ 对任意 $x \in \{0, 1\}^n$ 成立.

于是 $x \in L \Leftrightarrow \exists r_n, M'(x, r_n) \in \textsf{3SAT} \Leftrightarrow \exists r_n, \exists \text{ assignment } u, M'(x, r_n)(u) = \text{True}$. 构造线路 $C_n$ 接受输入 $x, r_n, u$, 根据 $M'$ 计算出 3CNF $M'(x, r_n)$, 再检验赋值 $u$ 是否是可满足赋值. $C_n$ 显然是多项式规模的, 因此 $L \in \textbf{NP}_{\textbf{/poly}}$.
\section{}
\def\BPL{\textbf{BPL}}
任取 $L \in \BPL$, 存在 $O(\log n)$ 空间的 PTM $M$ 使得 $\mathbb P[M(x) = L(x)] \ge 2/3$. 对于输入 $x$, 考虑 configuration graph $G_{M, x}$, 其有 $m = \text{poly}(n)$ 个点, 每条转移边都有 $0/0.5/1$ 的转移概率, 可以写成一个转移矩阵 $A \in \mathbb R^{m \times m}$.

利用矩阵乘法可以计算出 $t$ 步的转移概率(即 $A^t_{i, j}$ 表示从点 $i$ 出发走 $t$ 步, 走到点 $j$ 的概率), 其中 $t = \text{poly}(n)$ 为 $M$ 的运行时间. 此时 $x \in L \Leftrightarrow A^t_{q_{\text{start}}, q_{\text{accept}}} \ge 2/3$, 可以多项式时间地将 $A^t$ 算出, 故 $L \in \P$.

\end{document}
