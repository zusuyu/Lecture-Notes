\documentclass[8pt]{article}
\usepackage[UTF8]{ctex}
\usepackage[a4paper]{geometry}

\usepackage{amsthm,amsmath,amssymb}
\usepackage{graphicx}
\usepackage{subfigure}
\usepackage{amsmath}
\usepackage{tabularx}
\usepackage{color}
\usepackage{hyperref}
\usepackage{ulem}
\usepackage{multirow}
\usepackage[cache=false]{minted}
\hypersetup{
	colorlinks=true,
	linkcolor=blue
}

\usepackage{appendix}
\geometry{a4paper,centering,scale=0.8}
\geometry{left=2.0cm, right=2.0cm, top=2.5cm, bottom=2.5cm}
\usepackage[format=hang,font=small,textfont=it]{caption}
\usepackage[nottoc]{tocbibind}

\usepackage{algorithm}
\usepackage{algorithmicx}
\usepackage{algpseudocode}
\usepackage{amssymb}
\usepackage{extarrows}
\usepackage{qcircuit}
\usepackage{fancyhdr}
\usepackage{cleveref}
\usepackage{totpages}
\usepackage{pgf}
\usepackage{tikz}
\usetikzlibrary{arrows,automata}
\usetikzlibrary{arrows.meta}%画箭头用的包

\makeatletter
\def\@maketitle{%
	\newpage
	\begin{center}%
		\let \footnote \thanks
		{\LARGE \@title \par}%
		\vskip 1.5em%
		{\large
			\lineskip .5em%
			\begin{tabular}[t]{c}%
				\@author
			\end{tabular}\par}%
		\vskip 1em%
		{\large \@date}%
	\end{center}%
	\par
	\vskip 1.5em}
\makeatother

\newtheoremstyle{compact}%
{3pt}{3pt}%
{}{}%
{\bfseries}{\textcolor{red}{.}}%  % Note that final punctuation is omitted.
{.5em}{\mbox{\textcolor{red}{\thmname{#1}\thmnumber{ #2}}\thmnote{ (\textcolor{blue}{#3})}}}
\theoremstyle{compact}
\newtheorem{innercustomgeneric}{\customgenericname}
\providecommand{\customgenericname}{}
\newcommand{\newcustomtheorem}[2]{%
	\newenvironment{#1}[1]
	{%
		\renewcommand\customgenericname{#2}%
		\renewcommand\theinnercustomgeneric{##1}%
		\innercustomgeneric
	}
	{\endinnercustomgeneric}
}

\DeclareMathOperator{\card}{card}

\newtheorem{theorem}{定理}
\newtheorem{lemma}{引理}
\newtheorem{definition}{定义}
\newtheorem{proposition}{命题}
\newtheorem{corollary}{推论}
\newtheorem{remark}{注}
\newtheorem{Proof}{证明}

\def\obj#1{\textbf{\uline{#1}}}
\def\num#1{\textnormal{\textbf{\mbox{\textcolor{blue}{(#1)}}}}}
\def\le{\leqslant}
\def\ge{\geqslant}
\def\im{\text{im }}
\def\Pr#1{\text{Pr}\left[{#1}\right]}
\def\E#1{\mathbb{E}\left[{#1}\right]}
\def\Var#1{\text{Var}\left[{#1}\right]}
\def\rep#1{\llcorner{#1}\lrcorner}

\title{\heiti\zihao{1} 计算理论导论 \ 第七次作业}
\author{\kaishu\zihao{-3} 周书予\\2000013060@stu.pku.edu.cn}

\CTEXoptions[today=old]
\date{\today}

\begin{document}
\fancypagestyle{plain}{
	\fancyhf{}
	\lhead{计算理论导论}
	\chead{2022 Spring}
	\rhead{第七次作业}
	\cfoot{第 \thepage 页, 共 \pageref{TotPages} 页}
}
\pagestyle{plain}



\crefname{theorem}{定理}{定理}
\crefname{lemma}{引理}{引理}
\crefname{figure}{图}{图}
\crefname{table}{表}{表}	
\maketitle

\def\IP{\textbf{IP}}
\def\IPP{\textbf{IP'}}
\def\PSPACE{\textbf{PSPACE}}

\section{}
我们知道 $\IP = \PSPACE$. 在 $\PSPACE \subseteq \IP$ 的证明中, 我们构造了一种基于多项式求值的, 用于证明 $\textsf{TQBF} \in \IP$ 的验证协议, 这种协议对于正确的实例 $\Psi \in \textsf{TQBF}$ 一定会接受, 即具有 perfect completeness. 也就是说, 我们证明的实际上是 $\PSPACE \subseteq \IPP$.

注意到又有 $\IPP \subseteq \IP \subseteq \PSPACE$, 因此 $\IP = \IPP$.

\def\MIP{\textbf{MIP}}
\def\NEXP{\textbf{NEXP}}

\section{}
任取 $L \in \MIP$, 存在多项式时间图灵机(verifier) $V_1, V_2$ 使得 \begin{align*}
	\begin{split}
		x &\in L \Rightarrow \exists P_1, P_2, \mathbb P_r[\textsf{out}(V_1, V_2, P_1, P_2, x, r) = 1] \ge \frac23 \\
		x &\notin L \Rightarrow \forall P_1, P_2, \mathbb P_r[\textsf{out}(V_1, V_2, P_1, P_2, x, r) = 1] \le \frac13
	\end{split}
\end{align*}
其中 $\textsf{out}(V_1, V_2, P_1, P_2, x, r)$ 表示 verifier $V_1, V_2$ 与 multiprover $P_1, P_2$ 基于输入 $x$ 与随机串 $r$ 进行交互验证的输出结果.

考虑构造 NDTM $M$, 其先利用 nondeterminism \obj{搜索}出两个 prover 图灵机 $P_1, P_2$, 再依次\obj{枚举}随机串 $r$ 的所有可能, 然后确定性地计算 $\textsf{out}(V_1, V_2, P_1, P_2, x, r)$, 并统计其中等于 $1$ 的占比, 如果这个占比超过 $\frac23$ 就接受. 换言之, $M$ 接受 $x$ 当且仅当存在 $P_1, P_2$ 使得 $\mathbb P_r[\textsf{out}(V_1, V_2, P_1, P_2, x, r) = 1] \ge \frac23$, 故 $L = L(M)$.

由于 $P_1, P_2$ 本身都是 $\{0, 1\}^{p(|x|)} \to \{0, 1\}^{q(|x|)}$ 的函数, $p, q$ 是多项式, 因此 $P_1, P_2$ 可以用 $q(|x|) \cdot 2^{p(|x|)}$ 位二进制串来表示, 即关于输入规模指数级. $r$ 的规模也是多项式, 因此枚举 $r$ 所花费的时间也是指数级的.

因此 $M$ 的运行时间是关于输入规模指数级, 故 $L = L(M) \in \NEXP$.

\def\AM{\textbf{AM}}
\def\BPNP{\textbf{BP}\cdot\textbf{NP}}
\def\NP{\textbf{NP}}
\section{}

\begin{itemize}
	\item $\AM[2] \subseteq \BPNP$.
	
	任取 $L \in \AM[2]$, 存在多项式时间 verifier $V$, 其在交互证明过程中先向 prover 发送随机串 $r$, 得到对方反馈 $a = P(x, r)$ 后, 输出交互证明结果 $V(x, r, a)$, 满足 \begin{align*}
		\begin{split}
			x &\in L \Rightarrow \exists P, \mathbb P_r[V(x, r, P(x, r)) = 1] \ge \frac23 \\
			x &\notin L \Rightarrow \forall P, \mathbb P_r[V(x, r, P(x, r)) = 1] \le \frac13 \\
		\end{split}
	\end{align*}
	考虑语言 $L' = \{(x, r) | \exists a, V(x, r, a) = 1\} \in \NP$, 从而有多项式时间规约 $f$ 满足 $(x, r) \in L' \Leftrightarrow \phi_{x, r} = f(x, r) \in \textsf{3SAT}$. 注意到 $V(x, r, P(x, r)) = 1 \Rightarrow \phi_{x, r} \in \textsf{3SAT}$, 从而可以进一步得到 \begin{align*}
		\begin{split}
			\exists P, \mathbb P_r[V(x, r, P(x, r)) = 1] \ge \frac23 & \Rightarrow \mathbb P_r[\phi_{x, r} \in \textsf{3SAT}] \ge \frac23\\
			\forall P, \mathbb P_r[V(x, r, P(x, r)) = 1] \le \frac13 & \Rightarrow \mathbb P_r[\phi_{x, r} \in \textsf{3SAT}] \le \frac13\\
		\end{split}
	\end{align*}
	即 $L \le_r \textsf{3SAT}$, 说明 $L \in \BPNP$.

	\item $\BPNP \subseteq \AM[2]$.
	
	任取 $L \in \BPNP$, 存在多项式时间可计算函数 $f$, 记 $\phi_{x, r} = f(x, r)$, 则有 \begin{align*}
		\begin{split}
			x &\in L \Rightarrow \mathbb P_r[\phi_{x, r} \in \textsf{3SAT}] \ge \frac23 \\
			x &\notin L \Rightarrow \mathbb P_r[\phi_{x, r} \in \textsf{3SAT}] \le \frac13 \\
		\end{split}
	\end{align*} 可以设计如下的交互式证明协议: verifier 发送随机串 $r$, prover 根据 $f$ 计算出 $\phi_{x, r}$ 并返回其一组可满足赋值 $u$, verifier 检验 $u$ 的满足性并输出结果.

	不难验证这个协议可以用于识别语言 $L$. 因此 $L \in \AM[2]$.

\end{itemize}

\end{document}
